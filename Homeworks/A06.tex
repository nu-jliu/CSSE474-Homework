\documentclass[11pt]{article}

\usepackage{enumitem}


%%TO EDIT
\newcommand{\dueclassnumber}{15}
\newcommand{\assignmentnum}{6}

% CHANGE issolution{0} to issolution{1} for homework submission.
% WRITE solutions inside the \solution{} commands.
% or, you can use \if\issolution1 … \fi

\def\issolution{1}
\def\myname{Allen Liu} % My name goes here
\def\mysec{01} % Section number goes here
\def\myCM{374} % Campus Mailbox goes here

\input{../common/flags}
\newcommand\vv[2]{\begin{bmatrix} {#1} \cr {#2} \end{bmatrix}}

% Leave the next line alone. This is for my answer keys.
%\def\isanswerkey{1}

\usepackage{bbm,fancyhdr,ifthen,setspace,hyperref,url}
\usepackage{amssymb,amsmath,enumitem,amsthm,mathrsfs}
\usepackage{graphicx,xspace,color}
\usepackage{hhline}
\usepackage{tikz}
\usetikzlibrary{automata, positioning, arrows,chains,scopes,fit}
\tikzset{
->, % makes the edges directed
>=stealth', % makes the arrow heads bold
node distance=2.4cm, % specifies the minimum distance between two nodes. Change if necessary.
every state/.style={thick, fill=gray!10}, % sets the properties for each ’state’ node
initial text=$ $, % sets the text that appears on the start arrow
}

\topmargin=-.5in
\headsep=0.0in
\oddsidemargin=-.35in
\evensidemargin=-.65in
\textwidth=7.25in
\textheight=9.75in
\footskip=0in
\usepackage{titlesec}
\titlespacing*{\paragraph}{0pt}{2ex plus 1ex minus .2ex}{1ex}
\fancyhf{} % clear all header and footers
\renewcommand{\headrulewidth}{0pt} % remove the header rule
%\rfoot{\thepage}
\pagestyle{fancy}

\ifx\myname\undefined
\def\myname{}
\fi

\ifx\isanswerkey\undefined
\def\isanswerkey{0}
\fi

\if\isanswerkey1
\def\issolution{1}
\fi


\newcommand{\getcourseyear}[1]{2022}
\newcommand{\getcourseterm}{Spring \getcourseyear{1}}

\newcommand{\getclassmonthnum}[1]{\ifthenelse{#1<16}{3}{\ifthenelse{#1<29}{4}{5}}}
\newcommand{\getclassmonthshort}[1]{\ifthenelse{#1<16}{Mar}{\ifthenelse{#1<29}{Apr}{May}}}
\newcommand{\getclassmonth}[1]{\ifthenelse{#1<16}{March}{\ifthenelse{#1<29}{April}{May}}}
\newcommand{\getclassdayofmonth}[1]{\ifthenelse{
#1=1}{7}{\ifthenelse{
#1=2}{8}{\ifthenelse{
#1=3}{10}{\ifthenelse{
#1=4}{11}{\ifthenelse{
#1=5}{14}{\ifthenelse{
#1=6}{15}{\ifthenelse{
#1=7}{17}{\ifthenelse{
#1=8}{18}{\ifthenelse{
#1=9}{21}{\ifthenelse{
#1=10}{22}{\ifthenelse{
#1=11}{24}{\ifthenelse{
#1=12}{25}{\ifthenelse{
#1=13}{28}{\ifthenelse{
#1=14}{29}{\ifthenelse{
#1=15}{31}{\ifthenelse{
#1=16}{1}{\ifthenelse{
#1=17}{4}{\ifthenelse{
#1=18}{5}{\ifthenelse{
#1=19}{7}{\ifthenelse{
#1=20}{8}{\ifthenelse{
#1=21}{18}{\ifthenelse{
#1=22}{19}{\ifthenelse{
#1=23}{21}{\ifthenelse{
#1=24}{22}{\ifthenelse{
#1=25}{25}{\ifthenelse{
#1=26}{26}{\ifthenelse{
#1=27}{28}{\ifthenelse{
#1=28}{29}{\ifthenelse{
#1=29}{2}{\ifthenelse{
#1=30}{3}{\ifthenelse{
#1=31}{5}{\ifthenelse{
#1=32}{6}{\ifthenelse{
#1=33}{9}{\ifthenelse{
#1=34}{10}{\ifthenelse{
#1=35}{12}{\ifthenelse{
#1=36}{13}{\ifthenelse{
#1=37}{16}{\ifthenelse{
#1=38}{17}{\ifthenelse{
#1=39}{19}{\ifthenelse{
#1=40}{20}{}}}}}}}}}}}}}}}}}}}}}}}}}}}}}}}}}}}}}}}}}

\newcommand{\getclassdate}[1]{\getclassmonth{#1}\xspace\getclassdayofmonth{#1}}
\newcommand{\getclassdateshort}[1]{\getclassmonthshort{#1}\xspace\getclassdayofmonth{#1}}
\newcommand{\getclassdatenum}[1]{\getclassmonthnum{#1}/\getclassdayofmonth{#1}}
\newcommand{\getMday}{Mon}
\newcommand{\getTday}{Tue}
\newcommand{\getWday}{Wed}
\newcommand{\getRday}{Thu}
\newcommand{\getFday}{Fri}
\newcommand{\getclassdayofweek}[1]{\ifthenelse{
#1=1}{\getMday}{\ifthenelse{#1=2}{\getTday}{\ifthenelse{#1=3}{\getRday}{\ifthenelse{#1=4}{\getFday}{\ifthenelse{
#1=5}{\getMday}{\ifthenelse{#1=6}{\getTday}{\ifthenelse{#1=7}{\getRday}{\ifthenelse{#1=8}{\getFday}{\ifthenelse{
#1=9}{\getMday}{\ifthenelse{#1=10}{\getTday}{\ifthenelse{#1=11}{\getRday}{\ifthenelse{#1=12}{\getFday}{\ifthenelse{
#1=13}{\getMday}{\ifthenelse{#1=14}{\getTday}{\ifthenelse{#1=15}{\getRday}{\ifthenelse{#1=16}{\getFday}{\ifthenelse{
#1=17}{\getMday}{\ifthenelse{#1=18}{\getTday}{\ifthenelse{#1=19}{\getRday}{\ifthenelse{#1=20}{\getFday}{\ifthenelse{
#1=21}{\getMday}{\ifthenelse{#1=22}{\getTday}{\ifthenelse{#1=23}{\getRday}{\ifthenelse{#1=24}{\getFday}{\ifthenelse{
#1=25}{\getMday}{\ifthenelse{#1=26}{\getTday}{\ifthenelse{#1=27}{\getRday}{\ifthenelse{#1=28}{\getFday}{\ifthenelse{
#1=29}{\getMday}{\ifthenelse{#1=30}{\getTday}{\ifthenelse{#1=31}{\getRday}{\ifthenelse{#1=32}{\getFday}{\ifthenelse{
#1=33}{\getMday}{\ifthenelse{#1=34}{\getTday}{\ifthenelse{#1=35}{\getRday}{\ifthenelse{#1=36}{\getFday}{\ifthenelse{
#1=37}{\getMday}{\ifthenelse{#1=38}{\getTday}{\ifthenelse{#1=39}{\getRday}{\ifthenelse{#1=40}{\getFday}\xspace
}}}}}}}}}}}}}}}}}}}}}}}}}}}}}}}}}}}}}}}}
\newcommand{\examdate}[1]{\ifthenelse{#1=1}{March 23, \getcourseyear{11}}{\ifthenelse{#1=2}{April 7, \getcourseyear{11}}{\ifthenelse{#1=3}{May 4, \getcourseyear{11}}{\ifthenelse{#1=4}{ } {}}}}}




\newcommand{\classdate}[1]{\getclassdate{#1}, \getcourseyear{#1}}
\newcommand{\dueclassdate}{\getclassdayofweek{\dueclassnumber} \getclassdateshort{\dueclassnumber}}


\newcommand\largeemptyspace{\vphantom{\textnormal{$\ds\int$}}}
\newcommand\nameblank{\if\issolution0\underline{\hskip11.25cm {\largeemptyspace}}
  \else\underline{\hskip.2cm{\LARGE\myname}\hskip6cm}\fi}
\newcommand\nameblankshort{\if\issolution0\underline{\hskip9.5cm {\largeemptyspace}}
  \else\underline{\hskip.2cm{\LARGE\myname {\largeemptyspace}}\hskip.2cm}\fi}
\newcommand\secblank{\if\issolution0\underline{\hskip1.5cm{\largeemptyspace}}
  \else\underline{\hskip.2cm{\LARGE\mysec {\largeemptyspace}}\hskip.2cm} \hskip.8cm \fi}
\newcommand\CMblank{\if\issolution0\underline{\hskip2.25cm{\largeemptyspace}}
  \else\underline{\hskip.2cm{\LARGE\myCM {\largeemptyspace}}\hskip.2cm}\fi}

\newcommand{\namegroupline}{Name: \nameblank Group \#: \underline{\hskip1.5cm{\largeemptyspace}}}
\newcommand{\nameline}{\begin{minipage}{0.6\linewidth} Name: \nameblank \end{minipage}}
\newcommand{\namelineshort}{Name: \nameblankshort}
\newcommand{\namesecline}{\begin{minipage}{0.7\linewidth} Name: \nameblank \end{minipage} \hfill \begin{minipage}{0.29\linewidth}Section \#: \secblank\end{minipage}}
\newcommand{\namesecCMline}{\begin{minipage}{0.6\linewidth}Name: \nameblankshort \end{minipage} \hfill \begin{minipage}{0.4\linewidth} Section \# \secblank CM\# \CMblank \end{minipage}}
\newcommand{\keyline}{{\color{red} SOLUTION KEY}}
%\newcommand{\nameline}{Name: \rule{11.5cm}{0.01cm} \hfill Section: \rule{1.5cm}{0.01cm}}
%\newcommand{\keyline}{Name: \rule{4cm}{0.01cm} SOLUTION KEY \rule{4cm}{0.01cm} \hfill Section: \rule{1.5cm}{0.01cm}}
%\newcommand{\groupline}{Group members present: \rule{8.5cm}{0.01cm} \hfill Group \#: \rule{1.5cm}{0.01cm}}
\newcommand{\course}{CSSE/MA 474\xspace}
\newcommand{\coursewithname}{CSSE/MA 474. Theory of Computation\xspace}


\newcommand{\wtitlestuff}{
\if\isanswerkey0
  \nameline
  \else
  \keyline
\fi
\begin{center}
\large \course Worksheet for Class \#\classnumber\\
\small \classdate{\classnumber}
\normalsize
\end{center}}

\newcommand{\lectitlestuff}{
\begin{center}
\Large \course Lecture \#\classnumber\\
\vskip 3pt \small Nate Chenette \\ \classdate{\classnumber}
\normalsize
\end{center}}

\newcommand{\othertitlestuff}{
\begin{center}
\Large \othertitle\\
\small \coursewithname\\
Class \#\classnumber, \classdate{\classnumber}\\
\normalsize
\end{center}}

\newcommand{\othertitlestuffnodate}{
\begin{center}
\Large \othertitle\\
\small \coursewithname\\
\normalsize
\end{center}}

\newcommand{\othernametitlestuff}{
\nameline\\
\othertitlestuff
}

\newcommand{\assignmenttitlestuff}{
\begin{center}
\Large \course Assignment \assignmentnum\\
\small Due date: \dueclassdate
\normalsize
\end{center}}

\newcommand{\assignmentnametitlestuff}{
\if\isanswerkey0
  \namesecCMline
  \else
  \keyline
\fi
\assignmenttitlestuff
}

\newcommand{\quiznametitlestuff}{
\if\isanswerkey0
  \namesecCMline
  \else
  \keyline
\fi
\begin{center}
\Large \course Quiz \quiznum\\
\small \classdate{\classnumber}
\normalsize
\end{center}}


\setlength{\parindent}{0in}
\setlength{\fboxsep}{.1in}

\renewcommand{\emptyset}{\varnothing}
\newcommand{\tvs}{\textvisiblespace}
\newcommand{\brk}{\vskip.2cm \hrule \vskip.2cm}
\newcommand{\ds}{\displaystyle}
\newcommand{\abs}[1]{\left\lvert {#1}\right\rvert}
\newcommand{\Lsym}{\text{L}}
\newcommand{\Rsym}{\text{R}}
\newcommand{\qacc}{q_{\textnormal{accept}}}
\newcommand{\qrej}{q_{\textnormal{reject}}}
\newcommand{\tmRej}{$\to$ \textbf{\textit{reject}}}
\newcommand{\tmAcc}{$\to$ \textbf{\textit{accept}}}
\def\lep{\le_\textnormal{P}}
\def\lem{\le_\textnormal{m}}
\def\ATM{A_\textnormal{TM}}
\newcommand{\vv}[2]{\begin{bmatrix} {#1} \cr {#2} \end{bmatrix}}
\newcommand{\vvt}[2]{\begin{bmatrix} {\tt #1} \cr {\tt #2} \end{bmatrix}}
\def\hs{\quad \texttt\#\quad }
\def\multiset#1#2{\ensuremath{\left(\kern-.3em\left(\genfrac{}{}{0pt}{}{#1}{#2}\right)\kern-.3em\right)}}
\def\time{\textsf{TIME}}
\def\ntime{\textsf{NTIME}}
\def\P{\textsf{P}}
\def\NP{\textsf{NP}}
   
%logic
\newcommand{\se}{\big|}
\newcommand{\lra}{\leftrightarrow}
\newcommand{\Lra}{\Leftrightarrow}
\newcommand{\we}{\wedge}
\def\thf{%
   \leavevmode
   \lower0.2ex\hbox{$\cdot$}%
   \kern-0.0em\raise0.7ex\hbox{$\cdot$}%
   \kern-0.0em\lower0.2ex\hbox{$\cdot$}%
   \thinspace}

%Number Systems
\newcommand{\bbZ}{\mathbb{Z}}
\newcommand{\bfZ}{\mathbf{Z}}
\newcommand{\bfZp}{\mathbf{Z}^+}
\newcommand{\bbN}{\mathbb{N}}
\newcommand{\bfN}{\mathbf{N}}
\newcommand{\bbQ}{\mathbb{Q}}
\newcommand{\bfQ}{\mathbf{Q}}
\newcommand{\bbR}{\mathbb{R}}
\newcommand{\bfR}{\mathbf{R}}
\newcommand{\bbC}{\mathbb{C}}
\newcommand{\bfC}{\mathbf{C}}

%sets
\newcommand{\U}{\mathscr{U}}
\newcommand{\ol}[1]{\overline{#1}}
\newcommand{\ssq}{\subseteq}
\newcommand{\sst}{\subset}
\def\ps{\mathcal{P}}
\def\sd{\,\triangle\,}
\def\sdonly{\triangle}
\def\es{\emptyset}

%cards
\def\hst{\heartsuit}
\def\cst{\clubsuit}
\def\sst{\spadesuit}
\def\dst{\diamondsuit}


\newcommand{\lcm}{{\rm lcm}}


\newcommand{\imgdir}{../images/}


\newcommand{\makeexamcover}{
\ifdefined\finalexam
\ \vskip2cm
\begin{center}
\huge \course Final Exam \\
\Large \finalexamdate \vskip1cm
\end{center}
\normalsize \instructions \vskip1cm
\begin{spacing}{1.5}
\begin{center}
\scorechart
\end{center}
\end{spacing}
\else \ifdefined\examnum
\ \vskip2cm
\begin{center}
\huge \course Exam \examnum \\
\Large \examdate{\examnum} \vskip1cm
\end{center}
\normalsize \instructions \vskip1cm
\begin{spacing}{1.5}
\begin{center}
\scorechart
\end{center}
\end{spacing}
\fi
\fi
}




\fancypagestyle{examcover}{% 
\fancyhf{}
\renewcommand{\footrulewidth}{0pt}
\lhead{\if\isanswerkey1{\keyline}\else{\nameline}\fi}
%\lhead{\if\isanswerkey1{\keyline}\else{\namesecline}\fi}
}



\fancypagestyle{exameverypage}{% 
\fancyhf{}
\renewcommand{\footrulewidth}{0pt}
\rhead{\if\isanswerkey1{\keyline}\else{}\fi}
\fancyfoot[R]{\thepage}
%\lhead{\if\isanswerkey1{\keyline}\else{\namesecline}\fi}
}

\newcommand{\definition}[1]{{\sc Definition}.~~{#1}\vskip.2cm}

\usepackage[framemethod=default]{mdframed}
\global\mdfdefinestyle{red1}{linecolor=red, linewidth=1pt, leftmargin=1cm, rightmargin=1cm}
\global\mdfdefinestyle{black1}{linecolor=black, linewidth=1pt,} %leftmargin=.1cm, rightmargin=.1cm}

\newcommand{\solution}[2][]{\if\issolution0 #1 \else \begin{mdframed}[style=black1] #2 \end{mdframed} \fi}

\newcommand{\cmblanka}[1]{\if\issolution0 	\underline{\hskip1cm{\largeemptyspace}}
\else 						  		\underline{\hskip.35cm {#1}\hskip.35cm{\largeemptyspace}}\fi}
\newcommand{\sblanka}[1]{\if\issolution0 		\underline{\hskip1.5cm{\largeemptyspace}}
\else 						  		\underline{\hskip.25cm {#1}\hskip.25cm{\largeemptyspace}}\fi}
\newcommand{\mblanka}[1]{\if\issolution0 	\underline{\hskip3cm{\largeemptyspace}}
\else 						  		\underline{\hskip.5cm {#1}\hskip.5cm{\largeemptyspace}}\fi}
\newcommand{\lblanka}[1]{\if\issolution0 		\underline{\hskip4.5cm{\largeemptyspace}}
\else 						  		\underline{\hskip.75cm {#1}\hskip.75cm{\largeemptyspace}}\fi}
\newcommand{\Lblanka}[1]{\if\issolution0		\underline{\hskip6cm{\largeemptyspace}}
\else 								\underline{\hskip1cm {#1}\hskip1cm{\largeemptyspace}}\fi}
\newcommand{\LLblanka}[1]{\if\issolution0	\underline{\hskip7.5cm{\largeemptyspace}}
\else 								\underline{\hskip1.25cm {#1}\hskip1.25cm{\largeemptyspace}}\fi}
\newcommand{\LLLblanka}[1]{\if\issolution0 	\underline{\hskip9cm{\largeemptyspace}}
\else 								\underline{\hskip1.5cm {#1}\hskip1.5cm{\largeemptyspace}}\fi}
\newcommand{\tinyspacea}[1]{\if\issolution0 	\hskip.2cm{\largeemptyspace}
\else 						  		{#1}{\largeemptyspace}\fi}
\newcommand{\cmspacea}[1]{\if\issolution0 	\hskip1cm{\largeemptyspace}
\else 						  		\hskip.15cm {#1}\hskip.15cm{\largeemptyspace}\fi}
\newcommand{\sspacea}[1]{\if\issolution0 		\hskip1.5cm{\largeemptyspace}
\else 						  		\hskip.25cm {#1}\hskip.25cm{\largeemptyspace}\fi}
\newcommand{\mspacea}[1]{\if\issolution0 	\hskip3cm{\largeemptyspace}
\else 						  		\hskip.25cm {#1}\hskip.25cm{\largeemptyspace}\fi}
\newcommand{\lspacea}[1]{\if\issolution0 		\hskip4.5cm{\largeemptyspace}
\else 						  		\hskip.25cm {#1}\hskip.25cm{\largeemptyspace}\fi}
\newcommand{\Lspacea}[1]{\if\issolution0 	\hskip6cm{\largeemptyspace}
\else 						  		\hskip.25cm {#1}\hskip.25cm{\largeemptyspace}\fi}


\newcommand{\sparagraph}[1]{\vskip-1cm\paragraph{#1}}

\if\isanswerkey1\input{macsse474-key}\fi


\usetikzlibrary{positioning}

\begin{document}

\assignmentnametitlestuff


\if\isanswerkey0
{\bf Please follow the homework guidelines from A01.}

\fi


\begin{enumerate}

\item (2.12) Convert the following CFG $G$ to an equivalent PDA, using the procedure given in Theorem 2.20 / Worksheet 13.
\begin{align*}
R &\to XRX \mid S \cr
S &\to {\tt a}T{\tt b} \mid {\tt b}T{\tt a} \cr
T &\to XTX \mid X \mid \epsilon \cr
X &\to {\tt a} \mid {\tt b}
\end{align*}
\solution{
\if\isanswerkey1\solConvertCFGToPDA\fi
\begin{tikzpicture}
\node[state, initial] (q0) {$q_0$};
\node[state, below of=q0] (q1) {$q_1$};
\node[state, below of=q1] (q2) {$q_{loop}$};
\node[state, below of=q2, accepting] (q3) {$q_f$};
\node[state, right of=q0] (q4) {$q_4$};
\node[state, right of=q4] (q5) {$q_5$};
\node[state, below of=q5] (q6) {$q_6$};
\node[state, right of=q6] (q7) {$q_7$};
\node[state, below of=q7] (q8) {$q_8$};
\node[state, right of=q8] (q9) {$q_9$};
\node[state, left of=q3] (q10) {$q_{10}$};
\node[state, left of=q10] (q11) {$q_{11}$};
\draw
(q0) edge[below, left] node{$\epsilon, \epsilon \to \$$} (q1)
(q1) edge[below, left] node{$\epsilon, \epsilon \to R$} (q2)
(q2) edge[above, above] node{$\epsilon, R \to X$} (q4)
(q4) edge[right, above] node{$\epsilon, \epsilon \to R$} (q5)
(q5) edge[below, below] node{$\epsilon, \epsilon \to X$} (q2)
(q2) edge[loop left] node{\makecell{$\epsilon, R \to S ~~~\epsilon, T \to \epsilon$ \\ $\epsilon, T \to X ~~~\epsilon, X \to a$ \\ ${\tt a}, {\tt a} \to \epsilon ~~~{\tt b}, {\tt b} \to \epsilon$ \\ $\epsilon, X \to {\tt b}$}} (q2)
(q2) edge[above, above] node{$\epsilon, S \to {\tt b}$} (q6)
(q6) edge[right, above] node{$\epsilon, \epsilon \to R$} (q7)
(q7) edge[below, below] node{$\epsilon, \epsilon \to {\tt a}$} (q2)
(q2) edge[above, above] node{$\epsilon, S \to {\tt a}$} (q8)
(q8) edge[right, above] node{$\epsilon, \epsilon \to R$} (q9)
(q9) edge[bend left, above] node{$\epsilon, \epsilon \to {\tt b}$} (q2)
(q2) edge[below, below] node{$\epsilon, T \to X$} (q10)
(q10) edge[left, below] node{$\epsilon, \epsilon \to T$} (q11)
(q11) edge[right, left] node{$\epsilon, \epsilon \to X$} (q2)
(q2) edge[below, right] node{$\epsilon, \$ \to \epsilon$} (q3)
;
\end{tikzpicture}
}


\item (2.15 and 2.16, modified) 
\begin{enumerate}
\item Give a counterexample to show that the following construction fails to prove that the class of context-free languages is closed under star. Let $A$ be a CFL that is generated by the CFG $G = (V,\Sigma,R,S)$. Add the new rules $S \to SS \mid \epsilon$ and call the resulting grammar $G'$. This grammar is supposed to generate $A^*$.
\solution{
\if\isanswerkey1\solBadProofCFLClosedUnderStar\fi
For the new Grammar $G'$, consider a CFG that contains rule rule $S \to {\tt 0}S{\tt 1} \mid \epsilon$, which gives a string $s={\tt 0}^n{\tt 1}^n$.Adding the rule $S \to SS \mid \epsilon$ will cause the grammar accept the string like ${\tt 001011}$, which is not in form of ${\left({\tt 0}^n{\tt 1}^n\right)}^*$. Hence it fails to prove that CFL is closed under star. 

}
\item Show that the class of context-free languages over an alphabet $\Sigma$ is closed under the regular operations: union, concatenation, and star. (You only need to briefly describe constructions of appropriate CFGs. Note that your solution for star could be a ``corrected" version of the failed idea from (a).)
\solution{
\if\isanswerkey1\solCFLsClosedUnderRegOps\fi
\subsubsection*{Union}
Suppose $A$ is a language that is generated by CFG $G_A=\left(V_A, \Sigma, R_A, S_A\right)$ and $B$ is a language that is generated by CFG $G_B=\left(V_B, \Sigma, R_B, S_B\right)$. Then, there is a CFG $G=\left(V, \Sigma, R, S\right)$ that generates language $A\cup B$. Where 
\begin{itemize}
    \item $V=V_A\cup V_B \cup \left\{S\right\}$ 
    \item $\Sigma$ are all same
    \item \begin{equation*}
        R(v)=\begin{cases}
            R_A(v) & v \in V_A\\
            R_B(v) & v \in V_B\\
            S \to S_A \mid S_B & v=S
        \end{cases}
    \end{equation*}
    \item $S$ is a new start variable
\end{itemize}
\subsubsection*{Concatenation}
Suppose $A$ is a language that is generated by CFG $G_A=\left(V_A, \Sigma, R_A, S_A\right)$ and $B$ is a language that is generated by CFG $G_B=\left(V_B, \Sigma, R_B, S_B\right)$. Then, there is a CFG $G=\left(V, \Sigma, R, S\right)$ that generates language $AB$. Where 
\begin{itemize}
    \item $V=V_A\cup V_B \cup \left\{S\right\}$ 
    \item $\Sigma$ are all same
    \item \begin{equation*}
        R(v)=\begin{cases}
            R_A(v) & v \in V_A\\
            R_B(v) & v \in V_B\\
            S \to S_AS_B & v=S
        \end{cases}
    \end{equation*}
    \item $S$ is a new start variable
\end{itemize}
\subsubsection*{Star}
Suppose $A$ is a language that is generated by CFG $G_A=\left(V_A, \Sigma, R_A, S_A\right)$Then, there is a CFG $G=\left(V, \Sigma, R, S\right)$ that generates language $A^*$. Where 
\begin{itemize}
    \item $V=V_A\cup \left\{S\right\}$ 
    \item $\Sigma$ are all same
    \item \begin{equation*}
        R(v)=\begin{cases}
            R_A(v) & v \in V_A\\
            S \to SS_A \mid \epsilon & v=S
        \end{cases}
    \end{equation*}
    \item $S$ is a new start variable
\end{itemize}
Hence, CFG is closed under operations union, concatenation and star.
}
\item Consider the following grammars:
\begin{center}
$G_\emptyset$: $S \to S$ \qquad $G_\epsilon$: $S \to \epsilon$ \qquad  $G_a$: $S \to a$
\end{center}
It is easy to see that $L(G_\emptyset) = \emptyset$, $L(G_\epsilon) = \{\epsilon\}$, and $L(G_a) = \{a\}$ for any $a \in \Sigma$. What can we conclude from this information, along with part (b)?
\solution{
\if\isanswerkey1\solImplicationOfCFLClosedUnderRegOps\fi
For a language $A=\left\{\Sigma^n\mid n\ge0\right\}$, it can be generated by a CFG, where $L\left(G\right)=A$
}

\end{enumerate}


\item (2.30ad) Use the pumping lemma for CFLs to show that the following languages are not context-free.
\begin{enumerate}
\item $A = \{{\tt 0}^n{\tt 1}^n{\tt 0}^n{\tt 1}^n \mid n \ge 0\}$
\solution{
\if\isanswerkey1\solNotContextFreeA\fi
Suppose language $A$ is a CFL. Let $p$ be the pumping length as in pumping lemma. Consider a string $s={\tt 0}^p{\tt 1}^p{\tt 0}^p{\tt 1}^p \in A$, where $|s|=4p\ge p$. Then the pumping lemma for CFL implies that $s=uvxyz$ as in pumping lemma. Since $|uxy|\le p$, $v$ and $y$ at most contains symbols in one group of ${\tt 0}$s and one group of ${\tt 1}$s. So for pumped string say $s'=uv^2xy^2z$, one group of ${\tt 0}$s or ${\tt 1}$s must have more symbols than other one. Then $s'\not\in A$. Hence, language $A$ is not a CFL. 
}

\item[(d)] $D = \{t_1\texttt{\#}t_2\texttt{\#}\cdots\texttt{\#}t_k \mid k \ge 2, \textnormal{ each } t_i \in \{{\tt a},{\tt b}\}^*, \textnormal{ and } t_i = t_j \textnormal{ for some } i \ne j\}$
\solution{
\if\isanswerkey1\solNotContextFreeD\fi
Suppose language $D$ is a CFL. Let $p$ be the pumping length as in pumping lemma. Consider a string $s={\tt a}^p{\tt b}^p\texttt{\#}{\tt a}^p{\tt b}^p \in D$. Then the pumping lemma for CFL implies that $s=uvxyz$ as in pumping lemma. Since $|uxy|\le p$, $v\in {\tt b}^+$ and $y\in {\tt a}^+$. So by pumping the string, say $s'=uv^2xy^2z$, it has more ${\tt a}$s in $t_1$ than that in $t_2$. So $t_1\ne t_2$, which gives $s'\not\in D$. Hence, language $D$ is not a CFL. 
}
\end{enumerate}
\end{enumerate}

% \subsection*{A07 problem}
% {\em The following problem will be due as part of A07. You can start on it now if you have time. Please do not submit it as part of A06.}
% \vskip.3cm

% (2.48) Let $\Sigma = \{0,1\}$. Let $C_1$ be the language of all strings that contain a {\tt 1} in their middle third. Let $C_2$ be the language of all strings that contain two {\tt 1}s in their middle third. In particular,
% \begin{align*}
% C_1 &= \{xyz \mid x,z \in \Sigma^* \textnormal{ and $y \in \Sigma^*{\tt 1}\Sigma^*$, where $|x| = |z| \ge |y|$}\} \cr
% C_2 &= \{xyz \mid x,z \in \Sigma^* \textnormal{ and $y \in \Sigma^*{\tt 1}\Sigma^*{\tt 1}\Sigma^*$, where $|x| = |z| \ge |y|$}\}.
% \end{align*}
% \begin{enumerate}
% \item Show that $C_1$ is a CFL.
% \item Show that $C_2$ is not a CFL.
% 
% 
% 
% 
% \end{enumerate}





\end{document}
