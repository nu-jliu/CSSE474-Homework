\documentclass[11pt]{article}

\usepackage{enumitem}


%%TO EDIT
\newcommand{\dueclassnumber}{17}
\newcommand{\assignmentnum}{7}

% CHANGE issolution{0} to issolution{1} for homework submission.
% WRITE solutions inside the \solution{} commands.
% or, you can use \if\issolution1 … \fi

\def\issolution{1}
\def\myname{Allen Liu} % My name goes here
\def\mysec{01} % Section number goes here
\def\myCM{374} % Campus Mailbox goes here

\input{../common/flags}
\newcommand\vv[2]{\begin{bmatrix} {#1} \cr {#2} \end{bmatrix}}

% Leave the next line alone. This is for my answer keys.
%\def\isanswerkey{1}

\usepackage{bbm,fancyhdr,ifthen,setspace,hyperref,url}
\usepackage{amssymb,amsmath,enumitem,amsthm,mathrsfs}
\usepackage{graphicx,xspace,color}
\usepackage{hhline}
\usepackage{tikz}
\usetikzlibrary{automata, positioning, arrows,chains,scopes,fit}
\tikzset{
->, % makes the edges directed
>=stealth', % makes the arrow heads bold
node distance=2.4cm, % specifies the minimum distance between two nodes. Change if necessary.
every state/.style={thick, fill=gray!10}, % sets the properties for each ’state’ node
initial text=$ $, % sets the text that appears on the start arrow
}

\topmargin=-.5in
\headsep=0.0in
\oddsidemargin=-.35in
\evensidemargin=-.65in
\textwidth=7.25in
\textheight=9.75in
\footskip=0in
\usepackage{titlesec}
\titlespacing*{\paragraph}{0pt}{2ex plus 1ex minus .2ex}{1ex}
\fancyhf{} % clear all header and footers
\renewcommand{\headrulewidth}{0pt} % remove the header rule
%\rfoot{\thepage}
\pagestyle{fancy}

\ifx\myname\undefined
\def\myname{}
\fi

\ifx\isanswerkey\undefined
\def\isanswerkey{0}
\fi

\if\isanswerkey1
\def\issolution{1}
\fi


\newcommand{\getcourseyear}[1]{2022}
\newcommand{\getcourseterm}{Spring \getcourseyear{1}}

\newcommand{\getclassmonthnum}[1]{\ifthenelse{#1<16}{3}{\ifthenelse{#1<29}{4}{5}}}
\newcommand{\getclassmonthshort}[1]{\ifthenelse{#1<16}{Mar}{\ifthenelse{#1<29}{Apr}{May}}}
\newcommand{\getclassmonth}[1]{\ifthenelse{#1<16}{March}{\ifthenelse{#1<29}{April}{May}}}
\newcommand{\getclassdayofmonth}[1]{\ifthenelse{
#1=1}{7}{\ifthenelse{
#1=2}{8}{\ifthenelse{
#1=3}{10}{\ifthenelse{
#1=4}{11}{\ifthenelse{
#1=5}{14}{\ifthenelse{
#1=6}{15}{\ifthenelse{
#1=7}{17}{\ifthenelse{
#1=8}{18}{\ifthenelse{
#1=9}{21}{\ifthenelse{
#1=10}{22}{\ifthenelse{
#1=11}{24}{\ifthenelse{
#1=12}{25}{\ifthenelse{
#1=13}{28}{\ifthenelse{
#1=14}{29}{\ifthenelse{
#1=15}{31}{\ifthenelse{
#1=16}{1}{\ifthenelse{
#1=17}{4}{\ifthenelse{
#1=18}{5}{\ifthenelse{
#1=19}{7}{\ifthenelse{
#1=20}{8}{\ifthenelse{
#1=21}{18}{\ifthenelse{
#1=22}{19}{\ifthenelse{
#1=23}{21}{\ifthenelse{
#1=24}{22}{\ifthenelse{
#1=25}{25}{\ifthenelse{
#1=26}{26}{\ifthenelse{
#1=27}{28}{\ifthenelse{
#1=28}{29}{\ifthenelse{
#1=29}{2}{\ifthenelse{
#1=30}{3}{\ifthenelse{
#1=31}{5}{\ifthenelse{
#1=32}{6}{\ifthenelse{
#1=33}{9}{\ifthenelse{
#1=34}{10}{\ifthenelse{
#1=35}{12}{\ifthenelse{
#1=36}{13}{\ifthenelse{
#1=37}{16}{\ifthenelse{
#1=38}{17}{\ifthenelse{
#1=39}{19}{\ifthenelse{
#1=40}{20}{}}}}}}}}}}}}}}}}}}}}}}}}}}}}}}}}}}}}}}}}}

\newcommand{\getclassdate}[1]{\getclassmonth{#1}\xspace\getclassdayofmonth{#1}}
\newcommand{\getclassdateshort}[1]{\getclassmonthshort{#1}\xspace\getclassdayofmonth{#1}}
\newcommand{\getclassdatenum}[1]{\getclassmonthnum{#1}/\getclassdayofmonth{#1}}
\newcommand{\getMday}{Mon}
\newcommand{\getTday}{Tue}
\newcommand{\getWday}{Wed}
\newcommand{\getRday}{Thu}
\newcommand{\getFday}{Fri}
\newcommand{\getclassdayofweek}[1]{\ifthenelse{
#1=1}{\getMday}{\ifthenelse{#1=2}{\getTday}{\ifthenelse{#1=3}{\getRday}{\ifthenelse{#1=4}{\getFday}{\ifthenelse{
#1=5}{\getMday}{\ifthenelse{#1=6}{\getTday}{\ifthenelse{#1=7}{\getRday}{\ifthenelse{#1=8}{\getFday}{\ifthenelse{
#1=9}{\getMday}{\ifthenelse{#1=10}{\getTday}{\ifthenelse{#1=11}{\getRday}{\ifthenelse{#1=12}{\getFday}{\ifthenelse{
#1=13}{\getMday}{\ifthenelse{#1=14}{\getTday}{\ifthenelse{#1=15}{\getRday}{\ifthenelse{#1=16}{\getFday}{\ifthenelse{
#1=17}{\getMday}{\ifthenelse{#1=18}{\getTday}{\ifthenelse{#1=19}{\getRday}{\ifthenelse{#1=20}{\getFday}{\ifthenelse{
#1=21}{\getMday}{\ifthenelse{#1=22}{\getTday}{\ifthenelse{#1=23}{\getRday}{\ifthenelse{#1=24}{\getFday}{\ifthenelse{
#1=25}{\getMday}{\ifthenelse{#1=26}{\getTday}{\ifthenelse{#1=27}{\getRday}{\ifthenelse{#1=28}{\getFday}{\ifthenelse{
#1=29}{\getMday}{\ifthenelse{#1=30}{\getTday}{\ifthenelse{#1=31}{\getRday}{\ifthenelse{#1=32}{\getFday}{\ifthenelse{
#1=33}{\getMday}{\ifthenelse{#1=34}{\getTday}{\ifthenelse{#1=35}{\getRday}{\ifthenelse{#1=36}{\getFday}{\ifthenelse{
#1=37}{\getMday}{\ifthenelse{#1=38}{\getTday}{\ifthenelse{#1=39}{\getRday}{\ifthenelse{#1=40}{\getFday}\xspace
}}}}}}}}}}}}}}}}}}}}}}}}}}}}}}}}}}}}}}}}
\newcommand{\examdate}[1]{\ifthenelse{#1=1}{March 23, \getcourseyear{11}}{\ifthenelse{#1=2}{April 7, \getcourseyear{11}}{\ifthenelse{#1=3}{May 4, \getcourseyear{11}}{\ifthenelse{#1=4}{ } {}}}}}




\newcommand{\classdate}[1]{\getclassdate{#1}, \getcourseyear{#1}}
\newcommand{\dueclassdate}{\getclassdayofweek{\dueclassnumber} \getclassdateshort{\dueclassnumber}}


\newcommand\largeemptyspace{\vphantom{\textnormal{$\ds\int$}}}
\newcommand\nameblank{\if\issolution0\underline{\hskip11.25cm {\largeemptyspace}}
  \else\underline{\hskip.2cm{\LARGE\myname}\hskip6cm}\fi}
\newcommand\nameblankshort{\if\issolution0\underline{\hskip9.5cm {\largeemptyspace}}
  \else\underline{\hskip.2cm{\LARGE\myname {\largeemptyspace}}\hskip.2cm}\fi}
\newcommand\secblank{\if\issolution0\underline{\hskip1.5cm{\largeemptyspace}}
  \else\underline{\hskip.2cm{\LARGE\mysec {\largeemptyspace}}\hskip.2cm} \hskip.8cm \fi}
\newcommand\CMblank{\if\issolution0\underline{\hskip2.25cm{\largeemptyspace}}
  \else\underline{\hskip.2cm{\LARGE\myCM {\largeemptyspace}}\hskip.2cm}\fi}

\newcommand{\namegroupline}{Name: \nameblank Group \#: \underline{\hskip1.5cm{\largeemptyspace}}}
\newcommand{\nameline}{\begin{minipage}{0.6\linewidth} Name: \nameblank \end{minipage}}
\newcommand{\namelineshort}{Name: \nameblankshort}
\newcommand{\namesecline}{\begin{minipage}{0.7\linewidth} Name: \nameblank \end{minipage} \hfill \begin{minipage}{0.29\linewidth}Section \#: \secblank\end{minipage}}
\newcommand{\namesecCMline}{\begin{minipage}{0.6\linewidth}Name: \nameblankshort \end{minipage} \hfill \begin{minipage}{0.4\linewidth} Section \# \secblank CM\# \CMblank \end{minipage}}
\newcommand{\keyline}{{\color{red} SOLUTION KEY}}
%\newcommand{\nameline}{Name: \rule{11.5cm}{0.01cm} \hfill Section: \rule{1.5cm}{0.01cm}}
%\newcommand{\keyline}{Name: \rule{4cm}{0.01cm} SOLUTION KEY \rule{4cm}{0.01cm} \hfill Section: \rule{1.5cm}{0.01cm}}
%\newcommand{\groupline}{Group members present: \rule{8.5cm}{0.01cm} \hfill Group \#: \rule{1.5cm}{0.01cm}}
\newcommand{\course}{CSSE/MA 474\xspace}
\newcommand{\coursewithname}{CSSE/MA 474. Theory of Computation\xspace}


\newcommand{\wtitlestuff}{
\if\isanswerkey0
  \nameline
  \else
  \keyline
\fi
\begin{center}
\large \course Worksheet for Class \#\classnumber\\
\small \classdate{\classnumber}
\normalsize
\end{center}}

\newcommand{\lectitlestuff}{
\begin{center}
\Large \course Lecture \#\classnumber\\
\vskip 3pt \small Nate Chenette \\ \classdate{\classnumber}
\normalsize
\end{center}}

\newcommand{\othertitlestuff}{
\begin{center}
\Large \othertitle\\
\small \coursewithname\\
Class \#\classnumber, \classdate{\classnumber}\\
\normalsize
\end{center}}

\newcommand{\othertitlestuffnodate}{
\begin{center}
\Large \othertitle\\
\small \coursewithname\\
\normalsize
\end{center}}

\newcommand{\othernametitlestuff}{
\nameline\\
\othertitlestuff
}

\newcommand{\assignmenttitlestuff}{
\begin{center}
\Large \course Assignment \assignmentnum\\
\small Due date: \dueclassdate
\normalsize
\end{center}}

\newcommand{\assignmentnametitlestuff}{
\if\isanswerkey0
  \namesecCMline
  \else
  \keyline
\fi
\assignmenttitlestuff
}

\newcommand{\quiznametitlestuff}{
\if\isanswerkey0
  \namesecCMline
  \else
  \keyline
\fi
\begin{center}
\Large \course Quiz \quiznum\\
\small \classdate{\classnumber}
\normalsize
\end{center}}


\setlength{\parindent}{0in}
\setlength{\fboxsep}{.1in}

\renewcommand{\emptyset}{\varnothing}
\newcommand{\tvs}{\textvisiblespace}
\newcommand{\brk}{\vskip.2cm \hrule \vskip.2cm}
\newcommand{\ds}{\displaystyle}
\newcommand{\abs}[1]{\left\lvert {#1}\right\rvert}
\newcommand{\Lsym}{\text{L}}
\newcommand{\Rsym}{\text{R}}
\newcommand{\qacc}{q_{\textnormal{accept}}}
\newcommand{\qrej}{q_{\textnormal{reject}}}
\newcommand{\tmRej}{$\to$ \textbf{\textit{reject}}}
\newcommand{\tmAcc}{$\to$ \textbf{\textit{accept}}}
\def\lep{\le_\textnormal{P}}
\def\lem{\le_\textnormal{m}}
\def\ATM{A_\textnormal{TM}}
\newcommand{\vv}[2]{\begin{bmatrix} {#1} \cr {#2} \end{bmatrix}}
\newcommand{\vvt}[2]{\begin{bmatrix} {\tt #1} \cr {\tt #2} \end{bmatrix}}
\def\hs{\quad \texttt\#\quad }
\def\multiset#1#2{\ensuremath{\left(\kern-.3em\left(\genfrac{}{}{0pt}{}{#1}{#2}\right)\kern-.3em\right)}}
\def\time{\textsf{TIME}}
\def\ntime{\textsf{NTIME}}
\def\P{\textsf{P}}
\def\NP{\textsf{NP}}
   
%logic
\newcommand{\se}{\big|}
\newcommand{\lra}{\leftrightarrow}
\newcommand{\Lra}{\Leftrightarrow}
\newcommand{\we}{\wedge}
\def\thf{%
   \leavevmode
   \lower0.2ex\hbox{$\cdot$}%
   \kern-0.0em\raise0.7ex\hbox{$\cdot$}%
   \kern-0.0em\lower0.2ex\hbox{$\cdot$}%
   \thinspace}

%Number Systems
\newcommand{\bbZ}{\mathbb{Z}}
\newcommand{\bfZ}{\mathbf{Z}}
\newcommand{\bfZp}{\mathbf{Z}^+}
\newcommand{\bbN}{\mathbb{N}}
\newcommand{\bfN}{\mathbf{N}}
\newcommand{\bbQ}{\mathbb{Q}}
\newcommand{\bfQ}{\mathbf{Q}}
\newcommand{\bbR}{\mathbb{R}}
\newcommand{\bfR}{\mathbf{R}}
\newcommand{\bbC}{\mathbb{C}}
\newcommand{\bfC}{\mathbf{C}}

%sets
\newcommand{\U}{\mathscr{U}}
\newcommand{\ol}[1]{\overline{#1}}
\newcommand{\ssq}{\subseteq}
\newcommand{\sst}{\subset}
\def\ps{\mathcal{P}}
\def\sd{\,\triangle\,}
\def\sdonly{\triangle}
\def\es{\emptyset}

%cards
\def\hst{\heartsuit}
\def\cst{\clubsuit}
\def\sst{\spadesuit}
\def\dst{\diamondsuit}


\newcommand{\lcm}{{\rm lcm}}


\newcommand{\imgdir}{../images/}


\newcommand{\makeexamcover}{
\ifdefined\finalexam
\ \vskip2cm
\begin{center}
\huge \course Final Exam \\
\Large \finalexamdate \vskip1cm
\end{center}
\normalsize \instructions \vskip1cm
\begin{spacing}{1.5}
\begin{center}
\scorechart
\end{center}
\end{spacing}
\else \ifdefined\examnum
\ \vskip2cm
\begin{center}
\huge \course Exam \examnum \\
\Large \examdate{\examnum} \vskip1cm
\end{center}
\normalsize \instructions \vskip1cm
\begin{spacing}{1.5}
\begin{center}
\scorechart
\end{center}
\end{spacing}
\fi
\fi
}




\fancypagestyle{examcover}{% 
\fancyhf{}
\renewcommand{\footrulewidth}{0pt}
\lhead{\if\isanswerkey1{\keyline}\else{\nameline}\fi}
%\lhead{\if\isanswerkey1{\keyline}\else{\namesecline}\fi}
}



\fancypagestyle{exameverypage}{% 
\fancyhf{}
\renewcommand{\footrulewidth}{0pt}
\rhead{\if\isanswerkey1{\keyline}\else{}\fi}
\fancyfoot[R]{\thepage}
%\lhead{\if\isanswerkey1{\keyline}\else{\namesecline}\fi}
}

\newcommand{\definition}[1]{{\sc Definition}.~~{#1}\vskip.2cm}

\usepackage[framemethod=default]{mdframed}
\global\mdfdefinestyle{red1}{linecolor=red, linewidth=1pt, leftmargin=1cm, rightmargin=1cm}
\global\mdfdefinestyle{black1}{linecolor=black, linewidth=1pt,} %leftmargin=.1cm, rightmargin=.1cm}

\newcommand{\solution}[2][]{\if\issolution0 #1 \else \begin{mdframed}[style=black1] #2 \end{mdframed} \fi}

\newcommand{\cmblanka}[1]{\if\issolution0 	\underline{\hskip1cm{\largeemptyspace}}
\else 						  		\underline{\hskip.35cm {#1}\hskip.35cm{\largeemptyspace}}\fi}
\newcommand{\sblanka}[1]{\if\issolution0 		\underline{\hskip1.5cm{\largeemptyspace}}
\else 						  		\underline{\hskip.25cm {#1}\hskip.25cm{\largeemptyspace}}\fi}
\newcommand{\mblanka}[1]{\if\issolution0 	\underline{\hskip3cm{\largeemptyspace}}
\else 						  		\underline{\hskip.5cm {#1}\hskip.5cm{\largeemptyspace}}\fi}
\newcommand{\lblanka}[1]{\if\issolution0 		\underline{\hskip4.5cm{\largeemptyspace}}
\else 						  		\underline{\hskip.75cm {#1}\hskip.75cm{\largeemptyspace}}\fi}
\newcommand{\Lblanka}[1]{\if\issolution0		\underline{\hskip6cm{\largeemptyspace}}
\else 								\underline{\hskip1cm {#1}\hskip1cm{\largeemptyspace}}\fi}
\newcommand{\LLblanka}[1]{\if\issolution0	\underline{\hskip7.5cm{\largeemptyspace}}
\else 								\underline{\hskip1.25cm {#1}\hskip1.25cm{\largeemptyspace}}\fi}
\newcommand{\LLLblanka}[1]{\if\issolution0 	\underline{\hskip9cm{\largeemptyspace}}
\else 								\underline{\hskip1.5cm {#1}\hskip1.5cm{\largeemptyspace}}\fi}
\newcommand{\tinyspacea}[1]{\if\issolution0 	\hskip.2cm{\largeemptyspace}
\else 						  		{#1}{\largeemptyspace}\fi}
\newcommand{\cmspacea}[1]{\if\issolution0 	\hskip1cm{\largeemptyspace}
\else 						  		\hskip.15cm {#1}\hskip.15cm{\largeemptyspace}\fi}
\newcommand{\sspacea}[1]{\if\issolution0 		\hskip1.5cm{\largeemptyspace}
\else 						  		\hskip.25cm {#1}\hskip.25cm{\largeemptyspace}\fi}
\newcommand{\mspacea}[1]{\if\issolution0 	\hskip3cm{\largeemptyspace}
\else 						  		\hskip.25cm {#1}\hskip.25cm{\largeemptyspace}\fi}
\newcommand{\lspacea}[1]{\if\issolution0 		\hskip4.5cm{\largeemptyspace}
\else 						  		\hskip.25cm {#1}\hskip.25cm{\largeemptyspace}\fi}
\newcommand{\Lspacea}[1]{\if\issolution0 	\hskip6cm{\largeemptyspace}
\else 						  		\hskip.25cm {#1}\hskip.25cm{\largeemptyspace}\fi}


\newcommand{\sparagraph}[1]{\vskip-1cm\paragraph{#1}}

\if\isanswerkey1\input{macsse474-key}\fi


\usetikzlibrary{positioning}

\begin{document}

\assignmentnametitlestuff


\if\isanswerkey0
{\bf Please follow the homework guidelines from A01.}

\fi


\begin{enumerate}

\item (2.48) Let $\Sigma = \{0,1\}$. Let $C_1$ be the language of all strings that contain a {\tt 1} in their middle third. Let $C_2$ be the language of all strings that contain two {\tt 1}s in their middle third. In particular,
\begin{align*}
C_1 &= \{xyz \mid x,z \in \Sigma^* \textnormal{ and $y \in \Sigma^*{\tt 1}\Sigma^*$, where $|x| = |z| \ge |y|$}\} \cr
C_2 &= \{xyz \mid x,z \in \Sigma^* \textnormal{ and $y \in \Sigma^*{\tt 1}\Sigma^*{\tt 1}\Sigma^*$, where $|x| = |z| \ge |y|$}\}.
\end{align*}
\begin{enumerate}
\item Show that $C_1$ is a CFL.
\solution{
\if\isanswerkey1\solMiddleThirdHasOneCFL\fi
The language $C_1$ can be recognized by following CFG:
\begin{align*}
    S &\to S_1 \mid S_2 \\
    S_1 &\to XTX \mid T \\
    T &\to XXTX \mid {\tt 1}\\
    S_2 &\to XFX \mid F \\
    F &\to XFXX \mid {\tt 1} \\
    X &\to {\tt a} \mid  {\tt b} 
\end{align*}
}
\item Show that $C_2$ is not a CFL.
\solution{
\if\isanswerkey1\solMiddleThirdHasTwoOnesNotCFL\fi
Assume language $C_2$ is a context-free language, then pumping lemma applied. Let $p$ be the pumping length as in the pumping lemma. According to the lemma, string $s=uvxyz \in C_2$. Consider a string $s={\tt 0}^{p+1}{\tt 1}{\tt 0}^p{\tt 1}{\tt 0}^{p+1} \in C_2$, $|s|=3p+3 \ge p$ as in pumping lemma. Since $|vxy| \le p$, $v \in {\tt 0}^+$ and $y \in {\tt 0}^+$. So for a pumped string, let's say $s'=uv^2xy^2z$, at most 2 adjacent groups of {\tt 0}s would have been stretched. So this will result in the string has $|x|\not=|y|$ or $|y| > |z|$, which makes the $s' \not\in C_2$. Hence language $C_2$ is not regular.
}

\end{enumerate}

\item (3.1) Consider the Turing machine $M_2$, whose state diagram is below. (The notation $a \to b, \Rsym$ denotes reading $a$ and overwriting with $b$ at the current head location, then moving the head one step right.) In each case, give the sequence of configurations that $M_2$ enters when started on the indicated input string.

\begin{minipage}{.15\linewidth}
\begin{enumerate}
\item {\tt 0}
\item {\tt 00}
\item {\tt 000}
\item {\tt 000000}
\end{enumerate}

\end{minipage}\hfill
\begin{minipage}{.8\linewidth}
\begin{tikzpicture}[node distance=3cm]
\node [state, initial] (1) {$q_1$};
\node [state, right=3cm of 1] (2) {$q_2$};
\node [state, right=4cm of 2] (3) {$q_3$};
\node [state, below=2cm of 3] (4) {$q_4$};
\node [state, above right= 1.5cm and 2cm of 2] (5) {$q_5$};
\node [state, below=1cm of 2] (acc) {$\qacc$};
\node [state, below=2cm of 1] (rej) {$\qrej$};
\draw 
(1) edge[below] node{${\tt 0} \to \tvs,\Rsym$} (2)
(1) edge[left] node[align=center]{$\tvs \to \Rsym$\\${\tt x} \to \Rsym$} (rej)
(2) edge[below] node{${\tt 0} \to {\tt x},\Rsym$} (3)
(2) edge[right] node[align=center]{$\tvs \to \Rsym$} (acc)
(2) edge[loop above] node{${\tt x} \to \Rsym$} (2)
(3) edge[loop above] node{${\tt x} \to \Rsym$} (3)
(3) edge[bend right=10, left] node{${\tt 0} \to \Rsym$} (4)
(3) edge[above,sloped] node{$\tvs \to \Lsym$} (5)
(4) edge[bend right=10, right] node{${\tt 0} \to {\tt x}, \Rsym$} (3)
(4) edge[loop right] node{${\tt x} \to \Rsym$} (4)
(5) edge[above,sloped] node{$\tvs \to \Rsym$} (2)
(5) edge[loop above] node[align=center]{${\tt 0} \to \Lsym$ \\${\tt x} \to \Lsym$} (5)
(4) edge[bend left=7, above] node{$\tvs \to \Rsym$} (rej)
;
\end{tikzpicture}
\end{minipage}
\solution{
\if\isanswerkey1\solTMConfigurationSequence\fi
\begin{enumerate}
    \item \begin{align*}
        &q_1{\tt 0}\\
        &\tvs q_2\tvs\\
        &\qacc
    \end{align*}
    \item \begin{align*}
        &q_1{\tt 00}\\
        &\tvs q_2{\tt 0}\\
        &\tvs{\tt x}q_3\tvs\\
        &\qrej
    \end{align*}
    \item \begin{align*}
        &q_1{\tt 000}\\
        &\tvs q_2{\tt 00}\\
        &\tvs{\tt x}q_3{\tt 0}\\
        &\tvs{\tt x0}q_4\tvs\\
        &\qrej
    \end{align*}
    \item \begin{align*}
        &q_1{\tt 000000}\\
        &\tvs q_2{\tt 00000}\\
        &\tvs{\tt x}q_3{\tt 0000}\\
        &\tvs{\tt x0}q_4{\tt 000}\\
        &\tvs{\tt x0x}q_3{\tt 00}\\
        &\tvs{\tt x0x0}q_4{\tt 0}\\
        &\tvs{\tt x0x0x}q_3\tvs\\
        &\tvs{\tt x0x0}q_5{\tt x}\\
        &\tvs{\tt x0x}q_5{\tt 0x}\\
        &\tvs{\tt x0}q_5{\tt x0x}\\
        &\tvs{\tt x}q_5{\tt 0x0x}\\
        &\tvs q_5{\tt x0x0x}\\
        &q_5\tvs{\tt x0x0x}\\
        &\tvs q_2{\tt x0x0x}\\
        &\tvs{\tt x}q_2{\tt 0x0x}\\
        &\tvs{\tt xx}q_3{\tt x0x}\\
        &\tvs{\tt xxx}q_3{\tt 0x}\\
        &\tvs{\tt xxx0}q_4{\tt x}\\
        &\tvs{\tt xxx0x}q_4\tvs\\
        &\qrej
    \end{align*}
\end{enumerate}
}


\item (3.8) Give implementation-level descriptions (i.e., clear, complete English descriptions of the steps of the machine) of Turing machines that decide the following languages over the alphabet $\{{\tt 0},{\tt 1}\}$.
\begin{enumerate}
\item $\{w \mid w \textnormal{ contains an equal number of {\tt 0}s and {\tt 1}s}\}$
\item $\{w \mid w \textnormal{ contains twice as many {\tt 0}s as {\tt 1}s}\}$
\item $\{w \mid w \textnormal{ does not contain twice as many {\tt 0}s as {\tt 1}s}\}$
\end{enumerate}
\solution{
\if\isanswerkey1\solTMDescriptionsDecideLanguages\fi
\begin{enumerate}
    \item 
    \begin{enumerate}
        \item Scan to the right to make sure that the input has ${\tt 0} \ge 1$ and ${\tt 1} \ge 1$ or the input is empty. 
        \begin{itemize}
            \item If the input is empty $\to \qacc$
        \end{itemize}
        \item Scan to the right and cross of the first {\tt 0}
        \begin{itemize}
            \item If no {\tt 0} is found, scan to the right to find for {\tt 1}s
            \begin{itemize}
                \item If no {\tt 1} is found $\to \qacc$
                \item If a {\tt 1} has been found $\to \qrej$
            \end{itemize}
        \end{itemize}
        \item Rewind to the left and scan to the right to cross off {\tt 1}
        \begin{itemize}
            \item If no {\tt 1} has been found $\to \qrej$
        \end{itemize}
        \item Rewind to the left and repeat step ii. 
    \end{enumerate}
    \item
    \begin{enumerate}
        \item Scan to the right to make sure that the input has ${\tt 0} \ge 1$ and ${\tt 1} \ge 1$ or the input is empty. 
        \begin{itemize}
            \item If the input is empty $\to \qacc$
        \end{itemize}
        \item Scan to the right and cross of the first {\tt 1}
        \begin{itemize}
            \item If no {\tt 1} is found, scan to the right to find for {\tt 0}s
            \begin{itemize}
                \item If no {\tt 0} is found $\to \qacc$
                \item If a {\tt 0} has been found $\to \qrej$
            \end{itemize}
        \end{itemize}
        \item Rewind to the left and scan to the right to cross off first 2 {\tt 0}s
        \begin{itemize}
            \item If no {\tt 0} has been found $\to \qrej$
            \item If only one {\tt 0} has been found $\to \qrej$
        \end{itemize}
        \item Rewind to the left and repeat step ii. 
    \end{enumerate}
    \item
    \begin{enumerate}
        \item Scan to the right to make sure that the input has ${\tt 0} \ge 1$ and ${\tt 1} \ge 1$ or the input is empty. 
        \begin{itemize}
            \item If the input is empty $\to \qrej$
        \end{itemize}
        \item Scan to the right and cross of the first {\tt 1}
        \begin{itemize}
            \item If no {\tt 1} is found, scan to the right to find for {\tt 0}s
            \begin{itemize}
                \item If no {\tt 0} is found $\to \qrej$
                \item If a {\tt 0} has been found $\to \qacc$
            \end{itemize}
        \end{itemize}
        \item Rewind to the left and scan to the right to cross off first 2 {\tt 0}s
        \begin{itemize}
            \item If no {\tt 0} has been found $\to \qacc$
            \item If only one {\tt 0} has been found $\to \qacc$
        \end{itemize}
        \item Rewind to the left and repeat step ii. 
    \end{enumerate}
\end{enumerate}
}


\item (3.9) Let a $k$-PDA be a pushdown automata that has $k$ stacks. Thus a 0-PDA is an NFA and a 1-PDA is a conventional PDA. You already know that 1-PDAs are more powerful (recognize a larger class of languages) than 0-PDAs.
\begin{enumerate}
\item Show that 2-PDAs are more powerful than 1-PDAs.
\solution{
\if\isanswerkey1\solTwoPDAMorePowerfulOnePDA\fi
For a 2-PDAs, they can recognize some languages that is not recognizable on 1-PDA. For example, language $A=\left\{{\tt a}^n{\tt b}^n{\tt c}^n \mid n > 0\right\}$, it is not recognizable by 1-PDA since it is not CFL. It can be proved by the following procedure: \newline Assume language $A$ is a CFL, pumping lemma applied. Let $p$ be the pumping length as in pumping lemma. Then, $s=uvxyz \in A$ as in pumping lemma. Consider a string $s={\tt a}^p{\tt b}^p{\tt c}^p \in A$, $|s|=3p \ge p$. Since in the pumping lemma, $|vxy| \le p$, $v$ and $y$ at most contains 2 of the 3 kinds in $\left\{{\tt a}, {\tt b}, {\tt c}\right\}$. For a pumped string, say $s'=uv^2xy^2z$, since $|v| \ge 0$ and $|y| \ge 0$, it must have two kinds that have more symbols than the $3^{\textnormal{rd}}$ one. So $s' \not\in A$. Hence language $A$ is not a CFL. 
\newline For a 2-PDA, the language $A$ can be recognized by the following 2-PDA: 
\begin{itemize}
    \item Begin with pushing the start character \$ to both stacks
    \item When the input reads symbol {\tt a}, it pushed it into the both stacks
    \item When the input reads symbol {\tt b}, it popped a symbol from the first stack
    \item When the input reads symbol {\tt c}, it popped a symbol from the second stack
    \item When the input is empty and there is \$ in both stack $\to \qacc$
    \item Otherwise $\to \qrej$
\end{itemize}
Hence the 2-PDAs can recognize some languages that are not recognizable by 1-PDA, so it is more powerful than 1-PDA.
}
\item Show that 3-PDAs are not more powerful than 2-PDAs. You may use the result of Theorem 3.13 that any multi-tape Turing machine can be simulated by a single-tape Turing machine.\\
(Hint: Simulate a Turing machine tape with two stacks.) 
\solution{
\if\isanswerkey1\solThreePDANotMorePowerfulTwoPDA\fi
For a 2-PDA, at a given state $q_x$, where the first stack contains the characters $s_1$ and second stack contains the characters $s_2$. This is equivalent to a 2-tape Turing machine with first tape at $s_1q_x\tvs$ and second tape at $s_2q_x\tvs$. popping from a stack is equivalent to moving L and overwriting with character $\tvs$ and pushing ${\tt a}$ is equivalent with moving R and overwriting with character ${\tt a}$. So for a 3-PDA, it can be also converted into a 3-tape Turing machine. Since by Theorem 3.13, $k$-tape Turing machine can all be converted into a single-tape Turing machine. So both 2-PDAs and 3-PDAs can be converted into a single-tape Turing machine. Hence 3-PDAs is same powerful as 2-PDAs, so it is not more powerful than 2-PDAs.  
}
\end{enumerate}



\end{enumerate}





\end{document}
