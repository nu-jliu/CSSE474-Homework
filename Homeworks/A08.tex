\documentclass[11pt]{article}

\usepackage{enumitem}


%%TO EDIT
\newcommand{\dueclassnumber}{21}
\newcommand{\assignmentnum}{8}

% CHANGE issolution{0} to issolution{1} for homework submission.
% WRITE solutions inside the \solution{} commands.
% or, you can use \if\issolution1 … \fi

\def\issolution{1}
\def\myname{Allen Liu} % My name goes here
\def\mysec{01} % Section number goes here
\def\myCM{374} % Campus Mailbox goes here

\input{../common/flags}
\newcommand\vv[2]{\begin{bmatrix} {#1} \cr {#2} \end{bmatrix}}

% Leave the next line alone. This is for my answer keys.
%\def\isanswerkey{1}

\usepackage{bbm,fancyhdr,ifthen,setspace,hyperref,url}
\usepackage{amssymb,amsmath,enumitem,amsthm,mathrsfs}
\usepackage{graphicx,xspace,color}
\usepackage{hhline}
\usepackage{tikz}
\usetikzlibrary{automata, positioning, arrows,chains,scopes,fit}
\tikzset{
->, % makes the edges directed
>=stealth', % makes the arrow heads bold
node distance=2.4cm, % specifies the minimum distance between two nodes. Change if necessary.
every state/.style={thick, fill=gray!10}, % sets the properties for each ’state’ node
initial text=$ $, % sets the text that appears on the start arrow
}

\topmargin=-.5in
\headsep=0.0in
\oddsidemargin=-.35in
\evensidemargin=-.65in
\textwidth=7.25in
\textheight=9.75in
\footskip=0in
\usepackage{titlesec}
\titlespacing*{\paragraph}{0pt}{2ex plus 1ex minus .2ex}{1ex}
\fancyhf{} % clear all header and footers
\renewcommand{\headrulewidth}{0pt} % remove the header rule
%\rfoot{\thepage}
\pagestyle{fancy}

\ifx\myname\undefined
\def\myname{}
\fi

\ifx\isanswerkey\undefined
\def\isanswerkey{0}
\fi

\if\isanswerkey1
\def\issolution{1}
\fi


\newcommand{\getcourseyear}[1]{2022}
\newcommand{\getcourseterm}{Spring \getcourseyear{1}}

\newcommand{\getclassmonthnum}[1]{\ifthenelse{#1<16}{3}{\ifthenelse{#1<29}{4}{5}}}
\newcommand{\getclassmonthshort}[1]{\ifthenelse{#1<16}{Mar}{\ifthenelse{#1<29}{Apr}{May}}}
\newcommand{\getclassmonth}[1]{\ifthenelse{#1<16}{March}{\ifthenelse{#1<29}{April}{May}}}
\newcommand{\getclassdayofmonth}[1]{\ifthenelse{
#1=1}{7}{\ifthenelse{
#1=2}{8}{\ifthenelse{
#1=3}{10}{\ifthenelse{
#1=4}{11}{\ifthenelse{
#1=5}{14}{\ifthenelse{
#1=6}{15}{\ifthenelse{
#1=7}{17}{\ifthenelse{
#1=8}{18}{\ifthenelse{
#1=9}{21}{\ifthenelse{
#1=10}{22}{\ifthenelse{
#1=11}{24}{\ifthenelse{
#1=12}{25}{\ifthenelse{
#1=13}{28}{\ifthenelse{
#1=14}{29}{\ifthenelse{
#1=15}{31}{\ifthenelse{
#1=16}{1}{\ifthenelse{
#1=17}{4}{\ifthenelse{
#1=18}{5}{\ifthenelse{
#1=19}{7}{\ifthenelse{
#1=20}{8}{\ifthenelse{
#1=21}{18}{\ifthenelse{
#1=22}{19}{\ifthenelse{
#1=23}{21}{\ifthenelse{
#1=24}{22}{\ifthenelse{
#1=25}{25}{\ifthenelse{
#1=26}{26}{\ifthenelse{
#1=27}{28}{\ifthenelse{
#1=28}{29}{\ifthenelse{
#1=29}{2}{\ifthenelse{
#1=30}{3}{\ifthenelse{
#1=31}{5}{\ifthenelse{
#1=32}{6}{\ifthenelse{
#1=33}{9}{\ifthenelse{
#1=34}{10}{\ifthenelse{
#1=35}{12}{\ifthenelse{
#1=36}{13}{\ifthenelse{
#1=37}{16}{\ifthenelse{
#1=38}{17}{\ifthenelse{
#1=39}{19}{\ifthenelse{
#1=40}{20}{}}}}}}}}}}}}}}}}}}}}}}}}}}}}}}}}}}}}}}}}}

\newcommand{\getclassdate}[1]{\getclassmonth{#1}\xspace\getclassdayofmonth{#1}}
\newcommand{\getclassdateshort}[1]{\getclassmonthshort{#1}\xspace\getclassdayofmonth{#1}}
\newcommand{\getclassdatenum}[1]{\getclassmonthnum{#1}/\getclassdayofmonth{#1}}
\newcommand{\getMday}{Mon}
\newcommand{\getTday}{Tue}
\newcommand{\getWday}{Wed}
\newcommand{\getRday}{Thu}
\newcommand{\getFday}{Fri}
\newcommand{\getclassdayofweek}[1]{\ifthenelse{
#1=1}{\getMday}{\ifthenelse{#1=2}{\getTday}{\ifthenelse{#1=3}{\getRday}{\ifthenelse{#1=4}{\getFday}{\ifthenelse{
#1=5}{\getMday}{\ifthenelse{#1=6}{\getTday}{\ifthenelse{#1=7}{\getRday}{\ifthenelse{#1=8}{\getFday}{\ifthenelse{
#1=9}{\getMday}{\ifthenelse{#1=10}{\getTday}{\ifthenelse{#1=11}{\getRday}{\ifthenelse{#1=12}{\getFday}{\ifthenelse{
#1=13}{\getMday}{\ifthenelse{#1=14}{\getTday}{\ifthenelse{#1=15}{\getRday}{\ifthenelse{#1=16}{\getFday}{\ifthenelse{
#1=17}{\getMday}{\ifthenelse{#1=18}{\getTday}{\ifthenelse{#1=19}{\getRday}{\ifthenelse{#1=20}{\getFday}{\ifthenelse{
#1=21}{\getMday}{\ifthenelse{#1=22}{\getTday}{\ifthenelse{#1=23}{\getRday}{\ifthenelse{#1=24}{\getFday}{\ifthenelse{
#1=25}{\getMday}{\ifthenelse{#1=26}{\getTday}{\ifthenelse{#1=27}{\getRday}{\ifthenelse{#1=28}{\getFday}{\ifthenelse{
#1=29}{\getMday}{\ifthenelse{#1=30}{\getTday}{\ifthenelse{#1=31}{\getRday}{\ifthenelse{#1=32}{\getFday}{\ifthenelse{
#1=33}{\getMday}{\ifthenelse{#1=34}{\getTday}{\ifthenelse{#1=35}{\getRday}{\ifthenelse{#1=36}{\getFday}{\ifthenelse{
#1=37}{\getMday}{\ifthenelse{#1=38}{\getTday}{\ifthenelse{#1=39}{\getRday}{\ifthenelse{#1=40}{\getFday}\xspace
}}}}}}}}}}}}}}}}}}}}}}}}}}}}}}}}}}}}}}}}
\newcommand{\examdate}[1]{\ifthenelse{#1=1}{March 23, \getcourseyear{11}}{\ifthenelse{#1=2}{April 7, \getcourseyear{11}}{\ifthenelse{#1=3}{May 4, \getcourseyear{11}}{\ifthenelse{#1=4}{ } {}}}}}




\newcommand{\classdate}[1]{\getclassdate{#1}, \getcourseyear{#1}}
\newcommand{\dueclassdate}{\getclassdayofweek{\dueclassnumber} \getclassdateshort{\dueclassnumber}}


\newcommand\largeemptyspace{\vphantom{\textnormal{$\ds\int$}}}
\newcommand\nameblank{\if\issolution0\underline{\hskip11.25cm {\largeemptyspace}}
  \else\underline{\hskip.2cm{\LARGE\myname}\hskip6cm}\fi}
\newcommand\nameblankshort{\if\issolution0\underline{\hskip9.5cm {\largeemptyspace}}
  \else\underline{\hskip.2cm{\LARGE\myname {\largeemptyspace}}\hskip.2cm}\fi}
\newcommand\secblank{\if\issolution0\underline{\hskip1.5cm{\largeemptyspace}}
  \else\underline{\hskip.2cm{\LARGE\mysec {\largeemptyspace}}\hskip.2cm} \hskip.8cm \fi}
\newcommand\CMblank{\if\issolution0\underline{\hskip2.25cm{\largeemptyspace}}
  \else\underline{\hskip.2cm{\LARGE\myCM {\largeemptyspace}}\hskip.2cm}\fi}

\newcommand{\namegroupline}{Name: \nameblank Group \#: \underline{\hskip1.5cm{\largeemptyspace}}}
\newcommand{\nameline}{\begin{minipage}{0.6\linewidth} Name: \nameblank \end{minipage}}
\newcommand{\namelineshort}{Name: \nameblankshort}
\newcommand{\namesecline}{\begin{minipage}{0.7\linewidth} Name: \nameblank \end{minipage} \hfill \begin{minipage}{0.29\linewidth}Section \#: \secblank\end{minipage}}
\newcommand{\namesecCMline}{\begin{minipage}{0.6\linewidth}Name: \nameblankshort \end{minipage} \hfill \begin{minipage}{0.4\linewidth} Section \# \secblank CM\# \CMblank \end{minipage}}
\newcommand{\keyline}{{\color{red} SOLUTION KEY}}
%\newcommand{\nameline}{Name: \rule{11.5cm}{0.01cm} \hfill Section: \rule{1.5cm}{0.01cm}}
%\newcommand{\keyline}{Name: \rule{4cm}{0.01cm} SOLUTION KEY \rule{4cm}{0.01cm} \hfill Section: \rule{1.5cm}{0.01cm}}
%\newcommand{\groupline}{Group members present: \rule{8.5cm}{0.01cm} \hfill Group \#: \rule{1.5cm}{0.01cm}}
\newcommand{\course}{CSSE/MA 474\xspace}
\newcommand{\coursewithname}{CSSE/MA 474. Theory of Computation\xspace}


\newcommand{\wtitlestuff}{
\if\isanswerkey0
  \nameline
  \else
  \keyline
\fi
\begin{center}
\large \course Worksheet for Class \#\classnumber\\
\small \classdate{\classnumber}
\normalsize
\end{center}}

\newcommand{\lectitlestuff}{
\begin{center}
\Large \course Lecture \#\classnumber\\
\vskip 3pt \small Nate Chenette \\ \classdate{\classnumber}
\normalsize
\end{center}}

\newcommand{\othertitlestuff}{
\begin{center}
\Large \othertitle\\
\small \coursewithname\\
Class \#\classnumber, \classdate{\classnumber}\\
\normalsize
\end{center}}

\newcommand{\othertitlestuffnodate}{
\begin{center}
\Large \othertitle\\
\small \coursewithname\\
\normalsize
\end{center}}

\newcommand{\othernametitlestuff}{
\nameline\\
\othertitlestuff
}

\newcommand{\assignmenttitlestuff}{
\begin{center}
\Large \course Assignment \assignmentnum\\
\small Due date: \dueclassdate
\normalsize
\end{center}}

\newcommand{\assignmentnametitlestuff}{
\if\isanswerkey0
  \namesecCMline
  \else
  \keyline
\fi
\assignmenttitlestuff
}

\newcommand{\quiznametitlestuff}{
\if\isanswerkey0
  \namesecCMline
  \else
  \keyline
\fi
\begin{center}
\Large \course Quiz \quiznum\\
\small \classdate{\classnumber}
\normalsize
\end{center}}


\setlength{\parindent}{0in}
\setlength{\fboxsep}{.1in}

\renewcommand{\emptyset}{\varnothing}
\newcommand{\tvs}{\textvisiblespace}
\newcommand{\brk}{\vskip.2cm \hrule \vskip.2cm}
\newcommand{\ds}{\displaystyle}
\newcommand{\abs}[1]{\left\lvert {#1}\right\rvert}
\newcommand{\Lsym}{\text{L}}
\newcommand{\Rsym}{\text{R}}
\newcommand{\qacc}{q_{\textnormal{accept}}}
\newcommand{\qrej}{q_{\textnormal{reject}}}
\newcommand{\tmRej}{$\to$ \textbf{\textit{reject}}}
\newcommand{\tmAcc}{$\to$ \textbf{\textit{accept}}}
\def\lep{\le_\textnormal{P}}
\def\lem{\le_\textnormal{m}}
\def\ATM{A_\textnormal{TM}}
\newcommand{\vv}[2]{\begin{bmatrix} {#1} \cr {#2} \end{bmatrix}}
\newcommand{\vvt}[2]{\begin{bmatrix} {\tt #1} \cr {\tt #2} \end{bmatrix}}
\def\hs{\quad \texttt\#\quad }
\def\multiset#1#2{\ensuremath{\left(\kern-.3em\left(\genfrac{}{}{0pt}{}{#1}{#2}\right)\kern-.3em\right)}}
\def\time{\textsf{TIME}}
\def\ntime{\textsf{NTIME}}
\def\P{\textsf{P}}
\def\NP{\textsf{NP}}
   
%logic
\newcommand{\se}{\big|}
\newcommand{\lra}{\leftrightarrow}
\newcommand{\Lra}{\Leftrightarrow}
\newcommand{\we}{\wedge}
\def\thf{%
   \leavevmode
   \lower0.2ex\hbox{$\cdot$}%
   \kern-0.0em\raise0.7ex\hbox{$\cdot$}%
   \kern-0.0em\lower0.2ex\hbox{$\cdot$}%
   \thinspace}

%Number Systems
\newcommand{\bbZ}{\mathbb{Z}}
\newcommand{\bfZ}{\mathbf{Z}}
\newcommand{\bfZp}{\mathbf{Z}^+}
\newcommand{\bbN}{\mathbb{N}}
\newcommand{\bfN}{\mathbf{N}}
\newcommand{\bbQ}{\mathbb{Q}}
\newcommand{\bfQ}{\mathbf{Q}}
\newcommand{\bbR}{\mathbb{R}}
\newcommand{\bfR}{\mathbf{R}}
\newcommand{\bbC}{\mathbb{C}}
\newcommand{\bfC}{\mathbf{C}}

%sets
\newcommand{\U}{\mathscr{U}}
\newcommand{\ol}[1]{\overline{#1}}
\newcommand{\ssq}{\subseteq}
\newcommand{\sst}{\subset}
\def\ps{\mathcal{P}}
\def\sd{\,\triangle\,}
\def\sdonly{\triangle}
\def\es{\emptyset}

%cards
\def\hst{\heartsuit}
\def\cst{\clubsuit}
\def\sst{\spadesuit}
\def\dst{\diamondsuit}


\newcommand{\lcm}{{\rm lcm}}


\newcommand{\imgdir}{../images/}


\newcommand{\makeexamcover}{
\ifdefined\finalexam
\ \vskip2cm
\begin{center}
\huge \course Final Exam \\
\Large \finalexamdate \vskip1cm
\end{center}
\normalsize \instructions \vskip1cm
\begin{spacing}{1.5}
\begin{center}
\scorechart
\end{center}
\end{spacing}
\else \ifdefined\examnum
\ \vskip2cm
\begin{center}
\huge \course Exam \examnum \\
\Large \examdate{\examnum} \vskip1cm
\end{center}
\normalsize \instructions \vskip1cm
\begin{spacing}{1.5}
\begin{center}
\scorechart
\end{center}
\end{spacing}
\fi
\fi
}




\fancypagestyle{examcover}{% 
\fancyhf{}
\renewcommand{\footrulewidth}{0pt}
\lhead{\if\isanswerkey1{\keyline}\else{\nameline}\fi}
%\lhead{\if\isanswerkey1{\keyline}\else{\namesecline}\fi}
}



\fancypagestyle{exameverypage}{% 
\fancyhf{}
\renewcommand{\footrulewidth}{0pt}
\rhead{\if\isanswerkey1{\keyline}\else{}\fi}
\fancyfoot[R]{\thepage}
%\lhead{\if\isanswerkey1{\keyline}\else{\namesecline}\fi}
}

\newcommand{\definition}[1]{{\sc Definition}.~~{#1}\vskip.2cm}

\usepackage[framemethod=default]{mdframed}
\global\mdfdefinestyle{red1}{linecolor=red, linewidth=1pt, leftmargin=1cm, rightmargin=1cm}
\global\mdfdefinestyle{black1}{linecolor=black, linewidth=1pt,} %leftmargin=.1cm, rightmargin=.1cm}

\newcommand{\solution}[2][]{\if\issolution0 #1 \else \begin{mdframed}[style=black1] #2 \end{mdframed} \fi}

\newcommand{\cmblanka}[1]{\if\issolution0 	\underline{\hskip1cm{\largeemptyspace}}
\else 						  		\underline{\hskip.35cm {#1}\hskip.35cm{\largeemptyspace}}\fi}
\newcommand{\sblanka}[1]{\if\issolution0 		\underline{\hskip1.5cm{\largeemptyspace}}
\else 						  		\underline{\hskip.25cm {#1}\hskip.25cm{\largeemptyspace}}\fi}
\newcommand{\mblanka}[1]{\if\issolution0 	\underline{\hskip3cm{\largeemptyspace}}
\else 						  		\underline{\hskip.5cm {#1}\hskip.5cm{\largeemptyspace}}\fi}
\newcommand{\lblanka}[1]{\if\issolution0 		\underline{\hskip4.5cm{\largeemptyspace}}
\else 						  		\underline{\hskip.75cm {#1}\hskip.75cm{\largeemptyspace}}\fi}
\newcommand{\Lblanka}[1]{\if\issolution0		\underline{\hskip6cm{\largeemptyspace}}
\else 								\underline{\hskip1cm {#1}\hskip1cm{\largeemptyspace}}\fi}
\newcommand{\LLblanka}[1]{\if\issolution0	\underline{\hskip7.5cm{\largeemptyspace}}
\else 								\underline{\hskip1.25cm {#1}\hskip1.25cm{\largeemptyspace}}\fi}
\newcommand{\LLLblanka}[1]{\if\issolution0 	\underline{\hskip9cm{\largeemptyspace}}
\else 								\underline{\hskip1.5cm {#1}\hskip1.5cm{\largeemptyspace}}\fi}
\newcommand{\tinyspacea}[1]{\if\issolution0 	\hskip.2cm{\largeemptyspace}
\else 						  		{#1}{\largeemptyspace}\fi}
\newcommand{\cmspacea}[1]{\if\issolution0 	\hskip1cm{\largeemptyspace}
\else 						  		\hskip.15cm {#1}\hskip.15cm{\largeemptyspace}\fi}
\newcommand{\sspacea}[1]{\if\issolution0 		\hskip1.5cm{\largeemptyspace}
\else 						  		\hskip.25cm {#1}\hskip.25cm{\largeemptyspace}\fi}
\newcommand{\mspacea}[1]{\if\issolution0 	\hskip3cm{\largeemptyspace}
\else 						  		\hskip.25cm {#1}\hskip.25cm{\largeemptyspace}\fi}
\newcommand{\lspacea}[1]{\if\issolution0 		\hskip4.5cm{\largeemptyspace}
\else 						  		\hskip.25cm {#1}\hskip.25cm{\largeemptyspace}\fi}
\newcommand{\Lspacea}[1]{\if\issolution0 	\hskip6cm{\largeemptyspace}
\else 						  		\hskip.25cm {#1}\hskip.25cm{\largeemptyspace}\fi}


\newcommand{\sparagraph}[1]{\vskip-1cm\paragraph{#1}}

\if\isanswerkey1\input{macsse474-key}\fi


\usetikzlibrary{positioning}

\begin{document}

\assignmentnametitlestuff


\if\isanswerkey0
{\bf Please follow the homework guidelines from A01.}

\fi


\begin{enumerate}

\item (3.15) Show that the collection of decidable languages is closed under the operation of
\begin{enumerate}
\item union
\item concatenation
\item star
\item complementation
\item intersection
\end{enumerate}
\solution{
\if\isanswerkey1\solDecidableClosures\fi
Assume we have a Turing machine $M_A$ that is decidable for language $A$ and a Turing machine that is decidable for language $B$. 
\begin{enumerate}
    \item For the union, we can build a TM $M$ that is decidable for language $A\cup B$. For a input string $s$, the TM is constructed as:
    \begin{itemize}
        \item Run input string $s$ on $M_A$
        \begin{itemize}
            \item If accepted by $M_A$ $\to$ accept
            \item If not, rewind to the beginning and run the string $s$ in $M_B$
            \begin{itemize}
                \item If accepted by $M_B$ $\to$ accept
                \item If not $\to$ reject
            \end{itemize}
        \end{itemize}
    \end{itemize}
    \item For the concatenation, we can build a TM $M$ that is decidable for language $AB$. For a input $s$, the TM is constructed as:
    \begin{itemize}
        \item Divide the string $s$ into two parts $s_1$ and $s_2$ 
        \item Run input string $s_1$ on $M_A$
        \begin{itemize}
            \item If $s_1$ is accepted by $M_A$, run string $s_2$ on $M_B$
            \begin{itemize}
                \item If $s_2$ accepted by $M_B$ $\to$ accept
                \item If not $\to$ reject
            \end{itemize}
            \item If not $\to$ reject
        \end{itemize}
    \end{itemize}
    \item For star, we can build a TM $M$ that is decidable for language $A^*$. For non-empty input string $s$, the TM is constructed as:
    \begin{enumerate}
        \item Divide the non-empty string $s$ into two parts $s_1$ and $s_2$, where string $s_2$ can be $\epsilon$
        \item Run $M_A$ on string $s_1$
        \begin{itemize}
            \item If string $s_1$ is accepted by $M_A$
            \begin{itemize}
                \item If the string $s_2$ is $\epsilon$ $\to$ accept
                \item If not, let $s_2$ be string $s$ in step ii, repeat step ii
            \end{itemize}
            \item If not $\to$ reject
        \end{itemize}
    \end{enumerate}
    \item For conplementation, we can build a TM $M$ that is decidable for language $\overline{A}$. The $M$ is constructed by making all accept states into reject state and reject states into accept state.
    \item For intersection, we can build a TM $M$ that is decidable for language $A \cap B$. For a input string $s$, the $M$ is built as:
    \begin{itemize}
        \item Run string $s$ on $M_A$
        \begin{itemize}
            \item If $s$ is accepted by $M_A$, rewind to the beginning and run it on $M_B$
            \begin{itemize}
                \item If accepted by $M_B$ $\to$ accept
                \item If not $\to$ reject
            \end{itemize}
            \item If not $\to$ reject
        \end{itemize}
    \end{itemize}
\end{enumerate}
}



\item (3.16abcd) Show that the collection of Turing-recognizable languages is closed under the operation of
\begin{enumerate}
\item union
\item concatenation
\item star
\item intersection
\end{enumerate}
\solution{
\if\isanswerkey1\solTRecognizableClosures\fi
Assume there is a Turing machine $M_A$ that is recognizable for language $A$ and a Turing machine $M_B$ that is recognizable for language $B$
\begin{enumerate}
    \item For the union, we can build a TM $M$ that is recognizable for language $A \cup B$. For input string $s$, it can be constructed as follow:
    \begin{itemize}
        \item Run $s$ on $M_A$ and and $M_B$ at the same time
        \begin{itemize}
            \item If either one of them accept $\to$ accept
            \item Otherwise $\to$ reject
        \end{itemize}
    \end{itemize}
    \item For concatenation, we can build a TM $M$ that is recognizable for language $AB$. For a input string $s$, $M$ is built as following:
    \begin{itemize}
        \item Divide the string $s$ into two parts, string $s_1$ and $s_2$
        \item Run $s_1$ on $M_A$
        \item If $s_1$ is accepted by $M_A$, run $s_2$ on $M_B$
        \item If both accepted $\to$ accept
        \item Otherwise $\to$ reject
    \end{itemize}
    \item For star, we can build a TM $M$ that is recognizable for language $A^*$. For a input string $s$, the $M$ can be constructed as following:
    \begin{enumerate}
        \item Split the string $s$ into $s_1$ and $s_2$, where $s_2$ can be empty
        \item Run $s_1$ in $M_A$
        \item If $M_A$ accepts $s_1$ and $s_2=\epsilon$ $\to$ accept
        \item If $s_2$ is not empty, let $s_2$ be string $s$ in step i, repeat step i
        \item If $M_A$ rejects $s_1$ or loop forever $\to$ reject
    \end{enumerate}
    \item For intersection, we can build a TM $M$ that recognizes the language $A \cap B$. For string $s$, the $M$ is constructed as:
    \begin{itemize}
        \item Run string $s$ in machine $M_A$
        \item If $M_A$ accepts $s$, rewind to the beginning and run in machine $M_B$
        \item If $M_B$ also accepts $s$ $\to$ accept
        \item Otherwise $\to$ reject
    \end{itemize}
\end{enumerate}
}

\item (3.18)
Note: {\bf Lexicographic order} of strings refers to the familiar dictionary order. The modified {\bf shortlex order}, also known as {\bf string order}, is identical to lexicographic, except that shorter strings precede longer strings: e.g., for $\Sigma = \{{\tt 0},{\tt 1}\}$, the elements of $\Sigma^*$ in string order would begin $\epsilon, {\tt 0}, {\tt 1}, {\tt 00}, {\tt 01}, {\tt 10}, {\tt 11}, {\tt 000}, \ldots$.

Show that a language is decidable iff some enumerator enumerates the language in the standard string order. (Hint: for the ($\Leftarrow$) direction, there are two cases, depending on whether the language is finite or infinite.)

\solution{
\if\isanswerkey1\solDecidableIffEnumerateStringOrder\fi
First show that TM $M$ decides language $A$ $\Rightarrow$ enumerator $E$ enumerates $A$:
\\Assume Turing machine $M$ recognizes language $A$, let strings $s_1$, $s_2$, $s_3$, ..., $s_i$, ..., $s_n$ be all strings of $\Sigma^*$ in the lexicographic order. So that there will be an enumerator which enumerates language $A$:
\\$E=$ ``On input $s_1$, $s_2$, ... $s_n$
\begin{enumerate}
    \item Repeat for $i=1, 2, 3, ..., n$
    \item Run string $s_i$ on machine $M$
    \item If $M$ accepts string $s_i$, prints $s_i$''
\end{enumerate}
So that $E$ will show all strings accepts by TM $M$ in the lexicographic order. So that 
\begin{equation*}
    \exists ~E \mid E ~\text{enumerates language} ~A
\end{equation*}
Hence $M$ decides $A$ $\Rightarrow$ $\exists ~E$ enumerates $A$
\\Next, we show that TM $M$ decides language $A\Leftarrow$ enumerator $E$ enumerates $A$:
\\If language $A$ is finite, $\exists ~M \mid M ~\text{decides language} ~A$ since for all finite language, there must be a Turing machine that decides it.
\\If language $A$ is infinite, we can construct a Turing machine $M$ that decides language $A$, where machine $M$ can be constructed as:
\\$M=$ ``On input string $s$,
\begin{enumerate}
    \item Run enumerator $E$
    \item Every time when $E$ prints a string, compare it with string $s$
    \item If the two string are equal $\to$ accept
    \item When all strings are printed, and no strings are equal $\to$ reject''
\end{enumerate}
So that $\exists ~M \mid M ~\textnormal{decides language } A$, hence enumerator $E$ enumerates $A$ $\Rightarrow$ $M$ decides $A$. In conclusion, $\exists ~M \mid M$ decides language $A$ $\Longleftrightarrow$ $\exists ~E \mid E$ enumerates $A$
}

\item (3.19) Show that every infinite Turing-recognizable language has an infinite decidable subset. (Hint: use the result of the previous problem.)
\solution{
\if\isanswerkey1\solInfiniteDecidableSubset\fi
From the previous problem, we concluded that $\exists ~M \mid M$ decides language $A$ $\Longleftrightarrow$ $\exists ~E \mid E$ enumerates $A$. Assume language $B$ is recognizable by Turing machine $M_B$, so that there will be an enumerator $E_B$ that recognizes language $B$. So for language $C \subseteq B$, since there is a $E_B$ that enumerates language $B$, since $C \subseteq B$, so there must be an enumerator $E_C$ that enumerates language $C$. Based on the conclusion from problem above, since there is a enumerator for language $C$, so $\exists ~M_C \mid M_C ~\textnormal{language } C \subseteq B$.
}

\item (3.22) Let $A$ be the language containing only the single string $s$, where
\[s = \begin{cases} {\tt 0} & \textnormal{if life never will be found on Mars.} \cr {\tt 1} & \textnormal{if life will be found on Mars someday.}\end{cases}\]
Is $A$ decidable? Why or why not? For the purposes of this problem, assume that the question of whether life will be found on Mars has an unambiguous YES or NO answer.
\solution{
\if\isanswerkey1\solLifeOnMars\fi
Since for language $A$, there is only two possible string $s$ for $A$ where $A$ can be either {\tt 0} or {\tt 1}. The two string are all finite, so that language $A$ is also finite. Since for a finite language, there must be a Turing machine that decides it. So language $A$ is Turing decidable.
}


\end{enumerate}





\end{document}
