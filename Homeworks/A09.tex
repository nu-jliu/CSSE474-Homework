\documentclass[11pt]{article}

\usepackage{enumitem}


%%TO EDIT
\newcommand{\dueclassnumber}{23}
\newcommand{\assignmentnum}{9}

% CHANGE issolution{0} to issolution{1} for homework submission.
% WRITE solutions inside the \solution{} commands.
% or, you can use \if\issolution1 … \fi

\def\issolution{1}
\def\myname{Allen Liu} % My name goes here
\def\mysec{01} % Section number goes here
\def\myCM{374} % Campus Mailbox goes here

\def\isanswerkey{0}

\newcommand\vv[2]{\begin{bmatrix} {#1} \cr {#2} \end{bmatrix}}

% Leave the next line alone. This is for my answer keys.
%\def\isanswerkey{1}

\usepackage{bbm,fancyhdr,ifthen,setspace,hyperref,url}
\usepackage{amssymb,amsmath,enumitem,amsthm,mathrsfs}
\usepackage{graphicx,xspace,color}
\usepackage{hhline}
\usepackage{tikz}
\usetikzlibrary{automata, positioning, arrows,chains,scopes,fit}
\tikzset{
->, % makes the edges directed
>=stealth', % makes the arrow heads bold
node distance=2.4cm, % specifies the minimum distance between two nodes. Change if necessary.
every state/.style={thick, fill=gray!10}, % sets the properties for each ’state’ node
initial text=$ $, % sets the text that appears on the start arrow
}

\topmargin=-.5in
\headsep=0.0in
\oddsidemargin=-.35in
\evensidemargin=-.65in
\textwidth=7.25in
\textheight=9.75in
\footskip=0in
\usepackage{titlesec}
\titlespacing*{\paragraph}{0pt}{2ex plus 1ex minus .2ex}{1ex}
\fancyhf{} % clear all header and footers
\renewcommand{\headrulewidth}{0pt} % remove the header rule
%\rfoot{\thepage}
\pagestyle{fancy}

\ifx\myname\undefined
\def\myname{}
\fi

\ifx\isanswerkey\undefined
\def\isanswerkey{0}
\fi

\if\isanswerkey1
\def\issolution{1}
\fi


\newcommand{\getcourseyear}[1]{2022}
\newcommand{\getcourseterm}{Spring \getcourseyear{1}}

\newcommand{\getclassmonthnum}[1]{\ifthenelse{#1<16}{3}{\ifthenelse{#1<29}{4}{5}}}
\newcommand{\getclassmonthshort}[1]{\ifthenelse{#1<16}{Mar}{\ifthenelse{#1<29}{Apr}{May}}}
\newcommand{\getclassmonth}[1]{\ifthenelse{#1<16}{March}{\ifthenelse{#1<29}{April}{May}}}
\newcommand{\getclassdayofmonth}[1]{\ifthenelse{
#1=1}{7}{\ifthenelse{
#1=2}{8}{\ifthenelse{
#1=3}{10}{\ifthenelse{
#1=4}{11}{\ifthenelse{
#1=5}{14}{\ifthenelse{
#1=6}{15}{\ifthenelse{
#1=7}{17}{\ifthenelse{
#1=8}{18}{\ifthenelse{
#1=9}{21}{\ifthenelse{
#1=10}{22}{\ifthenelse{
#1=11}{24}{\ifthenelse{
#1=12}{25}{\ifthenelse{
#1=13}{28}{\ifthenelse{
#1=14}{29}{\ifthenelse{
#1=15}{31}{\ifthenelse{
#1=16}{1}{\ifthenelse{
#1=17}{4}{\ifthenelse{
#1=18}{5}{\ifthenelse{
#1=19}{7}{\ifthenelse{
#1=20}{8}{\ifthenelse{
#1=21}{18}{\ifthenelse{
#1=22}{19}{\ifthenelse{
#1=23}{21}{\ifthenelse{
#1=24}{22}{\ifthenelse{
#1=25}{25}{\ifthenelse{
#1=26}{26}{\ifthenelse{
#1=27}{28}{\ifthenelse{
#1=28}{29}{\ifthenelse{
#1=29}{2}{\ifthenelse{
#1=30}{3}{\ifthenelse{
#1=31}{5}{\ifthenelse{
#1=32}{6}{\ifthenelse{
#1=33}{9}{\ifthenelse{
#1=34}{10}{\ifthenelse{
#1=35}{12}{\ifthenelse{
#1=36}{13}{\ifthenelse{
#1=37}{16}{\ifthenelse{
#1=38}{17}{\ifthenelse{
#1=39}{19}{\ifthenelse{
#1=40}{20}{}}}}}}}}}}}}}}}}}}}}}}}}}}}}}}}}}}}}}}}}}

\newcommand{\getclassdate}[1]{\getclassmonth{#1}\xspace\getclassdayofmonth{#1}}
\newcommand{\getclassdateshort}[1]{\getclassmonthshort{#1}\xspace\getclassdayofmonth{#1}}
\newcommand{\getclassdatenum}[1]{\getclassmonthnum{#1}/\getclassdayofmonth{#1}}
\newcommand{\getMday}{Mon}
\newcommand{\getTday}{Tue}
\newcommand{\getWday}{Wed}
\newcommand{\getRday}{Thu}
\newcommand{\getFday}{Fri}
\newcommand{\getclassdayofweek}[1]{\ifthenelse{
#1=1}{\getMday}{\ifthenelse{#1=2}{\getTday}{\ifthenelse{#1=3}{\getRday}{\ifthenelse{#1=4}{\getFday}{\ifthenelse{
#1=5}{\getMday}{\ifthenelse{#1=6}{\getTday}{\ifthenelse{#1=7}{\getRday}{\ifthenelse{#1=8}{\getFday}{\ifthenelse{
#1=9}{\getMday}{\ifthenelse{#1=10}{\getTday}{\ifthenelse{#1=11}{\getRday}{\ifthenelse{#1=12}{\getFday}{\ifthenelse{
#1=13}{\getMday}{\ifthenelse{#1=14}{\getTday}{\ifthenelse{#1=15}{\getRday}{\ifthenelse{#1=16}{\getFday}{\ifthenelse{
#1=17}{\getMday}{\ifthenelse{#1=18}{\getTday}{\ifthenelse{#1=19}{\getRday}{\ifthenelse{#1=20}{\getFday}{\ifthenelse{
#1=21}{\getMday}{\ifthenelse{#1=22}{\getTday}{\ifthenelse{#1=23}{\getRday}{\ifthenelse{#1=24}{\getFday}{\ifthenelse{
#1=25}{\getMday}{\ifthenelse{#1=26}{\getTday}{\ifthenelse{#1=27}{\getRday}{\ifthenelse{#1=28}{\getFday}{\ifthenelse{
#1=29}{\getMday}{\ifthenelse{#1=30}{\getTday}{\ifthenelse{#1=31}{\getRday}{\ifthenelse{#1=32}{\getFday}{\ifthenelse{
#1=33}{\getMday}{\ifthenelse{#1=34}{\getTday}{\ifthenelse{#1=35}{\getRday}{\ifthenelse{#1=36}{\getFday}{\ifthenelse{
#1=37}{\getMday}{\ifthenelse{#1=38}{\getTday}{\ifthenelse{#1=39}{\getRday}{\ifthenelse{#1=40}{\getFday}\xspace
}}}}}}}}}}}}}}}}}}}}}}}}}}}}}}}}}}}}}}}}
\newcommand{\examdate}[1]{\ifthenelse{#1=1}{March 23, \getcourseyear{11}}{\ifthenelse{#1=2}{April 7, \getcourseyear{11}}{\ifthenelse{#1=3}{May 4, \getcourseyear{11}}{\ifthenelse{#1=4}{ } {}}}}}




\newcommand{\classdate}[1]{\getclassdate{#1}, \getcourseyear{#1}}
\newcommand{\dueclassdate}{\getclassdayofweek{\dueclassnumber} \getclassdateshort{\dueclassnumber}}


\newcommand\largeemptyspace{\vphantom{\textnormal{$\ds\int$}}}
\newcommand\nameblank{\if\issolution0\underline{\hskip11.25cm {\largeemptyspace}}
  \else\underline{\hskip.2cm{\LARGE\myname}\hskip6cm}\fi}
\newcommand\nameblankshort{\if\issolution0\underline{\hskip9.5cm {\largeemptyspace}}
  \else\underline{\hskip.2cm{\LARGE\myname {\largeemptyspace}}\hskip.2cm}\fi}
\newcommand\secblank{\if\issolution0\underline{\hskip1.5cm{\largeemptyspace}}
  \else\underline{\hskip.2cm{\LARGE\mysec {\largeemptyspace}}\hskip.2cm} \hskip.8cm \fi}
\newcommand\CMblank{\if\issolution0\underline{\hskip2.25cm{\largeemptyspace}}
  \else\underline{\hskip.2cm{\LARGE\myCM {\largeemptyspace}}\hskip.2cm}\fi}

\newcommand{\namegroupline}{Name: \nameblank Group \#: \underline{\hskip1.5cm{\largeemptyspace}}}
\newcommand{\nameline}{\begin{minipage}{0.6\linewidth} Name: \nameblank \end{minipage}}
\newcommand{\namelineshort}{Name: \nameblankshort}
\newcommand{\namesecline}{\begin{minipage}{0.7\linewidth} Name: \nameblank \end{minipage} \hfill \begin{minipage}{0.29\linewidth}Section \#: \secblank\end{minipage}}
\newcommand{\namesecCMline}{\begin{minipage}{0.6\linewidth}Name: \nameblankshort \end{minipage} \hfill \begin{minipage}{0.4\linewidth} Section \# \secblank CM\# \CMblank \end{minipage}}
\newcommand{\keyline}{{\color{red} SOLUTION KEY}}
%\newcommand{\nameline}{Name: \rule{11.5cm}{0.01cm} \hfill Section: \rule{1.5cm}{0.01cm}}
%\newcommand{\keyline}{Name: \rule{4cm}{0.01cm} SOLUTION KEY \rule{4cm}{0.01cm} \hfill Section: \rule{1.5cm}{0.01cm}}
%\newcommand{\groupline}{Group members present: \rule{8.5cm}{0.01cm} \hfill Group \#: \rule{1.5cm}{0.01cm}}
\newcommand{\course}{CSSE/MA 474\xspace}
\newcommand{\coursewithname}{CSSE/MA 474. Theory of Computation\xspace}


\newcommand{\wtitlestuff}{
\if\isanswerkey0
  \nameline
  \else
  \keyline
\fi
\begin{center}
\large \course Worksheet for Class \#\classnumber\\
\small \classdate{\classnumber}
\normalsize
\end{center}}

\newcommand{\lectitlestuff}{
\begin{center}
\Large \course Lecture \#\classnumber\\
\vskip 3pt \small Nate Chenette \\ \classdate{\classnumber}
\normalsize
\end{center}}

\newcommand{\othertitlestuff}{
\begin{center}
\Large \othertitle\\
\small \coursewithname\\
Class \#\classnumber, \classdate{\classnumber}\\
\normalsize
\end{center}}

\newcommand{\othertitlestuffnodate}{
\begin{center}
\Large \othertitle\\
\small \coursewithname\\
\normalsize
\end{center}}

\newcommand{\othernametitlestuff}{
\nameline\\
\othertitlestuff
}

\newcommand{\assignmenttitlestuff}{
\begin{center}
\Large \course Assignment \assignmentnum\\
\small Due date: \dueclassdate
\normalsize
\end{center}}

\newcommand{\assignmentnametitlestuff}{
\if\isanswerkey0
  \namesecCMline
  \else
  \keyline
\fi
\assignmenttitlestuff
}

\newcommand{\quiznametitlestuff}{
\if\isanswerkey0
  \namesecCMline
  \else
  \keyline
\fi
\begin{center}
\Large \course Quiz \quiznum\\
\small \classdate{\classnumber}
\normalsize
\end{center}}


\setlength{\parindent}{0in}
\setlength{\fboxsep}{.1in}

\renewcommand{\emptyset}{\varnothing}
\newcommand{\tvs}{\textvisiblespace}
\newcommand{\brk}{\vskip.2cm \hrule \vskip.2cm}
\newcommand{\ds}{\displaystyle}
\newcommand{\abs}[1]{\left\lvert {#1}\right\rvert}
\newcommand{\Lsym}{\text{L}}
\newcommand{\Rsym}{\text{R}}
\newcommand{\qacc}{q_{\textnormal{accept}}}
\newcommand{\qrej}{q_{\textnormal{reject}}}
\newcommand{\tmRej}{$\to$ \textbf{\textit{reject}}}
\newcommand{\tmAcc}{$\to$ \textbf{\textit{accept}}}
\def\lep{\le_\textnormal{P}}
\def\lem{\le_\textnormal{m}}
\def\ATM{A_\textnormal{TM}}
\newcommand{\vv}[2]{\begin{bmatrix} {#1} \cr {#2} \end{bmatrix}}
\newcommand{\vvt}[2]{\begin{bmatrix} {\tt #1} \cr {\tt #2} \end{bmatrix}}
\def\hs{\quad \texttt\#\quad }
\def\multiset#1#2{\ensuremath{\left(\kern-.3em\left(\genfrac{}{}{0pt}{}{#1}{#2}\right)\kern-.3em\right)}}
\def\time{\textsf{TIME}}
\def\ntime{\textsf{NTIME}}
\def\P{\textsf{P}}
\def\NP{\textsf{NP}}
   
%logic
\newcommand{\se}{\big|}
\newcommand{\lra}{\leftrightarrow}
\newcommand{\Lra}{\Leftrightarrow}
\newcommand{\we}{\wedge}
\def\thf{%
   \leavevmode
   \lower0.2ex\hbox{$\cdot$}%
   \kern-0.0em\raise0.7ex\hbox{$\cdot$}%
   \kern-0.0em\lower0.2ex\hbox{$\cdot$}%
   \thinspace}

%Number Systems
\newcommand{\bbZ}{\mathbb{Z}}
\newcommand{\bfZ}{\mathbf{Z}}
\newcommand{\bfZp}{\mathbf{Z}^+}
\newcommand{\bbN}{\mathbb{N}}
\newcommand{\bfN}{\mathbf{N}}
\newcommand{\bbQ}{\mathbb{Q}}
\newcommand{\bfQ}{\mathbf{Q}}
\newcommand{\bbR}{\mathbb{R}}
\newcommand{\bfR}{\mathbf{R}}
\newcommand{\bbC}{\mathbb{C}}
\newcommand{\bfC}{\mathbf{C}}

%sets
\newcommand{\U}{\mathscr{U}}
\newcommand{\ol}[1]{\overline{#1}}
\newcommand{\ssq}{\subseteq}
\newcommand{\sst}{\subset}
\def\ps{\mathcal{P}}
\def\sd{\,\triangle\,}
\def\sdonly{\triangle}
\def\es{\emptyset}

%cards
\def\hst{\heartsuit}
\def\cst{\clubsuit}
\def\sst{\spadesuit}
\def\dst{\diamondsuit}


\newcommand{\lcm}{{\rm lcm}}


\newcommand{\imgdir}{../images/}


\newcommand{\makeexamcover}{
\ifdefined\finalexam
\ \vskip2cm
\begin{center}
\huge \course Final Exam \\
\Large \finalexamdate \vskip1cm
\end{center}
\normalsize \instructions \vskip1cm
\begin{spacing}{1.5}
\begin{center}
\scorechart
\end{center}
\end{spacing}
\else \ifdefined\examnum
\ \vskip2cm
\begin{center}
\huge \course Exam \examnum \\
\Large \examdate{\examnum} \vskip1cm
\end{center}
\normalsize \instructions \vskip1cm
\begin{spacing}{1.5}
\begin{center}
\scorechart
\end{center}
\end{spacing}
\fi
\fi
}




\fancypagestyle{examcover}{% 
\fancyhf{}
\renewcommand{\footrulewidth}{0pt}
\lhead{\if\isanswerkey1{\keyline}\else{\nameline}\fi}
%\lhead{\if\isanswerkey1{\keyline}\else{\namesecline}\fi}
}



\fancypagestyle{exameverypage}{% 
\fancyhf{}
\renewcommand{\footrulewidth}{0pt}
\rhead{\if\isanswerkey1{\keyline}\else{}\fi}
\fancyfoot[R]{\thepage}
%\lhead{\if\isanswerkey1{\keyline}\else{\namesecline}\fi}
}

\newcommand{\definition}[1]{{\sc Definition}.~~{#1}\vskip.2cm}

\usepackage[framemethod=default]{mdframed}
\global\mdfdefinestyle{red1}{linecolor=red, linewidth=1pt, leftmargin=1cm, rightmargin=1cm}
\global\mdfdefinestyle{black1}{linecolor=black, linewidth=1pt,} %leftmargin=.1cm, rightmargin=.1cm}

\newcommand{\solution}[2][]{\if\issolution0 #1 \else \begin{mdframed}[style=black1] #2 \end{mdframed} \fi}

\newcommand{\cmblanka}[1]{\if\issolution0 	\underline{\hskip1cm{\largeemptyspace}}
\else 						  		\underline{\hskip.35cm {#1}\hskip.35cm{\largeemptyspace}}\fi}
\newcommand{\sblanka}[1]{\if\issolution0 		\underline{\hskip1.5cm{\largeemptyspace}}
\else 						  		\underline{\hskip.25cm {#1}\hskip.25cm{\largeemptyspace}}\fi}
\newcommand{\mblanka}[1]{\if\issolution0 	\underline{\hskip3cm{\largeemptyspace}}
\else 						  		\underline{\hskip.5cm {#1}\hskip.5cm{\largeemptyspace}}\fi}
\newcommand{\lblanka}[1]{\if\issolution0 		\underline{\hskip4.5cm{\largeemptyspace}}
\else 						  		\underline{\hskip.75cm {#1}\hskip.75cm{\largeemptyspace}}\fi}
\newcommand{\Lblanka}[1]{\if\issolution0		\underline{\hskip6cm{\largeemptyspace}}
\else 								\underline{\hskip1cm {#1}\hskip1cm{\largeemptyspace}}\fi}
\newcommand{\LLblanka}[1]{\if\issolution0	\underline{\hskip7.5cm{\largeemptyspace}}
\else 								\underline{\hskip1.25cm {#1}\hskip1.25cm{\largeemptyspace}}\fi}
\newcommand{\LLLblanka}[1]{\if\issolution0 	\underline{\hskip9cm{\largeemptyspace}}
\else 								\underline{\hskip1.5cm {#1}\hskip1.5cm{\largeemptyspace}}\fi}
\newcommand{\tinyspacea}[1]{\if\issolution0 	\hskip.2cm{\largeemptyspace}
\else 						  		{#1}{\largeemptyspace}\fi}
\newcommand{\cmspacea}[1]{\if\issolution0 	\hskip1cm{\largeemptyspace}
\else 						  		\hskip.15cm {#1}\hskip.15cm{\largeemptyspace}\fi}
\newcommand{\sspacea}[1]{\if\issolution0 		\hskip1.5cm{\largeemptyspace}
\else 						  		\hskip.25cm {#1}\hskip.25cm{\largeemptyspace}\fi}
\newcommand{\mspacea}[1]{\if\issolution0 	\hskip3cm{\largeemptyspace}
\else 						  		\hskip.25cm {#1}\hskip.25cm{\largeemptyspace}\fi}
\newcommand{\lspacea}[1]{\if\issolution0 		\hskip4.5cm{\largeemptyspace}
\else 						  		\hskip.25cm {#1}\hskip.25cm{\largeemptyspace}\fi}
\newcommand{\Lspacea}[1]{\if\issolution0 	\hskip6cm{\largeemptyspace}
\else 						  		\hskip.25cm {#1}\hskip.25cm{\largeemptyspace}\fi}


\newcommand{\sparagraph}[1]{\vskip-1cm\paragraph{#1}}

\if\isanswerkey1\input{macsse474-key}\fi


\usetikzlibrary{positioning}

\begin{document}

\assignmentnametitlestuff


\if\isanswerkey0
{\bf Please follow the homework guidelines from A01.}

\fi

In this assignment, you may use any TM constructions from class, such as $D_{\textnormal{A-DFA}}$, $D_{\textnormal{A-NFA}}$,$D_\textnormal{E-DFA}$, $D_\textnormal{EQ-DFA}$, $D_\textnormal{EQ-REX}$. You may also use any known results from previous chapters of the textbook, e.g. that from any regular expression $R$, one can construct (and in particular, a Turing machine can construct) a DFA recognizing $L(R)$, likewise for CFGs and PDAs, etc.

To show a language is decidable, it is sufficient to give a high-level description of a Turing machine that decides the language.



\begin{enumerate}
\if\isanswerkey0

\item[0.] (4.1)  {\bf Warm-up. You don't need to submit this problem. But please do it, and check your answers on the back of this sheet.} 

Answer all parts for the following DFA $M$ and give reasons for your answers.

\begin{minipage}{.5\linewidth}
\begin{enumerate}
\item Is $\langle M, {\tt 0100} \rangle \in A_\textnormal{DFA}$?
\item Is $\langle M, {\tt 011} \rangle \in A_\textnormal{DFA}$?
\item Is $\langle M \rangle \in A_\textnormal{DFA}$?
\item Is $\langle M, {\tt 0100} \rangle \in A_\textnormal{REX}$?
\item Is $\langle M \rangle \in E_\textnormal{DFA}$?
\item Is $\langle M,M \rangle \in EQ_\textnormal{DFA}$?
\end{enumerate}
\end{minipage}
\begin{minipage}{.5\linewidth}
\begin{tikzpicture}
\node[state, initial,accepting] at (0,0) (0) {};
\node[state] at (3,-0.5) (1) {};
\node[state] at (1.0,-3) (2) {};
\draw 
(0) edge[loop above] node{${\tt 0}$} (0)
(0) edge[bend left, above] node{${\tt 1}$} (1)
(1) edge[bend left, below right] node{${\tt 0}, {\tt 1}$} (2)
(2) edge[bend left, above left] node{${\tt 1}$} (1)
(2) edge[bend left, below left] node{${\tt 0}$} (0)
;
\end{tikzpicture}
\end{minipage}
\fi

\item (4.2) Consider the problem of determining whether a DFA and a regular expression are equivalent. Express this problem as a language and show that it is decidable.
\solution{
\if\isanswerkey1\solEquivDFARegExDecidable\fi
Define the problem as language $REG_\textnormal{DFA}$ where
\begin{equation*}
    REG_\textnormal{DFA}=\left\{\langle M, R\rangle\mid M~\textnormal{is a DFA and }R~\textnormal{is a regular expression where }L\left(M\right)=L\left(R\right)\right\}
\end{equation*}
Since by \textbf{Theorem 4.5}
\begin{displayquote}
    Let $EQ_\textnormal{DFA}=\left\{\langle A, B\rangle\mid A~\textnormal{and}B\textnormal{ are DFAs and }L\left(A\right)=L\left(B\right)\right\}$, $EQ_{DFA}$ is a decidable language
\end{displayquote}
So that we can build a TM that decides whether two DFAs are equivalent, and we also know all regular expression has its equivalent DFA. Hence, there is a Turing-machine $D_\textnormal{REG-DFA}$ that decides the language $REG_\textnormal{DFA}$. Let $D_R$ be the equivalent DFA of regular expression $R$, similar to the \textsc{Proof} of \textbf{Theorem 4.5}, we can define a language $C$ where 
\begin{equation}
    L(C)=\left(L\left(M\right)\cap \overline{L\left(D_R\right)}\right) \cup \left(\overline{L\left(M\right)}\cap L\left(D_R\right)\right)\label{eqn:eq_DFA}
\end{equation}
$D_\textnormal{REG-DFA}$ is constructed based on \textsc{Proof} of \textbf{Theorem 4.5} and \textbf{Theorem 4.4}
\begin{displayquote}
    Let $E_{DFA}=\left\{\langle A\rangle\mid A\textnormal{ is a DFA and }L\left(A\right)=\emptyset\right\}$, $E_\textnormal{DFA}$ is a decidable language
\end{displayquote}
$D_\textnormal{REG-DFA}=$``On input $\langle M,R\rangle$, where $M$ is a DFA and $R$ is a regular expression:
\begin{enumerate}
    \item Convert regular expression $R$ into its equivalent DFA $M_R$
    \item Construct DFA $C$ as described in eqn. \ref{eqn:eq_DFA}
    \item Mark the start state of $C$
    \begin{enumerate}
        \item Mark all new states reachable from all marked state
    \end{enumerate}
    \item If there is new state marked, repeat step i
    \item \begin{itemize}
        \item If no accept state is marked $\to$ \textbf{\textit{accept}}
        \item Otherwise $\to$ \textbf{\textit{reject}}''
    \end{itemize}
\end{enumerate}
}

\item (4.3) Let $ALL_\textnormal{DFA} = \{\langle A \rangle \mid A \textnormal{ is a DFA and } L(A) = \Sigma^*\}$.\\ Show that $ALL_\textnormal{DFA}$ is decidable.
\solution{
\if\isanswerkey1\solAllDFADecidable\fi
For language $ALL_\textnormal{DFA}$, we can build a Turing-machine $D_\textnormal{ALL-DFA}$ that decides it, where machine $D_\textnormal{All-DFA}$ is defined as:\\
$D_\textnormal{All-DFA}=$``On input $\langle A\rangle$, $A$ is a DFA
\begin{enumerate}
    \item Mark the start state of $A$, $q_0$
    \item Check if $q_0$ is accept state, if not $\to$ \textbf{\textit{reject}}
    \begin{enumerate}
        \item Mark all new states reachable from all marked state
        \item If any one of new marked state is not an accept state $\to$ \textbf{\textit{reject}}
    \end{enumerate}
    \item If there is new state marked, repeat steps i$-$ii
    \item 
    \begin{itemize}
        \item If all marked states are accept state $\to$ \textbf{\textit{accept}}
        \item Otherwise $\to$ \textbf{\textit{reject}}''
    \end{itemize}
\end{enumerate}
}

\item (4.10) Let $INF_\textnormal{DFA} = \{\langle A \rangle \mid A \textnormal{ is a DFA and } L(A) \textnormal{ is an infinite language}\}$.\\ Show that $INF_\textnormal{DFA}$ is decidable.
\solution{
\if\isanswerkey1\solInfDFADecidable\fi
For language $INF_\textnormal{DFA}$, we can build a Turing-machine $D_\textnormal{INF-DFA}$ that decides it, where machine $D_\textnormal{INF-DFA}$ is defined as:\\
$D_\textnormal{INF-DFA}=$``On input $\langle A\rangle$, $A$ is a DFA
\begin{enumerate}
    \item Mark the start state of $A$ as \texttt{x}
    \item Mark all new states reachable from all states marked \texttt{x} along the unmarked edge as \texttt{x}
    \item Mark all states marked \texttt{x} reachable from all marked marked \texttt{x} along the unmarked edge as \texttt{y}
    \item Mark all new edge taken
    \item If any new states has been marked as \texttt{y}, perform following steps starting from all the new states marked \texttt{y}
    \begin{enumerate}
        \item Mark the all states marked \texttt{x} that is reachable from those new states as \texttt{z}
        \item Repeat step iii until no new states has been marked as \texttt{z}
        \item Mark the states marked \texttt{x} that is reachable from states marked \texttt{z} as \texttt{z}
        \item Clear mark on all states marked with \texttt{z}
    \end{enumerate}
    \item If there is new state marked, repeat steps (c)$-$(e)
    \item 
    \begin{itemize}
        \item If any of state marked with \texttt{x} is accept state $\to$ \textbf{\textit{reject}}
        \item Otherwise $\to$ \textbf{\textit{accept}}''
    \end{itemize}
\end{enumerate}
}

\item (4.13) Let $A = \{\langle R,S \rangle \mid \textnormal{$R$ and $S$ are regular expressions and $L(R) \subseteq L(S)$}\}$.\\ Show that $A$ is decidable.
\solution{
\if\isanswerkey1\solRegExpsSubsetDecidable\fi
Let $D_R$ be the equivalent DFA of regular expression $R$, and $D_S$ be the equivalent DFA of $S$. Similar to the \textsc{Proof} of \textbf{Theorem 4.4} and \textbf{Theorem 4.5}, we can define a language $C$ where
\begin{equation}
    L\left(C\right)=\overline{L\left(D_S\right)}\cap L\left(D_R\right) \label{eqn:subset}
\end{equation}
Then we can construct a Turing-machine that recognizes language $A$, $D_A$ as\\
$D_A=$``On input $\langle R,S\rangle$, $R$ and $S$ are regular expression
\begin{enumerate}
    \item Convert regular expression $R$ and $S$ to their equivalent DFA, $D_R$ and $D_S$
    \item Construct DFA $C$ defined in eqn. \ref{eqn:subset}
    \item Mark the start state of $C$
    \item Repeat the step i until no new states has been marked
    \begin{enumerate}
        \item Mark the new state reachable from all marked state
    \end{enumerate}
    \item \begin{itemize}
        \item If any of marked state is accept state $\to$ \textbf{\textit{reject}}
        \item Otherwise $\to$ \textbf{\textit{accept}}''
    \end{itemize}
\end{enumerate}
}

\item (4.26) Let $PAL_\textnormal{DFA} = \{ \langle M \rangle \mid M \textnormal{ is a DFA that accepts some palindrome}\}$.\\ Show that $PAL_\textnormal{DFA}$ is decidable. Hint: a result involving CFLs from Worksheet 14 can be helpful.
\solution{
\if\isanswerkey1\solDFAAcceptingPalindromeDecidable\fi
For language $A=\{w\mid w \textnormal{ is a palindrome}\}$, there is a CFG that describe it
\begin{align*}
    S &\to T\\
    T &\to {\tt 0}T{\tt 0} \mid {\tt 1}T{\tt 1}\mid X\\
    X &\to {\tt 0} \mid {\tt 1} \mid \epsilon
\end{align*}
$A$ is a CFL. 
Let $L_\textnormal{reg}$ denote the \textbf{\textit{regular languages}}, since $M$ is DFA, so that $L(M)\textnormal{ is a } L_\textnormal{reg}$. Since $L_\textnormal{reg} \subseteq \textnormal{CFL}$, $L(M)$ is also a CFL. Let's define PDA $M_A$ that $A=L(M_A)$. As language $A$ and $L(M)$ are all CFL, so that
\begin{equation}
    \exists ~\textnormal{CFG }C \mid L(C)=L(M) \cap L(M_A)\label{eqn:cfg}
\end{equation}
Since by \textbf{Theorem 4.8}
\begin{displayquote}
    Let $E_\textnormal{CFG}=\left\{\langle G\rangle \mid G \textnormal{ is a CFG and }L(G)=\emptyset\right\}$, $E_\textnormal{CFG}$ is a decidable language
\end{displayquote}
There is a TM $D_\textnormal{E-CFG}$ that decides if a CFG is empty. So that we can construct the TM $D_\textnormal{PAL-DFA}$ that decides language $PAL_\textnormal{DFA}$ based on the \textsc{Proof} of \textbf{Theorem 4.8} where\\
$D_\textnormal{PAL-DFA}=$``On input $\langle M\rangle$, where $M$ is a DFA
\begin{enumerate}
    \item Construct a PDA $M_A$ where $L(M_A)=$ \{$w\mid w$ is a palindrome\}
    \item Construct PDA $C$ as defined in eqn. \ref{eqn:cfg}
    \item Mark all terminals in $C$
    \item Repeat the following step until no new variables are marked
    \begin{enumerate}
        \item Mark all variable $A$ in $C$ such that there exist a rule $A\to U_1U_2...U_i...U_k$, where for $i=1...k$, all $U_i$ has been marked
    \end{enumerate}
    \item \begin{itemize}
        \item If the start variable in $C$, $S$ is marked $\to$ \textbf{\textit{accept}}
        \item Otherwise $\to$ \textbf{\textit{reject}}''
    \end{itemize}
\end{enumerate}
}

\end{enumerate}

\if\isanswerkey0
\pagebreak

\subsubsection*{Answers}
\begin{enumerate}
\item[0.] Answers to warm-up problem:

\begin{enumerate}
\item Yes, because $M$ is a valid DFA, and it accepts ${\tt 0100}$.
\item No, because while $M$ is a valid DFA, it rejects ${\tt 011}$.
\item No, the input has only a single component and thus is not in the correct form.
\item No, the first part of the input is not a regular expression, so the input is not in the correct form.
\item No, the language of $M$ is not empty.
\item Yes, because $M$ accepts the same language as itself.
\end{enumerate}

\end{enumerate}
\fi



\end{document}
