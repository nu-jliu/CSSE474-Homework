\documentclass[11pt]{article}

\usepackage{enumitem}


%%TO EDIT
\newcommand{\dueclassnumber}{25}
\newcommand{\assignmentnum}{10}

% CHANGE issolution{0} to issolution{1} for homework submission.
% WRITE solutions inside the \solution{} commands.
% or, you can use \if\issolution1 … \fi

\def\issolution{1}
\def\myname{Allen Liu} % My name goes here
\def\mysec{01} % Section number goes here
\def\myCM{374} % Campus Mailbox goes here

\def\isanswerkey{0}

\newcommand\vv[2]{\begin{bmatrix} {#1} \cr {#2} \end{bmatrix}}

% Leave the next line alone. This is for my answer keys.
%\def\isanswerkey{1}

\usepackage{bbm,fancyhdr,ifthen,setspace,hyperref,url}
\usepackage{amssymb,amsmath,enumitem,amsthm,mathrsfs}
\usepackage{graphicx,xspace,color}
\usepackage{hhline}
\usepackage{tikz}
\usetikzlibrary{automata, positioning, arrows,chains,scopes,fit}
\tikzset{
->, % makes the edges directed
>=stealth', % makes the arrow heads bold
node distance=2.4cm, % specifies the minimum distance between two nodes. Change if necessary.
every state/.style={thick, fill=gray!10}, % sets the properties for each ’state’ node
initial text=$ $, % sets the text that appears on the start arrow
}

\topmargin=-.5in
\headsep=0.0in
\oddsidemargin=-.35in
\evensidemargin=-.65in
\textwidth=7.25in
\textheight=9.75in
\footskip=0in
\usepackage{titlesec}
\titlespacing*{\paragraph}{0pt}{2ex plus 1ex minus .2ex}{1ex}
\fancyhf{} % clear all header and footers
\renewcommand{\headrulewidth}{0pt} % remove the header rule
%\rfoot{\thepage}
\pagestyle{fancy}

\ifx\myname\undefined
\def\myname{}
\fi

\ifx\isanswerkey\undefined
\def\isanswerkey{0}
\fi

\if\isanswerkey1
\def\issolution{1}
\fi


\newcommand{\getcourseyear}[1]{2022}
\newcommand{\getcourseterm}{Spring \getcourseyear{1}}

\newcommand{\getclassmonthnum}[1]{\ifthenelse{#1<16}{3}{\ifthenelse{#1<29}{4}{5}}}
\newcommand{\getclassmonthshort}[1]{\ifthenelse{#1<16}{Mar}{\ifthenelse{#1<29}{Apr}{May}}}
\newcommand{\getclassmonth}[1]{\ifthenelse{#1<16}{March}{\ifthenelse{#1<29}{April}{May}}}
\newcommand{\getclassdayofmonth}[1]{\ifthenelse{
#1=1}{7}{\ifthenelse{
#1=2}{8}{\ifthenelse{
#1=3}{10}{\ifthenelse{
#1=4}{11}{\ifthenelse{
#1=5}{14}{\ifthenelse{
#1=6}{15}{\ifthenelse{
#1=7}{17}{\ifthenelse{
#1=8}{18}{\ifthenelse{
#1=9}{21}{\ifthenelse{
#1=10}{22}{\ifthenelse{
#1=11}{24}{\ifthenelse{
#1=12}{25}{\ifthenelse{
#1=13}{28}{\ifthenelse{
#1=14}{29}{\ifthenelse{
#1=15}{31}{\ifthenelse{
#1=16}{1}{\ifthenelse{
#1=17}{4}{\ifthenelse{
#1=18}{5}{\ifthenelse{
#1=19}{7}{\ifthenelse{
#1=20}{8}{\ifthenelse{
#1=21}{18}{\ifthenelse{
#1=22}{19}{\ifthenelse{
#1=23}{21}{\ifthenelse{
#1=24}{22}{\ifthenelse{
#1=25}{25}{\ifthenelse{
#1=26}{26}{\ifthenelse{
#1=27}{28}{\ifthenelse{
#1=28}{29}{\ifthenelse{
#1=29}{2}{\ifthenelse{
#1=30}{3}{\ifthenelse{
#1=31}{5}{\ifthenelse{
#1=32}{6}{\ifthenelse{
#1=33}{9}{\ifthenelse{
#1=34}{10}{\ifthenelse{
#1=35}{12}{\ifthenelse{
#1=36}{13}{\ifthenelse{
#1=37}{16}{\ifthenelse{
#1=38}{17}{\ifthenelse{
#1=39}{19}{\ifthenelse{
#1=40}{20}{}}}}}}}}}}}}}}}}}}}}}}}}}}}}}}}}}}}}}}}}}

\newcommand{\getclassdate}[1]{\getclassmonth{#1}\xspace\getclassdayofmonth{#1}}
\newcommand{\getclassdateshort}[1]{\getclassmonthshort{#1}\xspace\getclassdayofmonth{#1}}
\newcommand{\getclassdatenum}[1]{\getclassmonthnum{#1}/\getclassdayofmonth{#1}}
\newcommand{\getMday}{Mon}
\newcommand{\getTday}{Tue}
\newcommand{\getWday}{Wed}
\newcommand{\getRday}{Thu}
\newcommand{\getFday}{Fri}
\newcommand{\getclassdayofweek}[1]{\ifthenelse{
#1=1}{\getMday}{\ifthenelse{#1=2}{\getTday}{\ifthenelse{#1=3}{\getRday}{\ifthenelse{#1=4}{\getFday}{\ifthenelse{
#1=5}{\getMday}{\ifthenelse{#1=6}{\getTday}{\ifthenelse{#1=7}{\getRday}{\ifthenelse{#1=8}{\getFday}{\ifthenelse{
#1=9}{\getMday}{\ifthenelse{#1=10}{\getTday}{\ifthenelse{#1=11}{\getRday}{\ifthenelse{#1=12}{\getFday}{\ifthenelse{
#1=13}{\getMday}{\ifthenelse{#1=14}{\getTday}{\ifthenelse{#1=15}{\getRday}{\ifthenelse{#1=16}{\getFday}{\ifthenelse{
#1=17}{\getMday}{\ifthenelse{#1=18}{\getTday}{\ifthenelse{#1=19}{\getRday}{\ifthenelse{#1=20}{\getFday}{\ifthenelse{
#1=21}{\getMday}{\ifthenelse{#1=22}{\getTday}{\ifthenelse{#1=23}{\getRday}{\ifthenelse{#1=24}{\getFday}{\ifthenelse{
#1=25}{\getMday}{\ifthenelse{#1=26}{\getTday}{\ifthenelse{#1=27}{\getRday}{\ifthenelse{#1=28}{\getFday}{\ifthenelse{
#1=29}{\getMday}{\ifthenelse{#1=30}{\getTday}{\ifthenelse{#1=31}{\getRday}{\ifthenelse{#1=32}{\getFday}{\ifthenelse{
#1=33}{\getMday}{\ifthenelse{#1=34}{\getTday}{\ifthenelse{#1=35}{\getRday}{\ifthenelse{#1=36}{\getFday}{\ifthenelse{
#1=37}{\getMday}{\ifthenelse{#1=38}{\getTday}{\ifthenelse{#1=39}{\getRday}{\ifthenelse{#1=40}{\getFday}\xspace
}}}}}}}}}}}}}}}}}}}}}}}}}}}}}}}}}}}}}}}}
\newcommand{\examdate}[1]{\ifthenelse{#1=1}{March 23, \getcourseyear{11}}{\ifthenelse{#1=2}{April 7, \getcourseyear{11}}{\ifthenelse{#1=3}{May 4, \getcourseyear{11}}{\ifthenelse{#1=4}{ } {}}}}}




\newcommand{\classdate}[1]{\getclassdate{#1}, \getcourseyear{#1}}
\newcommand{\dueclassdate}{\getclassdayofweek{\dueclassnumber} \getclassdateshort{\dueclassnumber}}


\newcommand\largeemptyspace{\vphantom{\textnormal{$\ds\int$}}}
\newcommand\nameblank{\if\issolution0\underline{\hskip11.25cm {\largeemptyspace}}
  \else\underline{\hskip.2cm{\LARGE\myname}\hskip6cm}\fi}
\newcommand\nameblankshort{\if\issolution0\underline{\hskip9.5cm {\largeemptyspace}}
  \else\underline{\hskip.2cm{\LARGE\myname {\largeemptyspace}}\hskip.2cm}\fi}
\newcommand\secblank{\if\issolution0\underline{\hskip1.5cm{\largeemptyspace}}
  \else\underline{\hskip.2cm{\LARGE\mysec {\largeemptyspace}}\hskip.2cm} \hskip.8cm \fi}
\newcommand\CMblank{\if\issolution0\underline{\hskip2.25cm{\largeemptyspace}}
  \else\underline{\hskip.2cm{\LARGE\myCM {\largeemptyspace}}\hskip.2cm}\fi}

\newcommand{\namegroupline}{Name: \nameblank Group \#: \underline{\hskip1.5cm{\largeemptyspace}}}
\newcommand{\nameline}{\begin{minipage}{0.6\linewidth} Name: \nameblank \end{minipage}}
\newcommand{\namelineshort}{Name: \nameblankshort}
\newcommand{\namesecline}{\begin{minipage}{0.7\linewidth} Name: \nameblank \end{minipage} \hfill \begin{minipage}{0.29\linewidth}Section \#: \secblank\end{minipage}}
\newcommand{\namesecCMline}{\begin{minipage}{0.6\linewidth}Name: \nameblankshort \end{minipage} \hfill \begin{minipage}{0.4\linewidth} Section \# \secblank CM\# \CMblank \end{minipage}}
\newcommand{\keyline}{{\color{red} SOLUTION KEY}}
%\newcommand{\nameline}{Name: \rule{11.5cm}{0.01cm} \hfill Section: \rule{1.5cm}{0.01cm}}
%\newcommand{\keyline}{Name: \rule{4cm}{0.01cm} SOLUTION KEY \rule{4cm}{0.01cm} \hfill Section: \rule{1.5cm}{0.01cm}}
%\newcommand{\groupline}{Group members present: \rule{8.5cm}{0.01cm} \hfill Group \#: \rule{1.5cm}{0.01cm}}
\newcommand{\course}{CSSE/MA 474\xspace}
\newcommand{\coursewithname}{CSSE/MA 474. Theory of Computation\xspace}


\newcommand{\wtitlestuff}{
\if\isanswerkey0
  \nameline
  \else
  \keyline
\fi
\begin{center}
\large \course Worksheet for Class \#\classnumber\\
\small \classdate{\classnumber}
\normalsize
\end{center}}

\newcommand{\lectitlestuff}{
\begin{center}
\Large \course Lecture \#\classnumber\\
\vskip 3pt \small Nate Chenette \\ \classdate{\classnumber}
\normalsize
\end{center}}

\newcommand{\othertitlestuff}{
\begin{center}
\Large \othertitle\\
\small \coursewithname\\
Class \#\classnumber, \classdate{\classnumber}\\
\normalsize
\end{center}}

\newcommand{\othertitlestuffnodate}{
\begin{center}
\Large \othertitle\\
\small \coursewithname\\
\normalsize
\end{center}}

\newcommand{\othernametitlestuff}{
\nameline\\
\othertitlestuff
}

\newcommand{\assignmenttitlestuff}{
\begin{center}
\Large \course Assignment \assignmentnum\\
\small Due date: \dueclassdate
\normalsize
\end{center}}

\newcommand{\assignmentnametitlestuff}{
\if\isanswerkey0
  \namesecCMline
  \else
  \keyline
\fi
\assignmenttitlestuff
}

\newcommand{\quiznametitlestuff}{
\if\isanswerkey0
  \namesecCMline
  \else
  \keyline
\fi
\begin{center}
\Large \course Quiz \quiznum\\
\small \classdate{\classnumber}
\normalsize
\end{center}}


\setlength{\parindent}{0in}
\setlength{\fboxsep}{.1in}

\renewcommand{\emptyset}{\varnothing}
\newcommand{\tvs}{\textvisiblespace}
\newcommand{\brk}{\vskip.2cm \hrule \vskip.2cm}
\newcommand{\ds}{\displaystyle}
\newcommand{\abs}[1]{\left\lvert {#1}\right\rvert}
\newcommand{\Lsym}{\text{L}}
\newcommand{\Rsym}{\text{R}}
\newcommand{\qacc}{q_{\textnormal{accept}}}
\newcommand{\qrej}{q_{\textnormal{reject}}}
\newcommand{\tmRej}{$\to$ \textbf{\textit{reject}}}
\newcommand{\tmAcc}{$\to$ \textbf{\textit{accept}}}
\def\lep{\le_\textnormal{P}}
\def\lem{\le_\textnormal{m}}
\def\ATM{A_\textnormal{TM}}
\newcommand{\vv}[2]{\begin{bmatrix} {#1} \cr {#2} \end{bmatrix}}
\newcommand{\vvt}[2]{\begin{bmatrix} {\tt #1} \cr {\tt #2} \end{bmatrix}}
\def\hs{\quad \texttt\#\quad }
\def\multiset#1#2{\ensuremath{\left(\kern-.3em\left(\genfrac{}{}{0pt}{}{#1}{#2}\right)\kern-.3em\right)}}
\def\time{\textsf{TIME}}
\def\ntime{\textsf{NTIME}}
\def\P{\textsf{P}}
\def\NP{\textsf{NP}}
   
%logic
\newcommand{\se}{\big|}
\newcommand{\lra}{\leftrightarrow}
\newcommand{\Lra}{\Leftrightarrow}
\newcommand{\we}{\wedge}
\def\thf{%
   \leavevmode
   \lower0.2ex\hbox{$\cdot$}%
   \kern-0.0em\raise0.7ex\hbox{$\cdot$}%
   \kern-0.0em\lower0.2ex\hbox{$\cdot$}%
   \thinspace}

%Number Systems
\newcommand{\bbZ}{\mathbb{Z}}
\newcommand{\bfZ}{\mathbf{Z}}
\newcommand{\bfZp}{\mathbf{Z}^+}
\newcommand{\bbN}{\mathbb{N}}
\newcommand{\bfN}{\mathbf{N}}
\newcommand{\bbQ}{\mathbb{Q}}
\newcommand{\bfQ}{\mathbf{Q}}
\newcommand{\bbR}{\mathbb{R}}
\newcommand{\bfR}{\mathbf{R}}
\newcommand{\bbC}{\mathbb{C}}
\newcommand{\bfC}{\mathbf{C}}

%sets
\newcommand{\U}{\mathscr{U}}
\newcommand{\ol}[1]{\overline{#1}}
\newcommand{\ssq}{\subseteq}
\newcommand{\sst}{\subset}
\def\ps{\mathcal{P}}
\def\sd{\,\triangle\,}
\def\sdonly{\triangle}
\def\es{\emptyset}

%cards
\def\hst{\heartsuit}
\def\cst{\clubsuit}
\def\sst{\spadesuit}
\def\dst{\diamondsuit}


\newcommand{\lcm}{{\rm lcm}}


\newcommand{\imgdir}{../images/}


\newcommand{\makeexamcover}{
\ifdefined\finalexam
\ \vskip2cm
\begin{center}
\huge \course Final Exam \\
\Large \finalexamdate \vskip1cm
\end{center}
\normalsize \instructions \vskip1cm
\begin{spacing}{1.5}
\begin{center}
\scorechart
\end{center}
\end{spacing}
\else \ifdefined\examnum
\ \vskip2cm
\begin{center}
\huge \course Exam \examnum \\
\Large \examdate{\examnum} \vskip1cm
\end{center}
\normalsize \instructions \vskip1cm
\begin{spacing}{1.5}
\begin{center}
\scorechart
\end{center}
\end{spacing}
\fi
\fi
}




\fancypagestyle{examcover}{% 
\fancyhf{}
\renewcommand{\footrulewidth}{0pt}
\lhead{\if\isanswerkey1{\keyline}\else{\nameline}\fi}
%\lhead{\if\isanswerkey1{\keyline}\else{\namesecline}\fi}
}



\fancypagestyle{exameverypage}{% 
\fancyhf{}
\renewcommand{\footrulewidth}{0pt}
\rhead{\if\isanswerkey1{\keyline}\else{}\fi}
\fancyfoot[R]{\thepage}
%\lhead{\if\isanswerkey1{\keyline}\else{\namesecline}\fi}
}

\newcommand{\definition}[1]{{\sc Definition}.~~{#1}\vskip.2cm}

\usepackage[framemethod=default]{mdframed}
\global\mdfdefinestyle{red1}{linecolor=red, linewidth=1pt, leftmargin=1cm, rightmargin=1cm}
\global\mdfdefinestyle{black1}{linecolor=black, linewidth=1pt,} %leftmargin=.1cm, rightmargin=.1cm}

\newcommand{\solution}[2][]{\if\issolution0 #1 \else \begin{mdframed}[style=black1] #2 \end{mdframed} \fi}

\newcommand{\cmblanka}[1]{\if\issolution0 	\underline{\hskip1cm{\largeemptyspace}}
\else 						  		\underline{\hskip.35cm {#1}\hskip.35cm{\largeemptyspace}}\fi}
\newcommand{\sblanka}[1]{\if\issolution0 		\underline{\hskip1.5cm{\largeemptyspace}}
\else 						  		\underline{\hskip.25cm {#1}\hskip.25cm{\largeemptyspace}}\fi}
\newcommand{\mblanka}[1]{\if\issolution0 	\underline{\hskip3cm{\largeemptyspace}}
\else 						  		\underline{\hskip.5cm {#1}\hskip.5cm{\largeemptyspace}}\fi}
\newcommand{\lblanka}[1]{\if\issolution0 		\underline{\hskip4.5cm{\largeemptyspace}}
\else 						  		\underline{\hskip.75cm {#1}\hskip.75cm{\largeemptyspace}}\fi}
\newcommand{\Lblanka}[1]{\if\issolution0		\underline{\hskip6cm{\largeemptyspace}}
\else 								\underline{\hskip1cm {#1}\hskip1cm{\largeemptyspace}}\fi}
\newcommand{\LLblanka}[1]{\if\issolution0	\underline{\hskip7.5cm{\largeemptyspace}}
\else 								\underline{\hskip1.25cm {#1}\hskip1.25cm{\largeemptyspace}}\fi}
\newcommand{\LLLblanka}[1]{\if\issolution0 	\underline{\hskip9cm{\largeemptyspace}}
\else 								\underline{\hskip1.5cm {#1}\hskip1.5cm{\largeemptyspace}}\fi}
\newcommand{\tinyspacea}[1]{\if\issolution0 	\hskip.2cm{\largeemptyspace}
\else 						  		{#1}{\largeemptyspace}\fi}
\newcommand{\cmspacea}[1]{\if\issolution0 	\hskip1cm{\largeemptyspace}
\else 						  		\hskip.15cm {#1}\hskip.15cm{\largeemptyspace}\fi}
\newcommand{\sspacea}[1]{\if\issolution0 		\hskip1.5cm{\largeemptyspace}
\else 						  		\hskip.25cm {#1}\hskip.25cm{\largeemptyspace}\fi}
\newcommand{\mspacea}[1]{\if\issolution0 	\hskip3cm{\largeemptyspace}
\else 						  		\hskip.25cm {#1}\hskip.25cm{\largeemptyspace}\fi}
\newcommand{\lspacea}[1]{\if\issolution0 		\hskip4.5cm{\largeemptyspace}
\else 						  		\hskip.25cm {#1}\hskip.25cm{\largeemptyspace}\fi}
\newcommand{\Lspacea}[1]{\if\issolution0 	\hskip6cm{\largeemptyspace}
\else 						  		\hskip.25cm {#1}\hskip.25cm{\largeemptyspace}\fi}


\newcommand{\sparagraph}[1]{\vskip-1cm\paragraph{#1}}

\if\isanswerkey1\input{macsse474-key}\fi


\usetikzlibrary{positioning}

\begin{document}

\assignmentnametitlestuff

%In this assignment, you may use any TM constructions from class, such as $D_{\textnormal{A-DFA}}$, $D_{\textnormal{A-NFA}}$,$D_\textnormal{E-DFA}$, $D_\textnormal{EQ-DFA}$, $D_\textnormal{EQ-REX}$. You may also use any known results from previous chapters of the textbook, e.g. that from any regular expression $R$, one can construct (and in particular, a Turing machine can construct) a DFA recognizing $L(R)$, likewise for CFGs and PDAs, etc.
%
%To show a language is decidable, it is sufficient to give a high-level description of a Turing machine that decides the language.



\begin{enumerate}

\item (4.7) Let {\cal B} be the set of all infinite sequences over $\{0,1\}$. Show that ${\cal B}$ is uncountable using a proof by diagonalization.
\solution{
\if\isanswerkey1\solDiagonalizationZeroOneSequences\fi
\textsc{Proof}: Suppose to the contrary that there is a \textit{\textbf{one-to-one}} correspondence $f: {\cal B} \to \mathbbm{N}$. So that $\forall ~x \in {\cal B} \mid f(x)\in \mathbbm{N} ~\textnormal{and } f^{-1}(f(x))=x$. Consider for a correspondence $f$ such that $f(\texttt{1})=\texttt{000000}$..., $f(\texttt{2})=\texttt{111111}$..., $f(\texttt{3})=\texttt{010101}$..., $f(\texttt{4})=\texttt{101010}$..., $f(5)=$ ...,  which is shown in the table below:
\begin{center}
    \begin{tabular}{c|c}
        $n$ & $f(n)$ \\
        \hline
        \texttt{1} & \texttt{000000}... \\
        \texttt{2} & \texttt{111111}... \\
        \texttt{3} & \texttt{010101}... \\
        \texttt{4} & \texttt{101010}... \\
        $\vdots$ & $\vdots$
    \end{tabular}\\
\end{center}
So that we can construct the $x$ such that $\forall n \in \mathbbm{N} \mid f(n) \neq x$. So that to ensure that $x\neq f(\texttt{1})$, we take the flip digit of the first digit of $f(\texttt{1})=\underline{\texttt{0}}\texttt{00000}$..., which is \texttt{1}, to ensure that $x\neq f(\texttt{2})$, second digit is the flip of second digit of $f(\texttt{2})=\texttt{1}\underline{\texttt{1}}\texttt{1111}$..., which is \texttt{0}, to ensure that $x\neq f(\texttt{3})$, the third digit is the flip of third digit of $f(\texttt{3})=\texttt{01}\underline{\texttt{0}}\texttt{101}$..., which is \texttt{1}, to ensure that $x\neq f(\texttt{4})$,  the fourth digit is the flip of fourth digit of $f(\texttt{4})=\texttt{101}\underline{\texttt{0}}\texttt{10}$..., which is \texttt{1}, etc. So that $x=\texttt{1011}$..., which $\forall ~n \mid f(n)\neq x$. Hence $\exists ~x \in {\cal B} \mid \left\{\forall ~n \mid f(n)\neq x\right\}$, which contradicts the assumption, so that ${\cal B}$ is uncountable. 
}


\item In class, we showed that $\mathbbm{R}$ was uncountable by giving a diagonalization argument where we were able to build a number, $x$, that was not in the alleged 1-1 correspondence between $\mathbbm{N}$ and $\mathbbm{R}$ by making the $i$th digit of $x$ (after the decimal point) different than the $i$th digit of the $i$th element of the list.  Why doesn't this same argument work to show that the set of rational numbers $\mathbbm{Q}$ is uncountable?  
\solution{
\if\isanswerkey1\solWhyQNotUncountable\fi
Since if we construct the $x$ in the same procedure as that in \textsc{Proof} of $\mathbbm{R}$ is not countable, we have to construct a $\mathbbm{R}$ such that $i^{\textnormal{th}}$ digit of $x$ is different from that in $f(n)$. By doing so we might end up with a $x$ that has infinite digits with no repeating pattern, which makes it irrational. Hence $x \notin \mathbbm{Q}$. So that this proof does not work for rational set $\mathbbm{Q}$.
}

\item (4.30) Let $A$ be a Turing-recognizable language consisting of descriptions of Turing machines, $A = \{\langle M_1 \rangle, \langle M_2 \rangle, \ldots \}$, where every $M_i$ is a decider. Prove that some decidable language $T$ is not decided by any decider $M_i$ whose description appears in $A$. (Hints: You may find it helpful to consider an enumerator for $A$ that outputs TM descriptions in a specific order $\langle M_1 \rangle, \langle M_2 \rangle, \ldots$, and also consider all strings over the alphabet in a specific order: $\Sigma^* = \{s_1, s_2, \ldots\}$. Proof suggestion: build a decider $D$ that explicitly constructs $T = L(D)$ so that it is different from every $L(M_i)$.)
\solution{
\if\isanswerkey1\solDecidableLangNotCoveredByDescriptionsOfDeciders\fi
Since language $A$ is Turing-recognizable, so that we can build a $E_\textnormal{A}$ that enumerates language $A$, where its output is in order $\langle M_1\rangle, \langle M_2\rangle, ...$, and consider all strings over the alphabet in a specific order $\Sigma^*=s_1, s_2, s_3$, ... First, we can build decider $D_\textnormal{E-A}$ that decides whether a description of TM accept a string as following:\\
$D_\textnormal{E-A}=$ ``On input $\langle w, \langle M\rangle\rangle$, where $w$ is string and $\langle M\rangle$ is a description of a Turing-machine,
\begin{enumerate}
    \item Run $w$ on TM $M$
    \begin{enumerate}
        \item If it accept \tmAcc
        \item Otherwise \tmRej''
    \end{enumerate}
\end{enumerate}
And this problem can be converted to language $B=\left\{w \mid \left(\forall ~\langle M_i\rangle \in A \mid M_i \textnormal{ rejects } w\right)\right\}$, we can construct a decider $D_\textnormal{B}$ that decides language $B$ as:\\
$D_\textnormal{B}=$ ``On input $w$,
\begin{enumerate}
    \item If $w\notin\Sigma^*$ \tmRej
    \item Repeat step i$-$ii over $i=1, 2, 3, ..., n$
    \begin{enumerate}
        \item Run $D_\textnormal{E-A}$ on input $\langle w, \langle M_i\rangle\rangle$
        \item If it accepts \tmRej
    \item Otherwise \tmAcc''
    \end{enumerate}
\end{enumerate}
}


\item (5.10) Consider the problem of determining whether a two-tape Turing machine ever writes a non-blank symbol on its second tape when it is run on input $w$. Formulate this problem as a language and show that it is undecidable.
\solution{
\if\isanswerkey1\solUndecidableTMWritesNonblankSecondTape\fi
Define the problem as a language 
\begin{equation*}
    WNB_\textnormal{TM}=\left\{\langle M, w\rangle \mid M\textnormal{ write non-blank symbol when running input} ~w\right\}
\end{equation*}
\textsc{Proof}: Suppose to the contrary that there is a $D_\textnormal{WNB-TM}$ that decides language $WNB_\textnormal{TM}$
Then we can construct a TM $M'$ as following:\\
$M'=$ ``On input $\langle M, w\rangle$, 
\begin{enumerate}
    \item Run $M$ on $w$ using only its first tape
    \item If it accept, write a symbol $\sigma \in \Sigma$ on its second tape \tmAcc
    \item Otherwise, halt and \tmRej''
\end{enumerate}
Then we can construct a TM $M''$ such that it decides language $A_\textnormal{TM}$ as following:\\
$M''=$ ``On input $\langle M, w\rangle$,
\begin{enumerate}
    \item Run $D_\textnormal{WNB-TM}$ on input $\langle M', \langle M, w\rangle\rangle$
    \item If it accept \tmAcc
    \item Otherwise \tmRej''
\end{enumerate}
And it makes $A_\textnormal{TM}$ decidable. So that $WNB_\textnormal{TM}$ is decidable $\Longleftrightarrow$ $A_\textnormal{TM}$ is decidable. Since $A_\textnormal{TM}$ is not decidable, language $WNB_\textnormal{TM}$ is not decidable
}


\end{enumerate}




\end{document}
