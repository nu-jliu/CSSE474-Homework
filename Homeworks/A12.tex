\documentclass[11pt]{article}

\usepackage{enumitem}


%%TO EDIT
\newcommand{\dueclassnumber}{29}
\newcommand{\assignmentnum}{12}

% CHANGE issolution{0} to issolution{1} for homework submission.
% WRITE solutions inside the \solution{} commands.
% or, you can use \if\issolution1 … \fi

\def\issolution{1}
\def\myname{Allen Liu} % My name goes here
\def\mysec{01} % Section number goes here
\def\myCM{374} % Campus Mailbox goes here

\def\isanswerkey{0}


% Leave the next line alone. This is for my answer keys.
%\def\isanswerkey{1}

\usepackage{bbm,fancyhdr,ifthen,setspace,hyperref,url}
\usepackage{amssymb,amsmath,enumitem,amsthm,mathrsfs}
\usepackage{graphicx,xspace,color}
\usepackage{hhline}
\usepackage{tikz}
\usetikzlibrary{automata, positioning, arrows,chains,scopes,fit}
\tikzset{
->, % makes the edges directed
>=stealth', % makes the arrow heads bold
node distance=2.4cm, % specifies the minimum distance between two nodes. Change if necessary.
every state/.style={thick, fill=gray!10}, % sets the properties for each ’state’ node
initial text=$ $, % sets the text that appears on the start arrow
}

\topmargin=-.5in
\headsep=0.0in
\oddsidemargin=-.35in
\evensidemargin=-.65in
\textwidth=7.25in
\textheight=9.75in
\footskip=0in
\usepackage{titlesec}
\titlespacing*{\paragraph}{0pt}{2ex plus 1ex minus .2ex}{1ex}
\fancyhf{} % clear all header and footers
\renewcommand{\headrulewidth}{0pt} % remove the header rule
%\rfoot{\thepage}
\pagestyle{fancy}

\ifx\myname\undefined
\def\myname{}
\fi

\ifx\isanswerkey\undefined
\def\isanswerkey{0}
\fi

\if\isanswerkey1
\def\issolution{1}
\fi


\newcommand{\getcourseyear}[1]{2022}
\newcommand{\getcourseterm}{Spring \getcourseyear{1}}

\newcommand{\getclassmonthnum}[1]{\ifthenelse{#1<16}{3}{\ifthenelse{#1<29}{4}{5}}}
\newcommand{\getclassmonthshort}[1]{\ifthenelse{#1<16}{Mar}{\ifthenelse{#1<29}{Apr}{May}}}
\newcommand{\getclassmonth}[1]{\ifthenelse{#1<16}{March}{\ifthenelse{#1<29}{April}{May}}}
\newcommand{\getclassdayofmonth}[1]{\ifthenelse{
#1=1}{7}{\ifthenelse{
#1=2}{8}{\ifthenelse{
#1=3}{10}{\ifthenelse{
#1=4}{11}{\ifthenelse{
#1=5}{14}{\ifthenelse{
#1=6}{15}{\ifthenelse{
#1=7}{17}{\ifthenelse{
#1=8}{18}{\ifthenelse{
#1=9}{21}{\ifthenelse{
#1=10}{22}{\ifthenelse{
#1=11}{24}{\ifthenelse{
#1=12}{25}{\ifthenelse{
#1=13}{28}{\ifthenelse{
#1=14}{29}{\ifthenelse{
#1=15}{31}{\ifthenelse{
#1=16}{1}{\ifthenelse{
#1=17}{4}{\ifthenelse{
#1=18}{5}{\ifthenelse{
#1=19}{7}{\ifthenelse{
#1=20}{8}{\ifthenelse{
#1=21}{18}{\ifthenelse{
#1=22}{19}{\ifthenelse{
#1=23}{21}{\ifthenelse{
#1=24}{22}{\ifthenelse{
#1=25}{25}{\ifthenelse{
#1=26}{26}{\ifthenelse{
#1=27}{28}{\ifthenelse{
#1=28}{29}{\ifthenelse{
#1=29}{2}{\ifthenelse{
#1=30}{3}{\ifthenelse{
#1=31}{5}{\ifthenelse{
#1=32}{6}{\ifthenelse{
#1=33}{9}{\ifthenelse{
#1=34}{10}{\ifthenelse{
#1=35}{12}{\ifthenelse{
#1=36}{13}{\ifthenelse{
#1=37}{16}{\ifthenelse{
#1=38}{17}{\ifthenelse{
#1=39}{19}{\ifthenelse{
#1=40}{20}{}}}}}}}}}}}}}}}}}}}}}}}}}}}}}}}}}}}}}}}}}

\newcommand{\getclassdate}[1]{\getclassmonth{#1}\xspace\getclassdayofmonth{#1}}
\newcommand{\getclassdateshort}[1]{\getclassmonthshort{#1}\xspace\getclassdayofmonth{#1}}
\newcommand{\getclassdatenum}[1]{\getclassmonthnum{#1}/\getclassdayofmonth{#1}}
\newcommand{\getMday}{Mon}
\newcommand{\getTday}{Tue}
\newcommand{\getWday}{Wed}
\newcommand{\getRday}{Thu}
\newcommand{\getFday}{Fri}
\newcommand{\getclassdayofweek}[1]{\ifthenelse{
#1=1}{\getMday}{\ifthenelse{#1=2}{\getTday}{\ifthenelse{#1=3}{\getRday}{\ifthenelse{#1=4}{\getFday}{\ifthenelse{
#1=5}{\getMday}{\ifthenelse{#1=6}{\getTday}{\ifthenelse{#1=7}{\getRday}{\ifthenelse{#1=8}{\getFday}{\ifthenelse{
#1=9}{\getMday}{\ifthenelse{#1=10}{\getTday}{\ifthenelse{#1=11}{\getRday}{\ifthenelse{#1=12}{\getFday}{\ifthenelse{
#1=13}{\getMday}{\ifthenelse{#1=14}{\getTday}{\ifthenelse{#1=15}{\getRday}{\ifthenelse{#1=16}{\getFday}{\ifthenelse{
#1=17}{\getMday}{\ifthenelse{#1=18}{\getTday}{\ifthenelse{#1=19}{\getRday}{\ifthenelse{#1=20}{\getFday}{\ifthenelse{
#1=21}{\getMday}{\ifthenelse{#1=22}{\getTday}{\ifthenelse{#1=23}{\getRday}{\ifthenelse{#1=24}{\getFday}{\ifthenelse{
#1=25}{\getMday}{\ifthenelse{#1=26}{\getTday}{\ifthenelse{#1=27}{\getRday}{\ifthenelse{#1=28}{\getFday}{\ifthenelse{
#1=29}{\getMday}{\ifthenelse{#1=30}{\getTday}{\ifthenelse{#1=31}{\getRday}{\ifthenelse{#1=32}{\getFday}{\ifthenelse{
#1=33}{\getMday}{\ifthenelse{#1=34}{\getTday}{\ifthenelse{#1=35}{\getRday}{\ifthenelse{#1=36}{\getFday}{\ifthenelse{
#1=37}{\getMday}{\ifthenelse{#1=38}{\getTday}{\ifthenelse{#1=39}{\getRday}{\ifthenelse{#1=40}{\getFday}\xspace
}}}}}}}}}}}}}}}}}}}}}}}}}}}}}}}}}}}}}}}}
\newcommand{\examdate}[1]{\ifthenelse{#1=1}{March 23, \getcourseyear{11}}{\ifthenelse{#1=2}{April 7, \getcourseyear{11}}{\ifthenelse{#1=3}{May 4, \getcourseyear{11}}{\ifthenelse{#1=4}{ } {}}}}}




\newcommand{\classdate}[1]{\getclassdate{#1}, \getcourseyear{#1}}
\newcommand{\dueclassdate}{\getclassdayofweek{\dueclassnumber} \getclassdateshort{\dueclassnumber}}


\newcommand\largeemptyspace{\vphantom{\textnormal{$\ds\int$}}}
\newcommand\nameblank{\if\issolution0\underline{\hskip11.25cm {\largeemptyspace}}
  \else\underline{\hskip.2cm{\LARGE\myname}\hskip6cm}\fi}
\newcommand\nameblankshort{\if\issolution0\underline{\hskip9.5cm {\largeemptyspace}}
  \else\underline{\hskip.2cm{\LARGE\myname {\largeemptyspace}}\hskip.2cm}\fi}
\newcommand\secblank{\if\issolution0\underline{\hskip1.5cm{\largeemptyspace}}
  \else\underline{\hskip.2cm{\LARGE\mysec {\largeemptyspace}}\hskip.2cm} \hskip.8cm \fi}
\newcommand\CMblank{\if\issolution0\underline{\hskip2.25cm{\largeemptyspace}}
  \else\underline{\hskip.2cm{\LARGE\myCM {\largeemptyspace}}\hskip.2cm}\fi}

\newcommand{\namegroupline}{Name: \nameblank Group \#: \underline{\hskip1.5cm{\largeemptyspace}}}
\newcommand{\nameline}{\begin{minipage}{0.6\linewidth} Name: \nameblank \end{minipage}}
\newcommand{\namelineshort}{Name: \nameblankshort}
\newcommand{\namesecline}{\begin{minipage}{0.7\linewidth} Name: \nameblank \end{minipage} \hfill \begin{minipage}{0.29\linewidth}Section \#: \secblank\end{minipage}}
\newcommand{\namesecCMline}{\begin{minipage}{0.6\linewidth}Name: \nameblankshort \end{minipage} \hfill \begin{minipage}{0.4\linewidth} Section \# \secblank CM\# \CMblank \end{minipage}}
\newcommand{\keyline}{{\color{red} SOLUTION KEY}}
%\newcommand{\nameline}{Name: \rule{11.5cm}{0.01cm} \hfill Section: \rule{1.5cm}{0.01cm}}
%\newcommand{\keyline}{Name: \rule{4cm}{0.01cm} SOLUTION KEY \rule{4cm}{0.01cm} \hfill Section: \rule{1.5cm}{0.01cm}}
%\newcommand{\groupline}{Group members present: \rule{8.5cm}{0.01cm} \hfill Group \#: \rule{1.5cm}{0.01cm}}
\newcommand{\course}{CSSE/MA 474\xspace}
\newcommand{\coursewithname}{CSSE/MA 474. Theory of Computation\xspace}


\newcommand{\wtitlestuff}{
\if\isanswerkey0
  \nameline
  \else
  \keyline
\fi
\begin{center}
\large \course Worksheet for Class \#\classnumber\\
\small \classdate{\classnumber}
\normalsize
\end{center}}

\newcommand{\lectitlestuff}{
\begin{center}
\Large \course Lecture \#\classnumber\\
\vskip 3pt \small Nate Chenette \\ \classdate{\classnumber}
\normalsize
\end{center}}

\newcommand{\othertitlestuff}{
\begin{center}
\Large \othertitle\\
\small \coursewithname\\
Class \#\classnumber, \classdate{\classnumber}\\
\normalsize
\end{center}}

\newcommand{\othertitlestuffnodate}{
\begin{center}
\Large \othertitle\\
\small \coursewithname\\
\normalsize
\end{center}}

\newcommand{\othernametitlestuff}{
\nameline\\
\othertitlestuff
}

\newcommand{\assignmenttitlestuff}{
\begin{center}
\Large \course Assignment \assignmentnum\\
\small Due date: \dueclassdate
\normalsize
\end{center}}

\newcommand{\assignmentnametitlestuff}{
\if\isanswerkey0
  \namesecCMline
  \else
  \keyline
\fi
\assignmenttitlestuff
}

\newcommand{\quiznametitlestuff}{
\if\isanswerkey0
  \namesecCMline
  \else
  \keyline
\fi
\begin{center}
\Large \course Quiz \quiznum\\
\small \classdate{\classnumber}
\normalsize
\end{center}}


\setlength{\parindent}{0in}
\setlength{\fboxsep}{.1in}

\renewcommand{\emptyset}{\varnothing}
\newcommand{\tvs}{\textvisiblespace}
\newcommand{\brk}{\vskip.2cm \hrule \vskip.2cm}
\newcommand{\ds}{\displaystyle}
\newcommand{\abs}[1]{\left\lvert {#1}\right\rvert}
\newcommand{\Lsym}{\text{L}}
\newcommand{\Rsym}{\text{R}}
\newcommand{\qacc}{q_{\textnormal{accept}}}
\newcommand{\qrej}{q_{\textnormal{reject}}}
\newcommand{\tmRej}{$\to$ \textbf{\textit{reject}}}
\newcommand{\tmAcc}{$\to$ \textbf{\textit{accept}}}
\def\lep{\le_\textnormal{P}}
\def\lem{\le_\textnormal{m}}
\def\ATM{A_\textnormal{TM}}
\newcommand{\vv}[2]{\begin{bmatrix} {#1} \cr {#2} \end{bmatrix}}
\newcommand{\vvt}[2]{\begin{bmatrix} {\tt #1} \cr {\tt #2} \end{bmatrix}}
\def\hs{\quad \texttt\#\quad }
\def\multiset#1#2{\ensuremath{\left(\kern-.3em\left(\genfrac{}{}{0pt}{}{#1}{#2}\right)\kern-.3em\right)}}
\def\time{\textsf{TIME}}
\def\ntime{\textsf{NTIME}}
\def\P{\textsf{P}}
\def\NP{\textsf{NP}}
   
%logic
\newcommand{\se}{\big|}
\newcommand{\lra}{\leftrightarrow}
\newcommand{\Lra}{\Leftrightarrow}
\newcommand{\we}{\wedge}
\def\thf{%
   \leavevmode
   \lower0.2ex\hbox{$\cdot$}%
   \kern-0.0em\raise0.7ex\hbox{$\cdot$}%
   \kern-0.0em\lower0.2ex\hbox{$\cdot$}%
   \thinspace}

%Number Systems
\newcommand{\bbZ}{\mathbb{Z}}
\newcommand{\bfZ}{\mathbf{Z}}
\newcommand{\bfZp}{\mathbf{Z}^+}
\newcommand{\bbN}{\mathbb{N}}
\newcommand{\bfN}{\mathbf{N}}
\newcommand{\bbQ}{\mathbb{Q}}
\newcommand{\bfQ}{\mathbf{Q}}
\newcommand{\bbR}{\mathbb{R}}
\newcommand{\bfR}{\mathbf{R}}
\newcommand{\bbC}{\mathbb{C}}
\newcommand{\bfC}{\mathbf{C}}

%sets
\newcommand{\U}{\mathscr{U}}
\newcommand{\ol}[1]{\overline{#1}}
\newcommand{\ssq}{\subseteq}
\newcommand{\sst}{\subset}
\def\ps{\mathcal{P}}
\def\sd{\,\triangle\,}
\def\sdonly{\triangle}
\def\es{\emptyset}

%cards
\def\hst{\heartsuit}
\def\cst{\clubsuit}
\def\sst{\spadesuit}
\def\dst{\diamondsuit}


\newcommand{\lcm}{{\rm lcm}}


\newcommand{\imgdir}{../images/}


\newcommand{\makeexamcover}{
\ifdefined\finalexam
\ \vskip2cm
\begin{center}
\huge \course Final Exam \\
\Large \finalexamdate \vskip1cm
\end{center}
\normalsize \instructions \vskip1cm
\begin{spacing}{1.5}
\begin{center}
\scorechart
\end{center}
\end{spacing}
\else \ifdefined\examnum
\ \vskip2cm
\begin{center}
\huge \course Exam \examnum \\
\Large \examdate{\examnum} \vskip1cm
\end{center}
\normalsize \instructions \vskip1cm
\begin{spacing}{1.5}
\begin{center}
\scorechart
\end{center}
\end{spacing}
\fi
\fi
}




\fancypagestyle{examcover}{% 
\fancyhf{}
\renewcommand{\footrulewidth}{0pt}
\lhead{\if\isanswerkey1{\keyline}\else{\nameline}\fi}
%\lhead{\if\isanswerkey1{\keyline}\else{\namesecline}\fi}
}



\fancypagestyle{exameverypage}{% 
\fancyhf{}
\renewcommand{\footrulewidth}{0pt}
\rhead{\if\isanswerkey1{\keyline}\else{}\fi}
\fancyfoot[R]{\thepage}
%\lhead{\if\isanswerkey1{\keyline}\else{\namesecline}\fi}
}

\newcommand{\definition}[1]{{\sc Definition}.~~{#1}\vskip.2cm}

\usepackage[framemethod=default]{mdframed}
\global\mdfdefinestyle{red1}{linecolor=red, linewidth=1pt, leftmargin=1cm, rightmargin=1cm}
\global\mdfdefinestyle{black1}{linecolor=black, linewidth=1pt,} %leftmargin=.1cm, rightmargin=.1cm}

\newcommand{\solution}[2][]{\if\issolution0 #1 \else \begin{mdframed}[style=black1] #2 \end{mdframed} \fi}

\newcommand{\cmblanka}[1]{\if\issolution0 	\underline{\hskip1cm{\largeemptyspace}}
\else 						  		\underline{\hskip.35cm {#1}\hskip.35cm{\largeemptyspace}}\fi}
\newcommand{\sblanka}[1]{\if\issolution0 		\underline{\hskip1.5cm{\largeemptyspace}}
\else 						  		\underline{\hskip.25cm {#1}\hskip.25cm{\largeemptyspace}}\fi}
\newcommand{\mblanka}[1]{\if\issolution0 	\underline{\hskip3cm{\largeemptyspace}}
\else 						  		\underline{\hskip.5cm {#1}\hskip.5cm{\largeemptyspace}}\fi}
\newcommand{\lblanka}[1]{\if\issolution0 		\underline{\hskip4.5cm{\largeemptyspace}}
\else 						  		\underline{\hskip.75cm {#1}\hskip.75cm{\largeemptyspace}}\fi}
\newcommand{\Lblanka}[1]{\if\issolution0		\underline{\hskip6cm{\largeemptyspace}}
\else 								\underline{\hskip1cm {#1}\hskip1cm{\largeemptyspace}}\fi}
\newcommand{\LLblanka}[1]{\if\issolution0	\underline{\hskip7.5cm{\largeemptyspace}}
\else 								\underline{\hskip1.25cm {#1}\hskip1.25cm{\largeemptyspace}}\fi}
\newcommand{\LLLblanka}[1]{\if\issolution0 	\underline{\hskip9cm{\largeemptyspace}}
\else 								\underline{\hskip1.5cm {#1}\hskip1.5cm{\largeemptyspace}}\fi}
\newcommand{\tinyspacea}[1]{\if\issolution0 	\hskip.2cm{\largeemptyspace}
\else 						  		{#1}{\largeemptyspace}\fi}
\newcommand{\cmspacea}[1]{\if\issolution0 	\hskip1cm{\largeemptyspace}
\else 						  		\hskip.15cm {#1}\hskip.15cm{\largeemptyspace}\fi}
\newcommand{\sspacea}[1]{\if\issolution0 		\hskip1.5cm{\largeemptyspace}
\else 						  		\hskip.25cm {#1}\hskip.25cm{\largeemptyspace}\fi}
\newcommand{\mspacea}[1]{\if\issolution0 	\hskip3cm{\largeemptyspace}
\else 						  		\hskip.25cm {#1}\hskip.25cm{\largeemptyspace}\fi}
\newcommand{\lspacea}[1]{\if\issolution0 		\hskip4.5cm{\largeemptyspace}
\else 						  		\hskip.25cm {#1}\hskip.25cm{\largeemptyspace}\fi}
\newcommand{\Lspacea}[1]{\if\issolution0 	\hskip6cm{\largeemptyspace}
\else 						  		\hskip.25cm {#1}\hskip.25cm{\largeemptyspace}\fi}


\newcommand{\sparagraph}[1]{\vskip-1cm\paragraph{#1}}

\if\isanswerkey1\input{macsse474-key}\fi

\newcommand\dom[2]{\left[ \begin{array}{c} {#1} \cr\hline {#2} \end{array}\right]}
\newcommand\domt[2]{\left[ \begin{array}{c} {\tt #1} \cr\hline {\tt #2} \end{array}\right]}

\usetikzlibrary{positioning}

\begin{document}

\assignmentnametitlestuff


\begin{enumerate}


\item (5.28) {\bf Rice's theorem.} Let $P$ be any nontrivial property of the language of a Turing machine. Prove that the problem of determining whether a given Turing machine's language has property $P$ is undecidable.

In more formal terms, let $P$ be a language consisting of Turing machine descriptions where $P$ fulfills two conditions. First, $P$ is nontrivial---it contains some, but not all, TM descriptions. Second, $P$ is a property of the TM's language---whenever $L(M_1) = L(M_2)$, we have $\langle M_1 \rangle \in P$ iff $\langle M_2 \rangle \in P$. Here, $M_1$ and $M_2$ are any TMs. Prove that $P$ is an undecidable language.

Hints:
\begin{itemize}
\item Assume without loss of generality that $\langle T_\emptyset \rangle \notin P$ where $T_\emptyset$ is a TM which rejects all input: if not, then we could switch the task to showing $\overline{P}$ is undecidable, and this condition would apply. (Note that if either $P$ or $\overline{P}$ is decidable, so is the other.)
\item Since $P$ is nonempty, there exists some TM $T$ such that $\langle T \rangle \in P$. (In the case of proving $\overline{P}$ is decidable, the analogous statement can be made as we know $P$ doesn't contain all TM descriptions.)
\item Try reducing from $A_\textnormal{TM}$ to $P$. For an input $\langle M, w \rangle$ to $A_\textnormal{TM}$, consider the cases $\begin{cases} M \textnormal { accepts } w \cr M \textnormal { rejects } w, \end{cases}$ you want to construct a machine $M_w$ such that the two cases $\begin{cases} \langle M_w \rangle \in P \cr \langle M_w \rangle \notin P \end{cases}$ correspond to the two cases regarding $M$, so that you can run your purported $P$-decider to decide $A_\textnormal{TM}$. 
\end{itemize}
\solution{
\if\isanswerkey1\solRicesTheorem\fi
\textsc{Proof}: Assume that $\langle T_\emptyset \notin P$, suppose to contrary that there is a decider $D_\textnormal{P}$ that decides the language $P$, So that $D_\textnormal{P}$ can be described as following:\\
$D_\textnormal{P}=$ ``On input $\langle M\rangle$, 
\begin{enumerate}
    \item If $\langle M\rangle \in P$ \tmAcc
    \item Otherwise \tmRej''
\end{enumerate}
Since language $P \neq \emptyset$, so that $\exists ~\langle M'\rangle \mid \langle M'\rangle \in P$. And we can construct a decider $R$ that decides language $A_\textnormal{TM}$. The decider $R$ is defined as following:\\
$R=$ ``On input $\langle M, w\rangle$, 
\begin{enumerate}
    \item Construct a Turing-machine $M_w$ as following:\\
    $M_w=$ `On input $x$, $x$ is a string,
    \begin{enumerate}
        \item Run $w$ on $M$
        \begin{enumerate}
            \item If it rejects \tmRej
            \item Otherwise, run $M'$ on input $x$
            \begin{enumerate}
                \item If it accepts \tmAcc
                \item Otherwise \tmRej
            \end{enumerate}
        \end{enumerate}
    \end{enumerate}
    \item Run $\langle M_w\rangle$ in decider $D_\textnormal{P}$
    \begin{enumerate}
        \item If it accept \tmAcc
        \item Otherwise \tmRej''
    \end{enumerate}
\end{enumerate}
Since for the machine $M_w$, $M_w$ accepts $x$ iff $M$ accepts $w$ and $M'$ accepts $x$. If $M$ accepts $w$, so that $M_w$ accepts $x$ $\Longleftrightarrow$ $M'$ accepts $x$, which is equivalent with $L(M_w)=L(M')$. And since we know that $\langle M'\rangle \in P$, so as $\langle M_w\rangle$, so that $D_\textnormal{P}$ accepts $\langle M, w \rangle$. When $M$ rejects $w$, $L(M_w)=\emptyset$, so $\langle M_w\rangle \notin P$. So that $R$ rejects $\langle M, w\rangle$, which makes the decider $D_\textnormal{P}$ decides language $A_\textnormal{TM}$. Since we have proved that $A-\textnormal{TM}$ is not decidable, so that $D_\textnormal{P}$ does not exist, hence languague $P$ is not decidable.
}

\item (5.7) Show that if $A$ is Turing-recognizable and $A \lem \overline{A}$, then $A$ is decidable.
\solution{
\if\isanswerkey1\solTRecognizableMappingReductionFromSelfToComplementThenDecidable\fi
For language $A$, since we know that language $A$ is Turing-recognizable and $A \lem \overline{A}$, so that we know $\exists ~f \mid \left(\forall ~w \mid w \in A \Longleftrightarrow f(w) \in \overline{A}\right)$. So that we can conclude that $\overline{A} \lem A$. So that language $A$ is also Turing-recognizable. Then, language $A$ is both Turing-recognizable and Turing-corecognizable. Hence language $A$ is decidable.
}

\item (5.22) Show that $A$ is Turing-recognizable if and only if $A \lem A_\textnormal{TM}$.
\solution{
\if\isanswerkey1\solTRecognizableIffReducibleToATM\fi
By \textbf{Theorem 5.28}, since language $A_\textnormal{TM}$ is recognizable, and $A \lem A_\textnormal{TM}$. So that language $A$ is Turing-recognizable. And for language $A$ that is recognizable, so that $\exists ~M \mid A=L(M)$. So that we can define a function $f$ such that $\forall ~w \mid f(w)=\langle M, w\rangle$, which makes $f(w) \in A_\textnormal{TM}$. So that $A \lem A_\textnormal{TM}$
}


\item Think about both of these problems, but you only need to write out a complete solution of one of them---your choice, either (a) or (b). 
\begin{enumerate}
\item (5.17) Show that the Post Correspondence Problem is decidable over the unary alphabet $\Sigma = \{{\tt 1}\}$.
\solution{
\if\isanswerkey1\solUnaryPCPDecidable\fi
This problem can be converted to language that $A_\textnormal{PCP}=\{D \mid D \textnormal{ has a solution in PCP}\}$. Since $\Sigma=\texttt{\{1\}}$, then the only variable that matters is the number of characters in the top and the number of characters in the bottom. So that we can build a decider $D_\textnormal{A-PCP}$ that decides the language $A_\textnormal{PCP}$. The decider $D_\textnormal{A-PCP}$ is defined as following: \\
$D_\textnormal{A-PCP}=$ ``On input $D$, where $D$ is the set of domino's
\begin{enumerate}
    \item If there is a $d \in D$ such that $d$ has same number on top and bottom \tmAcc
    \item If $\forall ~d \in D \mid d \textnormal{ has more in the top}$ or $\forall ~d \in D \mid d \textnormal{ has more in the bottom}$ \tmRej
    \item Otherwise \tmAcc''
\end{enumerate}
So that language $A_\textnormal{PCP}$ is decidable.
}


\item (5.18) Show that the Post Correspondence Problem is undecidable over the binary alphabet $\Sigma = \{{\tt 0}, {\tt 1}\}$.
% \solution{
% \if\isanswerkey1\solBinaryPCPUndecidable\fi
% }

\end{enumerate}


\end{enumerate}




\end{document}
