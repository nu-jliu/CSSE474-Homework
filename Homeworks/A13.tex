\documentclass[11pt]{article}

\usepackage{enumitem}


%%TO EDIT
\newcommand{\dueclassnumber}{33}
\newcommand{\assignmentnum}{13}

% CHANGE issolution{0} to issolution{1} for homework submission.
% WRITE solutions inside the \solution{} commands.
% or, you can use \if\issolution1 … \fi

\def\issolution{1}
\def\myname{Allen Liu} % My name goes here
\def\mysec{01} % Section number goes here
\def\myCM{374} % Campus Mailbox goes here

\input{../common/flags}

% Leave the next line alone. This is for my answer keys.
%\def\isanswerkey{1}

\usepackage{bbm,fancyhdr,ifthen,setspace,hyperref,url}
\usepackage{amssymb,amsmath,enumitem,amsthm,mathrsfs}
\usepackage{graphicx,xspace,color}
\usepackage{hhline}
\usepackage{tikz}
\usetikzlibrary{automata, positioning, arrows,chains,scopes,fit}
\tikzset{
->, % makes the edges directed
>=stealth', % makes the arrow heads bold
node distance=2.4cm, % specifies the minimum distance between two nodes. Change if necessary.
every state/.style={thick, fill=gray!10}, % sets the properties for each ’state’ node
initial text=$ $, % sets the text that appears on the start arrow
}

\topmargin=-.5in
\headsep=0.0in
\oddsidemargin=-.35in
\evensidemargin=-.65in
\textwidth=7.25in
\textheight=9.75in
\footskip=0in
\usepackage{titlesec}
\titlespacing*{\paragraph}{0pt}{2ex plus 1ex minus .2ex}{1ex}
\fancyhf{} % clear all header and footers
\renewcommand{\headrulewidth}{0pt} % remove the header rule
%\rfoot{\thepage}
\pagestyle{fancy}

\ifx\myname\undefined
\def\myname{}
\fi

\ifx\isanswerkey\undefined
\def\isanswerkey{0}
\fi

\if\isanswerkey1
\def\issolution{1}
\fi


\newcommand{\getcourseyear}[1]{2022}
\newcommand{\getcourseterm}{Spring \getcourseyear{1}}

\newcommand{\getclassmonthnum}[1]{\ifthenelse{#1<16}{3}{\ifthenelse{#1<29}{4}{5}}}
\newcommand{\getclassmonthshort}[1]{\ifthenelse{#1<16}{Mar}{\ifthenelse{#1<29}{Apr}{May}}}
\newcommand{\getclassmonth}[1]{\ifthenelse{#1<16}{March}{\ifthenelse{#1<29}{April}{May}}}
\newcommand{\getclassdayofmonth}[1]{\ifthenelse{
#1=1}{7}{\ifthenelse{
#1=2}{8}{\ifthenelse{
#1=3}{10}{\ifthenelse{
#1=4}{11}{\ifthenelse{
#1=5}{14}{\ifthenelse{
#1=6}{15}{\ifthenelse{
#1=7}{17}{\ifthenelse{
#1=8}{18}{\ifthenelse{
#1=9}{21}{\ifthenelse{
#1=10}{22}{\ifthenelse{
#1=11}{24}{\ifthenelse{
#1=12}{25}{\ifthenelse{
#1=13}{28}{\ifthenelse{
#1=14}{29}{\ifthenelse{
#1=15}{31}{\ifthenelse{
#1=16}{1}{\ifthenelse{
#1=17}{4}{\ifthenelse{
#1=18}{5}{\ifthenelse{
#1=19}{7}{\ifthenelse{
#1=20}{8}{\ifthenelse{
#1=21}{18}{\ifthenelse{
#1=22}{19}{\ifthenelse{
#1=23}{21}{\ifthenelse{
#1=24}{22}{\ifthenelse{
#1=25}{25}{\ifthenelse{
#1=26}{26}{\ifthenelse{
#1=27}{28}{\ifthenelse{
#1=28}{29}{\ifthenelse{
#1=29}{2}{\ifthenelse{
#1=30}{3}{\ifthenelse{
#1=31}{5}{\ifthenelse{
#1=32}{6}{\ifthenelse{
#1=33}{9}{\ifthenelse{
#1=34}{10}{\ifthenelse{
#1=35}{12}{\ifthenelse{
#1=36}{13}{\ifthenelse{
#1=37}{16}{\ifthenelse{
#1=38}{17}{\ifthenelse{
#1=39}{19}{\ifthenelse{
#1=40}{20}{}}}}}}}}}}}}}}}}}}}}}}}}}}}}}}}}}}}}}}}}}

\newcommand{\getclassdate}[1]{\getclassmonth{#1}\xspace\getclassdayofmonth{#1}}
\newcommand{\getclassdateshort}[1]{\getclassmonthshort{#1}\xspace\getclassdayofmonth{#1}}
\newcommand{\getclassdatenum}[1]{\getclassmonthnum{#1}/\getclassdayofmonth{#1}}
\newcommand{\getMday}{Mon}
\newcommand{\getTday}{Tue}
\newcommand{\getWday}{Wed}
\newcommand{\getRday}{Thu}
\newcommand{\getFday}{Fri}
\newcommand{\getclassdayofweek}[1]{\ifthenelse{
#1=1}{\getMday}{\ifthenelse{#1=2}{\getTday}{\ifthenelse{#1=3}{\getRday}{\ifthenelse{#1=4}{\getFday}{\ifthenelse{
#1=5}{\getMday}{\ifthenelse{#1=6}{\getTday}{\ifthenelse{#1=7}{\getRday}{\ifthenelse{#1=8}{\getFday}{\ifthenelse{
#1=9}{\getMday}{\ifthenelse{#1=10}{\getTday}{\ifthenelse{#1=11}{\getRday}{\ifthenelse{#1=12}{\getFday}{\ifthenelse{
#1=13}{\getMday}{\ifthenelse{#1=14}{\getTday}{\ifthenelse{#1=15}{\getRday}{\ifthenelse{#1=16}{\getFday}{\ifthenelse{
#1=17}{\getMday}{\ifthenelse{#1=18}{\getTday}{\ifthenelse{#1=19}{\getRday}{\ifthenelse{#1=20}{\getFday}{\ifthenelse{
#1=21}{\getMday}{\ifthenelse{#1=22}{\getTday}{\ifthenelse{#1=23}{\getRday}{\ifthenelse{#1=24}{\getFday}{\ifthenelse{
#1=25}{\getMday}{\ifthenelse{#1=26}{\getTday}{\ifthenelse{#1=27}{\getRday}{\ifthenelse{#1=28}{\getFday}{\ifthenelse{
#1=29}{\getMday}{\ifthenelse{#1=30}{\getTday}{\ifthenelse{#1=31}{\getRday}{\ifthenelse{#1=32}{\getFday}{\ifthenelse{
#1=33}{\getMday}{\ifthenelse{#1=34}{\getTday}{\ifthenelse{#1=35}{\getRday}{\ifthenelse{#1=36}{\getFday}{\ifthenelse{
#1=37}{\getMday}{\ifthenelse{#1=38}{\getTday}{\ifthenelse{#1=39}{\getRday}{\ifthenelse{#1=40}{\getFday}\xspace
}}}}}}}}}}}}}}}}}}}}}}}}}}}}}}}}}}}}}}}}
\newcommand{\examdate}[1]{\ifthenelse{#1=1}{March 23, \getcourseyear{11}}{\ifthenelse{#1=2}{April 7, \getcourseyear{11}}{\ifthenelse{#1=3}{May 4, \getcourseyear{11}}{\ifthenelse{#1=4}{ } {}}}}}




\newcommand{\classdate}[1]{\getclassdate{#1}, \getcourseyear{#1}}
\newcommand{\dueclassdate}{\getclassdayofweek{\dueclassnumber} \getclassdateshort{\dueclassnumber}}


\newcommand\largeemptyspace{\vphantom{\textnormal{$\ds\int$}}}
\newcommand\nameblank{\if\issolution0\underline{\hskip11.25cm {\largeemptyspace}}
  \else\underline{\hskip.2cm{\LARGE\myname}\hskip6cm}\fi}
\newcommand\nameblankshort{\if\issolution0\underline{\hskip9.5cm {\largeemptyspace}}
  \else\underline{\hskip.2cm{\LARGE\myname {\largeemptyspace}}\hskip.2cm}\fi}
\newcommand\secblank{\if\issolution0\underline{\hskip1.5cm{\largeemptyspace}}
  \else\underline{\hskip.2cm{\LARGE\mysec {\largeemptyspace}}\hskip.2cm} \hskip.8cm \fi}
\newcommand\CMblank{\if\issolution0\underline{\hskip2.25cm{\largeemptyspace}}
  \else\underline{\hskip.2cm{\LARGE\myCM {\largeemptyspace}}\hskip.2cm}\fi}

\newcommand{\namegroupline}{Name: \nameblank Group \#: \underline{\hskip1.5cm{\largeemptyspace}}}
\newcommand{\nameline}{\begin{minipage}{0.6\linewidth} Name: \nameblank \end{minipage}}
\newcommand{\namelineshort}{Name: \nameblankshort}
\newcommand{\namesecline}{\begin{minipage}{0.7\linewidth} Name: \nameblank \end{minipage} \hfill \begin{minipage}{0.29\linewidth}Section \#: \secblank\end{minipage}}
\newcommand{\namesecCMline}{\begin{minipage}{0.6\linewidth}Name: \nameblankshort \end{minipage} \hfill \begin{minipage}{0.4\linewidth} Section \# \secblank CM\# \CMblank \end{minipage}}
\newcommand{\keyline}{{\color{red} SOLUTION KEY}}
%\newcommand{\nameline}{Name: \rule{11.5cm}{0.01cm} \hfill Section: \rule{1.5cm}{0.01cm}}
%\newcommand{\keyline}{Name: \rule{4cm}{0.01cm} SOLUTION KEY \rule{4cm}{0.01cm} \hfill Section: \rule{1.5cm}{0.01cm}}
%\newcommand{\groupline}{Group members present: \rule{8.5cm}{0.01cm} \hfill Group \#: \rule{1.5cm}{0.01cm}}
\newcommand{\course}{CSSE/MA 474\xspace}
\newcommand{\coursewithname}{CSSE/MA 474. Theory of Computation\xspace}


\newcommand{\wtitlestuff}{
\if\isanswerkey0
  \nameline
  \else
  \keyline
\fi
\begin{center}
\large \course Worksheet for Class \#\classnumber\\
\small \classdate{\classnumber}
\normalsize
\end{center}}

\newcommand{\lectitlestuff}{
\begin{center}
\Large \course Lecture \#\classnumber\\
\vskip 3pt \small Nate Chenette \\ \classdate{\classnumber}
\normalsize
\end{center}}

\newcommand{\othertitlestuff}{
\begin{center}
\Large \othertitle\\
\small \coursewithname\\
Class \#\classnumber, \classdate{\classnumber}\\
\normalsize
\end{center}}

\newcommand{\othertitlestuffnodate}{
\begin{center}
\Large \othertitle\\
\small \coursewithname\\
\normalsize
\end{center}}

\newcommand{\othernametitlestuff}{
\nameline\\
\othertitlestuff
}

\newcommand{\assignmenttitlestuff}{
\begin{center}
\Large \course Assignment \assignmentnum\\
\small Due date: \dueclassdate
\normalsize
\end{center}}

\newcommand{\assignmentnametitlestuff}{
\if\isanswerkey0
  \namesecCMline
  \else
  \keyline
\fi
\assignmenttitlestuff
}

\newcommand{\quiznametitlestuff}{
\if\isanswerkey0
  \namesecCMline
  \else
  \keyline
\fi
\begin{center}
\Large \course Quiz \quiznum\\
\small \classdate{\classnumber}
\normalsize
\end{center}}


\setlength{\parindent}{0in}
\setlength{\fboxsep}{.1in}

\renewcommand{\emptyset}{\varnothing}
\newcommand{\tvs}{\textvisiblespace}
\newcommand{\brk}{\vskip.2cm \hrule \vskip.2cm}
\newcommand{\ds}{\displaystyle}
\newcommand{\abs}[1]{\left\lvert {#1}\right\rvert}
\newcommand{\Lsym}{\text{L}}
\newcommand{\Rsym}{\text{R}}
\newcommand{\qacc}{q_{\textnormal{accept}}}
\newcommand{\qrej}{q_{\textnormal{reject}}}
\newcommand{\tmRej}{$\to$ \textbf{\textit{reject}}}
\newcommand{\tmAcc}{$\to$ \textbf{\textit{accept}}}
\def\lep{\le_\textnormal{P}}
\def\lem{\le_\textnormal{m}}
\def\ATM{A_\textnormal{TM}}
\newcommand{\vv}[2]{\begin{bmatrix} {#1} \cr {#2} \end{bmatrix}}
\newcommand{\vvt}[2]{\begin{bmatrix} {\tt #1} \cr {\tt #2} \end{bmatrix}}
\def\hs{\quad \texttt\#\quad }
\def\multiset#1#2{\ensuremath{\left(\kern-.3em\left(\genfrac{}{}{0pt}{}{#1}{#2}\right)\kern-.3em\right)}}
\def\time{\textsf{TIME}}
\def\ntime{\textsf{NTIME}}
\def\P{\textsf{P}}
\def\NP{\textsf{NP}}
   
%logic
\newcommand{\se}{\big|}
\newcommand{\lra}{\leftrightarrow}
\newcommand{\Lra}{\Leftrightarrow}
\newcommand{\we}{\wedge}
\def\thf{%
   \leavevmode
   \lower0.2ex\hbox{$\cdot$}%
   \kern-0.0em\raise0.7ex\hbox{$\cdot$}%
   \kern-0.0em\lower0.2ex\hbox{$\cdot$}%
   \thinspace}

%Number Systems
\newcommand{\bbZ}{\mathbb{Z}}
\newcommand{\bfZ}{\mathbf{Z}}
\newcommand{\bfZp}{\mathbf{Z}^+}
\newcommand{\bbN}{\mathbb{N}}
\newcommand{\bfN}{\mathbf{N}}
\newcommand{\bbQ}{\mathbb{Q}}
\newcommand{\bfQ}{\mathbf{Q}}
\newcommand{\bbR}{\mathbb{R}}
\newcommand{\bfR}{\mathbf{R}}
\newcommand{\bbC}{\mathbb{C}}
\newcommand{\bfC}{\mathbf{C}}

%sets
\newcommand{\U}{\mathscr{U}}
\newcommand{\ol}[1]{\overline{#1}}
\newcommand{\ssq}{\subseteq}
\newcommand{\sst}{\subset}
\def\ps{\mathcal{P}}
\def\sd{\,\triangle\,}
\def\sdonly{\triangle}
\def\es{\emptyset}

%cards
\def\hst{\heartsuit}
\def\cst{\clubsuit}
\def\sst{\spadesuit}
\def\dst{\diamondsuit}


\newcommand{\lcm}{{\rm lcm}}


\newcommand{\imgdir}{../images/}


\newcommand{\makeexamcover}{
\ifdefined\finalexam
\ \vskip2cm
\begin{center}
\huge \course Final Exam \\
\Large \finalexamdate \vskip1cm
\end{center}
\normalsize \instructions \vskip1cm
\begin{spacing}{1.5}
\begin{center}
\scorechart
\end{center}
\end{spacing}
\else \ifdefined\examnum
\ \vskip2cm
\begin{center}
\huge \course Exam \examnum \\
\Large \examdate{\examnum} \vskip1cm
\end{center}
\normalsize \instructions \vskip1cm
\begin{spacing}{1.5}
\begin{center}
\scorechart
\end{center}
\end{spacing}
\fi
\fi
}




\fancypagestyle{examcover}{% 
\fancyhf{}
\renewcommand{\footrulewidth}{0pt}
\lhead{\if\isanswerkey1{\keyline}\else{\nameline}\fi}
%\lhead{\if\isanswerkey1{\keyline}\else{\namesecline}\fi}
}



\fancypagestyle{exameverypage}{% 
\fancyhf{}
\renewcommand{\footrulewidth}{0pt}
\rhead{\if\isanswerkey1{\keyline}\else{}\fi}
\fancyfoot[R]{\thepage}
%\lhead{\if\isanswerkey1{\keyline}\else{\namesecline}\fi}
}

\newcommand{\definition}[1]{{\sc Definition}.~~{#1}\vskip.2cm}

\usepackage[framemethod=default]{mdframed}
\global\mdfdefinestyle{red1}{linecolor=red, linewidth=1pt, leftmargin=1cm, rightmargin=1cm}
\global\mdfdefinestyle{black1}{linecolor=black, linewidth=1pt,} %leftmargin=.1cm, rightmargin=.1cm}

\newcommand{\solution}[2][]{\if\issolution0 #1 \else \begin{mdframed}[style=black1] #2 \end{mdframed} \fi}

\newcommand{\cmblanka}[1]{\if\issolution0 	\underline{\hskip1cm{\largeemptyspace}}
\else 						  		\underline{\hskip.35cm {#1}\hskip.35cm{\largeemptyspace}}\fi}
\newcommand{\sblanka}[1]{\if\issolution0 		\underline{\hskip1.5cm{\largeemptyspace}}
\else 						  		\underline{\hskip.25cm {#1}\hskip.25cm{\largeemptyspace}}\fi}
\newcommand{\mblanka}[1]{\if\issolution0 	\underline{\hskip3cm{\largeemptyspace}}
\else 						  		\underline{\hskip.5cm {#1}\hskip.5cm{\largeemptyspace}}\fi}
\newcommand{\lblanka}[1]{\if\issolution0 		\underline{\hskip4.5cm{\largeemptyspace}}
\else 						  		\underline{\hskip.75cm {#1}\hskip.75cm{\largeemptyspace}}\fi}
\newcommand{\Lblanka}[1]{\if\issolution0		\underline{\hskip6cm{\largeemptyspace}}
\else 								\underline{\hskip1cm {#1}\hskip1cm{\largeemptyspace}}\fi}
\newcommand{\LLblanka}[1]{\if\issolution0	\underline{\hskip7.5cm{\largeemptyspace}}
\else 								\underline{\hskip1.25cm {#1}\hskip1.25cm{\largeemptyspace}}\fi}
\newcommand{\LLLblanka}[1]{\if\issolution0 	\underline{\hskip9cm{\largeemptyspace}}
\else 								\underline{\hskip1.5cm {#1}\hskip1.5cm{\largeemptyspace}}\fi}
\newcommand{\tinyspacea}[1]{\if\issolution0 	\hskip.2cm{\largeemptyspace}
\else 						  		{#1}{\largeemptyspace}\fi}
\newcommand{\cmspacea}[1]{\if\issolution0 	\hskip1cm{\largeemptyspace}
\else 						  		\hskip.15cm {#1}\hskip.15cm{\largeemptyspace}\fi}
\newcommand{\sspacea}[1]{\if\issolution0 		\hskip1.5cm{\largeemptyspace}
\else 						  		\hskip.25cm {#1}\hskip.25cm{\largeemptyspace}\fi}
\newcommand{\mspacea}[1]{\if\issolution0 	\hskip3cm{\largeemptyspace}
\else 						  		\hskip.25cm {#1}\hskip.25cm{\largeemptyspace}\fi}
\newcommand{\lspacea}[1]{\if\issolution0 		\hskip4.5cm{\largeemptyspace}
\else 						  		\hskip.25cm {#1}\hskip.25cm{\largeemptyspace}\fi}
\newcommand{\Lspacea}[1]{\if\issolution0 	\hskip6cm{\largeemptyspace}
\else 						  		\hskip.25cm {#1}\hskip.25cm{\largeemptyspace}\fi}


\newcommand{\sparagraph}[1]{\vskip-1cm\paragraph{#1}}

\if\isanswerkey1\input{macsse474-key}\fi
\begin{document}

\assignmentnametitlestuff


\begin{enumerate}


\item (7.6) Show that $\P$ is closed under union, concatenation, and complement. 
\solution{
\if\isanswerkey1\PclosedUnderUnionConcatenationComplement\fi
\begin{itemize}
    \item \textbf{Union}: Consider language $A, B$ such that $A \in \P$ and $B \in \P$. So that there exists a decider $D_\textnormal{A}$ and $D_\textnormal{B}$ that decides language $A$ and $B$ in polynomial time. For language $A \cup B$, we can build a decider $D_{\textnormal{A} \cup \textnormal{B}}$ that deas follow:\\
    $D_{\textnormal{A} \cup \textnormal{B}}=$ ``On input $w$,
    \begin{enumerate}
        \item Run $D_\textnormal{A}$ on input $w$
        \begin{enumerate}
            \item If $D_\textnormal{A}$ accepts $w$ \tmAcc
            \item If rejects, run $D_\textnormal{B}$ on input $w$
            \begin{enumerate}
                \item If $D_\textnormal{B}$ accepts $w$ \tmAcc
                \item Otherwise \tmRej''
            \end{enumerate}
        \end{enumerate}
    \end{enumerate}
    Since $A, B \in \P$, so that decider $D_\textnormal{A}$ and $D_\textnormal{B}$ both runs in polynomial time. So for decider $D_{\textnormal{A} \cup \textnormal{B}}$, step $a, ii \to O(n^k)$, step $i, \textnormal{A, B} \to O(1)$. So for total, it runs in $O\left(n^k\right) + O(1) + O\left(n^k\right) + 2 \times O(1)=O\left(n^k\right)$, which makes decider $D_{\textnormal{A} \cup \textnormal{B}}$ runs in polynomial time. Hence $A \cup B \in \P$.
    \item \textbf{Concatenation}: For language $A, B \in \P$, there exist a decider $D_\textnormal{A}, D_\textnormal{B}$ that decides $A, B$ in polynomial time. So that we can build a decider $D_\textnormal{AB}$ that decides language $AB$ in polynomial time as follow:\\
    $D_\textnormal{AB}=$ ``On input $w$
    \begin{enumerate}
        \item Split the string $w$ into $x$ and $y$ such that $s=xy$ and $x, y \in \Sigma^+$
        \item For each split case, repeat until all split cases has been run
        \begin{enumerate}
            \item Run $x$ in $D_\textnormal{A}$ and run $y$ in $D_\textnormal{B}$
            \begin{enumerate}
                \item If $D_\textnormal{A}$ accept $x$ and $D\textnormal{B}$ accepts $y$ \tmAcc
            \end{enumerate}
        \end{enumerate}
        \item If none of the case has been accepted \tmRej''
    \end{enumerate}
    For the decider $D_\textnormal{AB}$, steps $a, A, c \to O(1)$, step $b \to O(n)$ and step $i \to O\left(n^k\right)$, so that total run time is $O(1) + O(n)\times O\left(n^k\right) \times O(1) + O(1) = O\left(n^k\right)$, which makes the decider $D_\textnormal{AB}$ decides language $AB$ in polynomial time. Hence $AB \in \P$
    \item \textbf{Complement}: For a language $A \in \P$, there exists a decider $D_\textnormal{A}$ such that decides language $A$ in polynomial time. So that we can build a decider $D_\overline{\textnormal{A}}$ that decides language $\overline{A}$ as follow:
    $D_\overline{\textnormal{A}}=$ ``On input $w$,
    \begin{enumerate}
        \item Run $w$ in $D_\textnormal{A}$
        \begin{enumerate}
            \item If it $D_\textnormal{A}$ accepts $w$ \tmRej
            \item Otherwise \tmAcc''
        \end{enumerate}
    \end{enumerate}
    So that for decider $D_\overline{\textnormal{A}}$, its step $a \to O\left(n^k\right)$ and steps $i, ii \to O(1)$. So that total run-time is $O\left(n^k\right) + 2\times O(1)=O\left(n^k\right)$. Hence $\overline{A} \in \P$
\end{itemize}
}

\item (7.9) A {\bf triangle} in an undirected graph is a 3-clique. Show that $TRIANGLE \in \P$, where
\[TRIANGLE = \{\langle G \rangle \mid G \textnormal{ is an undirected graph and $G$ contains a triangle}\}.\]
\solution{
\if\isanswerkey1\TriangleInP\fi
For language $TRIANGLE$, we can build a decider $D_\textnormal{TRI}$ that accept language $TRIANGLE$ as follow:\\
$D_\textnormal{TRI}=$ ``On input $\langle G \rangle$,
\begin{enumerate}
    \item Start at a starting node $n_\textnormal{S}$ and mark it as \texttt{a}
    \item Repeat following until all nodes has been marked as \texttt{x}
    \begin{enumerate}
        \item For all nodes $n_i$ reachable from the node marked with \texttt{a} that has not been marked with \texttt{x} or \texttt{a} do the following steps
        \begin{enumerate}
            \item Mark it with \texttt{y}
            \item If there is a node reachable from the $n_i$ that has been marked with \texttt{y} \tmAcc
        \end{enumerate}
        \item Pick one node marked with \texttt{y} and mark it with \texttt{a}
        \item Mark the node marked with \texttt{a} with \texttt{x}''
    \end{enumerate}
\end{enumerate}
For the decider $D_\textnormal{TRI}$, its steps $a, A, iii \to O(1)$, steps $b, i, B, ii \to O(n)$. So that the total run time is $O(1) + O(n) \times \left(O(n) \times (1 + O(n)) + O(n) + O(1)) \right)=O\left(n^3\right)$, which makes the decider $D_\textnormal{TRI}$ runs in polynomial time. Hence $TRIANGLE \in \P$
}

\item (7.10) Show that $ALL_\textnormal{DFA}$ is in $\P$.
\solution{
\if\isanswerkey1\ALLDFAInP\fi
For language $ALL_\textnormal{DFA}$, we can build a decider $D_\textnormal{ALL-DFA}$ that decides it, where machine $D_\textnormal{All-DFA}$ is defined as:\\
$D_\textnormal{All-DFA}=$``On input $\langle A\rangle$, $A$ is a DFA
\begin{enumerate}
    \item Mark the start state of $A$, $q_0$
    \item Check if $q_0$ is accept state, if not $\to$ \textbf{\textit{reject}}
    \begin{enumerate}
        \item Mark all new states reachable from all marked state
        \item If any one of new marked state is not an accept state $\to$ \textbf{\textit{reject}}
    \end{enumerate}
    \item If there is new state marked, repeat steps i$-$ii
    \item 
    \begin{enumerate}
        \item If all marked states are accept state $\to$ \textbf{\textit{accept}}
        \item Otherwise $\to$ \textbf{\textit{reject}}''
    \end{enumerate}
\end{enumerate}
For the decider $D_\textnormal{ALL-DFA}$, its steps $a, b \to O(1)$, steps $b_{ii}, c, d_i, d_{ii} \to O(n)$, steps $b_i \to O\left(n^2\right)$. So total run-time is $O(1) + O(1) + O(n) \times \left( O(n) + O\left(n^2\right) \right) + O(n) + O(n) = O\left(n^3\right)$. This makes the decider $D_\textnormal{ALL-DFA}$ runs in polynomial time. Hence $ALL_\textnormal{DFA} \in \P$
}

\item (7.11a) Show that $EQ_\textnormal{DFA}$ is in $\P$.
\solution{
\if\isanswerkey1\EQDFAInP\fi
According \textbf{Theorem 4.4, 4.5}, there is a decider $D_\textnormal{EQ-DFA}$ that decides language $EQ_\textnormal{DFA}$:\\
$D_\textnormal{EQ-DFA}=$ ``On input $\langle M_1, M_2\rangle$,
\begin{enumerate}
    \item Construct a DFA $M$ such that $L(M)=\left(L(A) \cap \overline{L(B)}\right) \cup \left(\overline{L(A)} \cap L(B)\right)$
    \item Mark the start state of $M$
    \item Repeat until no new states marked
    \begin{enumerate}
        \item Mark all new states from the states that already has been marked
    \end{enumerate}
    \item If no accept state has been marked \tmAcc
    \item Otherwise \tmRej''
\end{enumerate}
For this decider $D_\textnormal{EQ-DFA}$, the steps $a, b \to O(1)$, steps $c, d, e \to O(n)$, step $c_i \to O\left(n^2\right)$. So that the overall runtime is $O(1) + O(1) + O(n) \times O\left(n^2\right) + O(n) + O(n) = O\left(n^3\right)$, so that the decider $D_\textnormal{EQ-DFA}$ runs in in polynomial time. Hence $EQ_\textnormal{DFA} \in \P$
}


\item (7.7) Show that $\NP$ is closed under union and concatenation.
\solution{
\if\isanswerkey1\NPclosedUnderUnionConcatenation\fi
\begin{itemize}
    \item \textbf{Union}: For language $A, B \in \NP$, there exist the decider $D_\textnormal{A}$ and $D_\textnormal{B}$ that decides language $A$ and $B$ nondeterministically in polynomial time. So that we can build a decider $D_{\textnormal{A} \cup \textnormal{B}}$ that decides language $A \cup B$ as follow:\\
    $D_{\textnormal{A} \cup \textnormal{B}}=$ ``On input $w$
    \begin{enumerate}
        \item Nondeterminstically run $D_\textnormal{A}$ on input $w$
        \begin{enumerate}
            \item If $D_\textnormal{A}$ accepts $w$ \tmAcc
            \item Otherwise run $D_\textnormal{B}$ on input $w$
            \begin{enumerate}
                \item If $D_\textnormal{B}$ accepts $w$ \tmAcc
                \item Otherwise \tmRej''
            \end{enumerate}
        \end{enumerate}
    \end{enumerate}
    Since $D_\textnormal{A}, D_\textnormal{B} \in \NP$, $D_\textnormal{A}$ and $D_\textnormal{B}$ all runs nondeterminstically on input $w$ at polynomial time, so that decider $D_{\textnormal{A} \cup \textnormal{B}}$ also runs nondeterminstically in polynomial time. Hence $A\cup B \in \NP$
    \item \textbf{Concatenation}: For language $A, B \in \NP$, there exist the decider $D_\textnormal{A}$ and $D_\textnormal{B}$ that decides language $A$ and $B$ nondeterministically in polynomial time. So that we can build a decider $D_\textnormal{AB}$ that decides the language $AB$:\\
    $D_\textnormal{AB}=$ ``On input $w$
    \begin{enumerate}
        \item Split $w$ into $x$ and $y$ such that $w=xy$ and $x, y \in \Sigma^+$
        \item For each split of $x, y$ repeat the following steps:
        \begin{enumerate}
            \item Run $D_\textnormal{A}$ on input $x$
            \item If $D_\textnormal{A}$ reject $x$ $\to \textnormal{continue}$
            \item Run $D_\textnormal{B}$ on input $y$
            \item If $D_\textnormal{B}$ accept $y$ \tmAcc
            \item Otherwise $\to \textnormal{continue}$''
        \end{enumerate}
        \item If no split has been accepted \tmRej
    \end{enumerate}
\end{itemize}
For decider $D_\textnormal{AB}$, since they all runs nondeterminstically in polynomial time, and the total time of the $D_\textnormal{AB}$ is $O(n)\times(\textsc{Time}\left(D_\textnormal{A}\right) + \textsc{Time}\left(D_\textnormal{B}\right))$, which is also polynomial time. And since $D_\textnormal{A}$ and $D_\textnormal{B}$ runs nondeterminstically, so that $D_\textnormal{AB}$ runs nondeterminstically. Hence $AB \in \NP$
}

\end{enumerate}




\end{document}
